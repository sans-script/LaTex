\documentclass[a4paper,12pt]{article}  % Classe do documento

\usepackage[utf8]{inputenc}            % Codificação do documento
\usepackage[T1]{fontenc}               % Encodings para caracteres especiais
\usepackage[brazil]{babel}             % Idioma do documento
\usepackage{geometry}                  % Para ajustar margens
\usepackage{graphicx}                  % Para incluir imagens
\usepackage{amsmath, amsfonts, amsthm, amssymb} % Para símbolos matemáticos
\usepackage{fancyhdr}                  % Para cabeçalhos e rodapés
\usepackage[dvipsnames]{xcolor}        % Para cores
\usepackage{thmtools}                  % Para definições de teoremas
\usepackage{titlesec}                  % Para personalizar títulos de seções
\usepackage{booktabs}                  % Adicione isso no preâmbulo
\usepackage{tabularx}                  % Para tabelas com largura ajustável
\usepackage{mathptmx}                  % Para usar Times New Roman
\usepackage{enumitem} % Para personalizar a lista
\usepackage{cancel} % Para cancelar termos em equações
\usepackage[version=4]{mhchem} % Para escrever fórmulas químicas
\usepackage{chemfig} % Para desenhar moléculas
\usepackage{colortbl} % Para colorir células de tabelas
\usepackage{multicol} % Para dividir o texto em colunas
\usepackage{tikz} % Para desenhar gráficos
\usepackage{tikz}
\usepackage{pgf-spectra}
\usepackage{siunitx}


% Configuração das margens
\geometry{top=2cm, bottom=2cm, left=1cm, right=1cm}

\titleformat{\section}
{\normalfont\Large\bfseries}{\thesection}{0.6em}{}

\titleformat{\subsection}
{\normalfont\large\bfseries}{\thesubsection}{0.3em}{}

\setcounter{secnumdepth}{0}  % Remove a numeração de seções
\def\checkmark{\tikz\fill[scale=0.4](0,.35) -- (.25,0) -- (1,.7) -- (.25,.15) -- cycle;}
\definecolor{indigo}{rgb}{0.29, 0.0, 0.51}

% Configura cabeçalhos e rodapés
\pagestyle{fancy}
\fancyhf{} % Limpa cabeçalhos e rodapés
\fancyhead[R]{Exercícios de Fixação} % Nome à direita no cabeçalho
\fancyhead[L]{Alexandre Santos}
\fancyfoot[R]{\thepage} % Número da página à direita no rodapé

\begin{document}



\section{Conjuntos Numéricos}

Os conjuntos numéricos constituem uma das fundações da matemática, formando a base sobre a qual diversas estruturas numéricas e teorias são edificadas. Eles representam coleções de números que compartilham características comuns e são categorizados de acordo com suas propriedades. Por exemplo, os números naturais são utilizados para contagem, enquanto os números inteiros incluem tanto positivos quanto negativos, e os números racionais são expressos como frações. Além disso, os conjuntos numéricos permitem a análise e a resolução de problemas matemáticos complexos, sendo essenciais para áreas como álgebra, cálculo e teoria dos números

\section{Interpretação de Símbolos de Conjuntos}

Os símbolos utilizados para descrever conjuntos numéricos são essenciais para a compreensão de conceitos matemáticos. Abaixo, apresentamos esses símbolos em forma de tabela:

\begin{table}[h]
    \centering
    \begin{tabularx}{\textwidth}{|c|X|}
        \hline
        \textbf{Símbolo}      & \textbf{Como se lê}             \\
        \hline
        \( \{ \} \)           & conjunto                        \\
        \( \in \)             & pertence a                      \\
        \( \notin \)          & não pertence a                  \\
        \( \subseteq \)       & é um subconjunto de             \\
        \( \subset \)         & é um subconjunto próprio de     \\
        \( \supseteq \)       & contém como subconjunto         \\
        \( \supset \)         & contém como subconjunto próprio \\
        \( \varnothing \)     & conjunto vazio                  \\
        \( \cup \)            & união                           \\
        \( \cap \)            & interseção                      \\
        \( - \)               & diferença (ou complementar)     \\
        \( \mid \) ou \( : \) & tal que                         \\
        \( \mathbb{N} \)      & conjunto dos números naturais   \\
        \( \mathbb{Z} \)      & conjunto dos números inteiros   \\
        \( \mathbb{Q} \)      & conjunto dos números racionais  \\
        \( \mathbb{R} \)      & conjunto dos números reais      \\
        \( \mathbb{C} \)      & conjunto dos números complexos  \\
        \( \forall \)         & para todo                       \\
        \( \exists \)         & existe                          \\
        \( \neg \)            & não                             \\
        \( \land \)           & e                               \\
        \( \lor \)            & ou                              \\
        \( \Rightarrow \)     & implica                         \\
        \( \Leftrightarrow \) & se e somente se                 \\
        \( < \)               & menor que                       \\
        \( \leq \)            & menor ou igual a                \\
        \( > \)               & maior que                       \\
        \( \geq \)            & maior ou igual a                \\
        \( \infty \)          & infinito                        \\
        \( \neq \)            & diferente de                    \\
        \( \approx \)         & aproximadamente igual a         \\
        \( \equiv \)          & identicamente igual a           \\
        \hline
    \end{tabularx}
    \caption{Interpretação de símbolos de conjuntos}
\end{table}


\subsection{Conjunto dos Números Naturais}
O conjunto dos números naturais é denotado por \( \mathbb{N} \) e é composto por todos os números inteiros não negativos, utilizados predominantemente para contagem e ordenação de objetos.

\begin{itemize}
    \item \textbf{Notação}: \( \mathbb{N} = \{0, 1, 2, 3, \ldots\} \)
    \item \textbf{Propriedades}:
          \begin{itemize}
              \item Inclui o zero, dependendo da definição utilizada (em algumas interpretações, o conjunto começa em 1).
              \item É um subconjunto dos números inteiros, o que implica que todos os números naturais são, por definição, números inteiros.
              \item É fechado sob as operações de adição e multiplicação, significando que a soma ou o produto de quaisquer dois números naturais resulta em um número natural.
          \end{itemize}
\end{itemize}

\subsection{Conjunto dos Números Inteiros}
O conjunto dos números inteiros é denotado por \( \mathbb{Z} \) e se expande ao conjunto dos números naturais ao incluir os números negativos.

\begin{itemize}
    \item \textbf{Notação}: \( \mathbb{Z} = \{\ldots, -3, -2, -1, 0, 1, 2, 3, \ldots\} \)
    \item \textbf{Propriedades}:
          \begin{itemize}
              \item Abrange todos os números naturais e seus correspondentes negativos, formando um sistema numérico que permite a inclusão de operações de subtração.
              \item Não inclui frações ou números decimais, sendo restrito a inteiros.
              \item É fechado sob as operações de adição, subtração e multiplicação, mas não é fechado sob a divisão (por exemplo, \( 1 \div 2 \) não resulta em um número inteiro).
          \end{itemize}
\end{itemize}

\subsection{Conjunto dos Números Racionais}
Os números racionais são aqueles que podem ser expressos como a razão de dois inteiros, onde o denominador não é zero.

\begin{itemize}
    \item \textbf{Notação}: \( \mathbb{Q} = \left\{ \frac{a}{b} \mid a \in \mathbb{Z}, b \in \mathbb{Z}^*, b \neq 0 \right\} \)
    \item \textbf{Propriedades}:
          \begin{itemize}
              \item Inclui números inteiros (por exemplo, o número 2 pode ser representado como \( \frac{2}{1} \)).
              \item Compreende números que podem ser expressos como frações, abrangendo também decimais que terminam ou que se repetem periodicamente.
              \item É fechado sob adição, subtração, multiplicação e divisão (desde que o divisor seja diferente de zero).
          \end{itemize}
\end{itemize}

\subsection{Conjunto dos Números Irracionais}
Os números irracionais são aqueles que não podem ser expressos como uma fração de dois inteiros, caracterizando-se por suas representações decimais não periódicas e infinitas.

\begin{itemize}
    \item \textbf{Exemplos}: \( \sqrt{2}, \pi, e \) (a base do logaritmo natural).
    \item \textbf{Propriedades}:
          \begin{itemize}
              \item Não estão incluídos no conjunto dos números racionais, ou seja, não podem ser expressos como uma razão de inteiros.
              \item Formam um conjunto denso nos números reais, o que significa que, entre quaisquer dois números racionais, existe sempre pelo menos um número irracional.
          \end{itemize}
\end{itemize}

\section{Operações com Conjuntos Numéricos}
As operações entre os conjuntos numéricos respeitam certas propriedades de fechamento. A seguir, são apresentadas algumas operações comuns e suas respectivas propriedades.

\subsection{Adição e Subtração}
\begin{itemize}
    \item Para \( a, b \in \mathbb{N} \): \( a + b \in \mathbb{N} \) e \( a - b \) pode não ser um número natural (por exemplo, \( 1 - 2 = -1 \) não pertence a \( \mathbb{N} \)).
    \item Para \( a, b \in \mathbb{Z} \): tanto \( a + b \in \mathbb{Z} \) quanto \( a - b \in \mathbb{Z} \).
    \item Para \( a, b \in \mathbb{Q} \): \( a + b \in \mathbb{Q} \) e \( a - b \in \mathbb{Q} \).
\end{itemize}

\subsection{Multiplicação}
\begin{itemize}
    \item Para \( a, b \in \mathbb{N} \): \( a \cdot b \in \mathbb{N} \).
    \item Para \( a, b \in \mathbb{Z} \): \( a \cdot b \in \mathbb{Z} \).
    \item Para \( a, b \in \mathbb{Q} \): \( a \cdot b \in \mathbb{Q} \).
\end{itemize}

\subsection{Divisão}
\begin{itemize}
    \item Para \( a, b \in \mathbb{N} \): \( a \div b \) não é necessariamente um número natural (por exemplo, \( 1 \div 2 \) não pertence a \( \mathbb{N} \)).
    \item Para \( a, b \in \mathbb{Z} \): \( a \div b \) pode não resultar em um número inteiro (por exemplo, \( 1 \div 2 = 0.5 \)).
    \item Para \( a, b \in \mathbb{Q} \): \( a \div b \in \mathbb{Q} \) (desde que \( b \neq 0 \)).
\end{itemize}

\subsection{Propriedades Aritméticas}

As operações aritméticas obedecem a diversas propriedades importantes, que incluem:

\begin{enumerate}[label=\arabic*.] % Números para os principais itens
    \item Associatividade:
          \begin{enumerate}[label=\alph*)] % Letras para os subitens
              \item \( (a + b) + c = a + (b + c) \)
              \item \( (a \cdot b) \cdot c = a \cdot (b \cdot c) \)
          \end{enumerate}
    \item Comutatividade:
          \begin{enumerate}[label=\alph*)]
              \item \( a + b = b + a \)
              \item \( a \cdot b = b \cdot a \)
          \end{enumerate}
    \item Distributividade:
          \begin{enumerate}[label=\alph*)]
              \item \( a \cdot (b + c) = a \cdot b + a \cdot c \)
          \end{enumerate}
\end{enumerate}

\section{Decomposição em Frações Parciais}

A decomposição em frações parciais é uma técnica que permite reescrever frações que têm polinômios no denominador (parte de baixo) como a soma de frações mais simples. Esse método é muito útil em cálculos de integrais e em outras áreas da matemática.
Imagine que temos uma fração como esta:
\[
\frac{3x + 1}{(x+1)(2x-1)}.
\]
O objetivo é dividir essa fração em partes mais simples, como:
\[
\frac{3x + 1}{(x+1)(2x-1)} = \frac{A}{x+1} + \frac{B}{2x-1},
\]
onde \( A \) e \( B \) são valores que vamos determinar. Vamos aos passos!

\subsection{Passo 1: Multiplicar para eliminar os denominadores}

Multiplicamos ambos os lados da equação pelo denominador completo \( (x+1)(2x-1) \):
\[
3x + 1 = A(2x - 1) + B(x + 1).
\]
Aqui, a ideia é que os denominadores desapareçam, deixando apenas os numeradores para trabalharmos.

\subsection{Passo 2: Expandir e organizar os termos}

Expandimos os termos do lado direito:
\[
A(2x - 1) = 2Ax - A, \quad \text{e} \quad B(x + 1) = Bx + B.
\]
Somando tudo, temos:
\[
3x + 1 = 2Ax - A + Bx + B.
\]
Agora, agrupamos os termos que têm \( x \) e os números sozinhos (constantes):
\[
3x + 1 = (2A + B)x + (-A + B).
\]

\subsection{Passo 3: Comparar os coeficientes}

Nesta etapa, comparamos os coeficientes (os números que multiplicam \( x \) e os números constantes) de ambos os lados. Isso nos dá um sistema de equações:
\begin{align*}
2A + B & = 3, \\
-B - A & = 1.
\end{align*}

\subsection{Passo 4: Resolver o sistema de equações}

Agora, resolvemos esse sistema para encontrar \( A \) e \( B \).

1. Da primeira equação \( 2A + B = 3 \), isolamos \( B \):
\[
B = 3 - 2A.
\]

2. Substituímos \( B = 3 - 2A \) na segunda equação \( -B - A = 1 \):
\[
-(3 - 2A) - A = 1.
\]
Resolvendo:
\[
-3 + 2A - A = 1 \quad \Rightarrow \quad -3 + A = 1 \quad \Rightarrow \quad A = 4.
\]

3. Substituímos \( A = 4 \) de volta em \( B = 3 - 2A \):
\[
B = 3 - 2 \cdot 4 = 3 - 8 = -5.
\]

\subsection{Passo 5: Reescrever a fração original}

Agora que sabemos os valores de \( A \) e \( B \), podemos reescrever a fração original:
\[
\frac{3x + 1}{(x+1)(2x-1)} = \frac{4}{x+1} - \frac{5}{2x-1}.
\]

Essa forma é muito mais fácil de trabalhar, especialmente em integrais!

\subsection{Outros Casos de Decomposição}

Vamos explorar outros tipos de frações e como lidar com elas.

\subsubsection{Fatores Repetidos}

Quando o denominador tem fatores repetidos, como \( \frac{1}{(x-1)^2} \), a decomposição inclui todos os fatores até a potência mais alta. Por exemplo:
\[
\frac{1}{(x-1)^2} = \frac{A}{x-1} + \frac{B}{(x-1)^2}.
\]

\subsubsection{Fatores Quadráticos}

Se o denominador tiver um fator quadrático irreduzível, como \( x^2 + bx + c \) (e \( \Delta < 0 \)), usamos numeradores na forma \( Ax + B \). Por exemplo:
\[
\frac{1}{x^2 + 1} = \frac{Ax + B}{x^2 + 1}.
\]

\subsubsection{Fatores Quadráticos Repetidos}

Se o denominador tem fatores quadráticos repetidos, como \( \frac{1}{(x^2+1)^2} \), a decomposição será:
\[
\frac{1}{(x^2+1)^2} = \frac{Ax + B}{x^2 + 1} + \frac{Cx + D}{(x^2 + 1)^2}.
\]

\subsection{Exercícios Resolvidos}

\paragraph{Exemplo 1.} Decomponha \( \frac{2}{(x-1)^2} \):
\[
\frac{2}{(x-1)^2} = \frac{A}{x-1} + \frac{B}{(x-1)^2}.
\]
Multiplicando por \( (x-1)^2 \):
\[
2 = A(x-1) + B.
\]
Expandindo:
\[
2 = Ax - A + B.
\]
Comparando os coeficientes:
\begin{align*}
A & = 0, \\
B & = 2.
\end{align*}
Portanto:
\[
\frac{2}{(x-1)^2} = \frac{0}{x-1} + \frac{2}{(x-1)^2} = \frac{2}{(x-1)^2}.
\]

\end{document}
