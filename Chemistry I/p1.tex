\documentclass[a4paper, 12pt]{article}
\usepackage[utf8]{inputenc}            % Codificação do documento
\usepackage[T1]{fontenc}               % Encodings para caracteres especiais
\usepackage[brazil]{babel}             % Idioma do documento
\usepackage{geometry}                  % Para ajustar margens
\usepackage{graphicx}                  % Para incluir imagens
\usepackage{amsmath, amsfonts, amsthm, amssymb} % Para símbolos matemáticos
\usepackage{fancyhdr}                  % Para cabeçalhos e rodapés
\usepackage[dvipsnames]{xcolor}        % Para cores
\usepackage{thmtools}                  % Para definições de teoremas
\usepackage{titlesec}                  % Para personalizar títulos de seções
\usepackage{booktabs}                  % Adicione isso no preâmbulo
\usepackage{tabularx}                  % Para tabelas com largura ajustável
\usepackage{mathptmx}                  % Para usar Times New Roman
\usepackage{enumitem} % Para personalizar a lista
\usepackage{cancel} % Para cancelar termos em equações
\usepackage[version=4]{mhchem} % Para escrever fórmulas químicas
\usepackage{chemfig} % Para desenhar moléculas
\usepackage{colortbl} % Para colorir células de tabelas
\usepackage{multicol} % Para dividir o texto em colunas
\usepackage{tikz} % Para desenhar gráficos
\usepackage{tikz}
\usepackage{pgf-spectra}
\usepackage{siunitx}


% Configuração das margens
\geometry{top=2cm, bottom=2cm, left=1cm, right=1cm}

\titleformat{\section}
{\normalfont\Large\bfseries}{\thesection}{0.6em}{}

\titleformat{\subsection}
{\normalfont\large\bfseries}{\thesubsection}{0.3em}{}

\setcounter{secnumdepth}{0}  % Remove a numeração de seções
\def\checkmark{\tikz\fill[scale=0.4](0,.35) -- (.25,0) -- (1,.7) -- (.25,.15) -- cycle;}
\definecolor{indigo}{rgb}{0.29, 0.0, 0.51}

% Configura cabeçalhos e rodapés
\pagestyle{fancy}
\fancyhf{} % Limpa cabeçalhos e rodapés
\fancyhead[R]{Exercícios de Fixação} % Nome à direita no cabeçalho
\fancyhead[L]{Alexandre Santos}
\fancyfoot[R]{\thepage} % Número da página à direita no rodapé
\fancyhead[R]{Atividade Avaliativa}

\begin{document}

\begin{enumerate}
    \item Foi solicitado para três estudantes que fizessem medidas de 2,000 gramas de carbonato de cálcio. Os resultados medidos por cada estudante encontram-se abaixo:

          \begin{center}
              \begin{tabular}{c|c|c}
                  \hline
                  \textbf{Estudante 1} & \textbf{Estudante 2} & \textbf{Estudante 3} \\
                  \hline
                  1,964g               & 1,972g               & 2,000g               \\
                  1,978g               & 1,968g               & 2,003g               \\
                  1,971g               & 1,970g               & 2,008g               \\
              \end{tabular}
          \end{center}

          \textcolor{blue}{
              Para determinar qual estudante foi o mais exato, calculamos a média das medidas de cada estudante e, em seguida, calculamos o erro absoluto em relação ao valor esperado de \textcolor{red}{2,000g}.
          }

          \textcolor{blue}{
              \text{Estudante 1:}
              \[
                  \textcolor{red}{E_1 = \frac{1,964 + 1,978 + 1,971}{3} = 1,971 \ \text{g}}
              \]
              \[
                  \textcolor{red}{\text{Erro Absoluto}_1 = |1,971 - 2,000| = 0,029 \ \text{g}}
              \]
              \text{Estudante 2:}
              \[
                  \textcolor{red}{E_2 = \frac{1,972 + 1,968 + 1,970}{3} = 1,970 \ \text{g}}
              \]
              \[
                  \textcolor{red}{\text{Erro Absoluto}_2 = |1,970 - 2,000| = 0,030 \ \text{g}}
              \]
              \text{Estudante 3:}
              \[
                  \textcolor{red}{E_3 = \frac{2,000 + 2,003 + 2,008}{3} = 2,004 \ \text{g}}
              \]
              \[
                  \textcolor{red}{\text{Erro Absoluto}_3 = |2,004 - 2,000| = 0,004 \ \text{g}}
              \]
          }

          \textcolor{blue}{
              Comparando os erros absolutos com o valor esperado de \textcolor{red}{2,000g}, o Estudante \textcolor{red}{3} foi o mais exato, pois seu erro absoluto é o menor.
          }

    \item Sobre os números quânticos responda:

          \begin{enumerate}
              \item[a)] Considere três átomos hipotéticos: A, B e C. Sabe-se que o átomo C é isótopo do átomo B e é isóbaro de A. O átomo A tem massa 48 e o átomo C tem número de nêutrons 28. Sabendo-se que o número de massa de B é 52, qual o conjunto dos quatro números quânticos do elétron diferenciador da espécie \( B^{+3} \)?
                    \\[2mm]
                    \textcolor{blue}{
                        Sabemos que:
                        \begin{itemize}
                            \item[] O átomo  \textcolor{red}{C} é isótopo de  \textcolor{red}{B}, ou seja, possuem o mesmo número atômico.
                            \item[] O átomo  \textcolor{red}{C} é isóbaro de  \textcolor{red}{A}, ou seja, possuem o mesmo número de massa, mas diferem no número de prótons.
                            \item[] O átomo  \textcolor{red}{A} tem número de massa  \textcolor{red}{48}, logo,  \textcolor{red}{C também tem 48}.
                            \item[] O átomo  \textcolor{red}{C} tem  \textcolor{red}{28 nêutrons}.
                            \item[] O número de massa de  \textcolor{red}{B} é  \textcolor{red}{52}.
                        \end{itemize}
                        Para encontrar o número atômico de  \textcolor{red}{C} (e consequentemente de  \textcolor{red}{B}, pois são isótopos), utilizamos a relação:
                        \[
                            \textcolor{red}{ Z_C = A_C - N_C = 48 - 28 = 20}
                        \]
                        Como  \textcolor{red}{B e C são isótopos},  \textcolor{red}{B também tem número atômico \( Z_B = 20 \)}.
                        Sabemos que o número de prótons em um átomo neutro é igual ao número de elétrons. Assim, um átomo neutro de  \textcolor{red}{B} possui  \textcolor{red}{20 elétrons}.
                        A configuração eletrônica do átomo  \textcolor{red}{B neutro} é:
                        \[
                            \textcolor{red}{1s^2 \ 2s^2 \ 2p^6 \ 3s^2 \ 3p^6 \ 4s^2}
                        \]
                        Para a espécie  \textcolor{red}{\( B^{+3} \)}, removemos  \textcolor{red}{três elétrons} da camada mais externa, ou seja:
                        \begin{itemize}
                            \item[]  \textcolor{red}{2 elétrons do 4s}
                            \item[]  \textcolor{red}{1 elétron do 3p}
                        \end{itemize}
                        Após a remoção, a configuração eletrônica de  \textcolor{red}{\( B^{+3} \)} fica:
                        \[
                            \textcolor{red}{1s^2 \ 2s^2 \ 2p^6 \ 3s^2 \ 3p^5}
                        \]
                        Agora, identificamos o elétron diferenciador. No subnível  \textcolor{red}{3p} (\( l = 1 \)), os elétrons ocupam os orbitais seguindo a  \textcolor{red}{Regra de Hund}:
                        Os três orbitais do \textcolor{red}{3p} têm \textcolor{red}{\( m_l = -1,\ 0,\ +1 \)}. A distribuição correta do  \textcolor{red}{3p\(^5\)} é:
                        \[
    \textcolor{red}{
        \begin{array}{c}
            \begin{tabular}{|m{0.5cm}|m{0.5cm}|m{0.5cm}|}
                \hline
                $\uparrow \downarrow$ & $\uparrow \downarrow$ & $\uparrow$ \\
                \hline
            \end{tabular} \\
            \begin{tabular}{m{0.5cm} m{0.5cm} m{0.5cm}}
                $-1$ & \text{ 0 } & $+1$
            \end{tabular}
        \end{array}
    }
\]
                        O último elétron adicionado ocupa o orbital de  \textcolor{red}{\( m_l = 0 \)}.
                        Dessa forma, os números quânticos do elétron diferenciador da espécie  \textcolor{red}{\( B^{+3} \)} são:
                        \[
                            \textcolor{red}{ n = 3, \quad l = 1, \quad m_l = 0, \quad m_s = -1/2}
                        \]
                    }

              \item[b)] Sabendo-se que o elétron de maior energia de um átomo apresenta o seguinte conjunto dos quatro números quânticos: \( n = 3, \ell = 2, m_\ell = 1, m_s = -1/2 \), quantos elétrons fazem parte do nível de valência desse átomo?
                    \\[2mm]
                    \textcolor{blue}{
                        Sabemos que:
                        \begin{itemize}
                            \item[] O elétron de maior energia tem os números quânticos:
                                  \[
                                      \textcolor{red}{n = 3, \quad \ell = 2, \quad m_\ell = 1, \quad m_s = -1/2}
                                  \]
                                  Isso indica que o elétron está no subnível \textcolor{red}{3d}, pois \textcolor{red}{\(\ell = 2\)} corresponde ao subnível \textcolor{red}{d}.
                            \item[] O subnível \textcolor{red}{3d} pode acomodar até \textcolor{red}{10 elétrons}, distribuídos em \textcolor{red}{5 orbitais} \textcolor{red}{(\( m_\ell = -2, -1, 0, +1, +2 \))}, cada um com 2 elétrons de spins opostos.
                            \item[] Como o elétron de maior energia está no subnível \textcolor{red}{3d}, o nível de valência é o terceiro nível \textcolor{red}{(\( n = 3 \))}, que inclui os subníveis \textcolor{red}{3s}, \textcolor{red}{3p} e \textcolor{red}{3d}.
                            \item[] A configuração eletrônica do átomo, considerando que o elétron de maior energia está no \textcolor{red}{3d}, é:
                                  \[
                                      \textcolor{red}{1s^2 \ 2s^2 \ 2p^6 \ 3s^2 \ 3p^6 \ 3d^x}
                                  \]
                                  onde \textcolor{red}{\(x\)} é o número de elétrons no subnível \textcolor{red}{3d}.
                            \item[] Como o elétron de maior energia tem \textcolor{red}{\( m_\ell = 1 \)} e \textcolor{red}{\( m_s = -1/2 \)}, isso indica que o subnível \textcolor{red}{3d} está parcialmente preenchido. O número total de elétrons no nível de valência (camada \textcolor{red}{\( n = 3 \)}) é a soma dos elétrons nos subníveis \textcolor{red}{3s}, \textcolor{red}{3p} e \textcolor{red}{3d}:
                                  \[
                                      \textcolor{red}{3s^2 \ 3p^6 \ 3d^x}
                                  \]
                                  Vamos distribuir os elétrons no subnível \textcolor{red}{3d} até chegar no elétron com os números quanticos dados:
                                  \[
    \textcolor{red}{
        \begin{array}{c}
            \begin{tabular}{|m{0.5cm}|m{0.5cm}|m{0.5cm}|m{0.5cm}|m{0.5cm}|}
                \hline
                $\uparrow \downarrow$ & $\uparrow \downarrow$ & $\uparrow \downarrow$ & $\uparrow \downarrow$ & $\uparrow$ \\
                \hline
            \end{tabular} \\
            \begin{tabular}{m{0.5cm} m{0.5cm} m{0.5cm} m{0.5cm} m{0.5cm}}
                $-2$ & $-1$ & \text{ 0 } & $+1$ & $+2$
            \end{tabular}
        \end{array}
    }
\]
                                  Como o subnível \textcolor{red}{3d} pode ter no máximo \textcolor{red}{10 elétrons}, o número de elétrons no nível de valência varia entre \textcolor{red}{8} e \textcolor{red}{18}, dependendo de \textcolor{red}{\(x\)}.
                            \item[] No caso específico em que o elétron de maior energia está no \textcolor{red}{3d} com \textcolor{red}{\( m_\ell = 1 \)} e \textcolor{red}{\( m_s = -1/2 \)}, ao contar os elétrons até chegar nesse elétron, obtemos \textcolor{red}{9 elétrons} para \textcolor{red}{x}:
                                  \[
                                      \textcolor{red}{2 + 6 + 9 = 17}
                                  \]
                        \end{itemize}
                        Portanto, o número de elétrons no nível de valência desse átomo é \textcolor{red}{17 elétrons}.
                    }
          \end{enumerate}

    \item A luz de três diferentes lasers (A, B e C), cada uma com comprimento de onda diferente, iluminou a mesma superfície metálica. O laser C não produziu fotoelétrons. Os lasers A e B produziram fotoelétrons, mas os fotoelétrons produzidos pelo laser B tinham uma velocidade maior que os produzidos pelo laser A. Disponha os lasers em ordem decrescente de comprimento de onda e justifique sua resposta.

    \item Um átomo de hidrogênio está em um estado excitado com \( n = 2 \), com uma energia \( E_2 = -3,4 \) eV. Ocorre uma transição para o estado \( n = 1 \), com energia \( E_1 = -13,6 \) eV, e um fóton é emitido. Qual a frequência da radiação emitida em Hz? (Dados: \( 1 \) eV = \( 1,6 \times 10^{-19} \) J).


    \item Considerando as regras de algarismos significativos responda:

          \begin{enumerate}
              \item[a)] Um recipiente contendo amostra de água a \( 25^\circ C \) tem \( 234,9 \)g. A densidade para esta amostra de água nesta temperatura é dada como \( 0,99707g \cdot mL\) \textsuperscript{-1}. Estabeleça o volume da amostra de água no recipiente. Forneça a resposta expressando o resultado com o número correto de algarismos significativos.
              \item[b)] Nas operações vistas a seguir, arredonde as respostas para que contenham o número correto de algarismos significativos. Apresente as respostas com as unidades corretas.
                    \begin{enumerate}
                        \item[I)] \( (29,38 \pm 0,04) / (0,105 \pm 1,073) \)
                        \item[II)] \( (1,42 \times 10^2) \times (0,030 \text{mL}) / (6,478 \text{mL}) \times (40,0 \text{mL}) \)
                    \end{enumerate}
          \end{enumerate}
\end{enumerate}

\end{document}
