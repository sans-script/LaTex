\documentclass[a4paper, 12pt]{article}
\usepackage[utf8]{inputenc}            % Codificação do documento
\usepackage[T1]{fontenc}               % Encodings para caracteres especiais
\usepackage[brazil]{babel}             % Idioma do documento
\usepackage{geometry}                  % Para ajustar margens
\usepackage{graphicx}                  % Para incluir imagens
\usepackage{amsmath, amsfonts, amsthm, amssymb} % Para símbolos matemáticos
\usepackage{fancyhdr}                  % Para cabeçalhos e rodapés
\usepackage[dvipsnames]{xcolor}        % Para cores
\usepackage{thmtools}                  % Para definições de teoremas
\usepackage{titlesec}                  % Para personalizar títulos de seções
\usepackage{booktabs}                  % Adicione isso no preâmbulo
\usepackage{tabularx}                  % Para tabelas com largura ajustável
\usepackage{mathptmx}                  % Para usar Times New Roman
\usepackage{enumitem} % Para personalizar a lista
\usepackage{cancel} % Para cancelar termos em equações
\usepackage[version=4]{mhchem} % Para escrever fórmulas químicas
\usepackage{chemfig} % Para desenhar moléculas
\usepackage{colortbl} % Para colorir células de tabelas
\usepackage{multicol} % Para dividir o texto em colunas
\usepackage{tikz} % Para desenhar gráficos
\usepackage{tikz}
\usepackage{pgf-spectra}
\usepackage{siunitx}


% Configuração das margens
\geometry{top=2cm, bottom=2cm, left=1cm, right=1cm}

\titleformat{\section}
{\normalfont\Large\bfseries}{\thesection}{0.6em}{}

\titleformat{\subsection}
{\normalfont\large\bfseries}{\thesubsection}{0.3em}{}

\setcounter{secnumdepth}{0}  % Remove a numeração de seções
\def\checkmark{\tikz\fill[scale=0.4](0,.35) -- (.25,0) -- (1,.7) -- (.25,.15) -- cycle;}
\definecolor{indigo}{rgb}{0.29, 0.0, 0.51}

% Configura cabeçalhos e rodapés
\pagestyle{fancy}
\fancyhf{} % Limpa cabeçalhos e rodapés
\fancyhead[R]{Exercícios de Fixação} % Nome à direita no cabeçalho
\fancyhead[L]{Alexandre Santos}
\fancyfoot[R]{\thepage} % Número da página à direita no rodapé
\begin{document}

\subsection*{Algarismos Significativos}

\begin{enumerate}
    \item Quantos algarismos são relatados em cada uma das seguintes medições?
          \begin{multicols}{2}
              \begin{itemize}
                  \item[a)] \text{0,000403s} \textcolor{red}{3} \textcolor{blue}{alg. sig.:} \textcolor{red}{4, 0, 3}
                  \item[b)] \text{200.000g} \textcolor{red}{6} \textcolor{blue}{alg. sig.:} \textcolor{red}{2, 0, 0, 0, 0, 0}
                  \item[c)] \text{1,01mL} \textcolor{red}{3} \textcolor{blue}{alg. sig.:} \textcolor{red}{1, 0, 1}
                  \item[d)] \text{6,02 $\times 10^{23}$ mol L$^{-1}$} \textcolor{red}{3} \textcolor{blue}{alg. sig.:} \textcolor{red}{6, 0, 2}
                  \item[e)] \text{3500cm} \textcolor{red}{4} \textcolor{blue}{alg. sig.:} \textcolor{red}{3, 5, 0, 0}
                  \item[f)] \text{11,342g/cm} \textcolor{red}{5} \textcolor{blue}{alg. sig.:} \textcolor{red}{1, 1, 3, 4, 2}
                  \item[g)] \text{55,0mL} \textcolor{red}{3} \textcolor{blue}{alg. sig.:} \textcolor{red}{5, 5, 0}
                  \item[h)] \text{2,004mg} \textcolor{red}{4} \textcolor{blue}{alg. sig.:} \textcolor{red}{2, 0, 0, 4}
                  \item[i)] \text{7,046 $\times 10^{5}$ m} \textcolor{red}{4} \textcolor{blue}{alg. sig.:} \textcolor{red}{7, 0, 4, 6}
                  \item[j)] \text{37mm} \textcolor{red}{2} \textcolor{blue}{alg. sig.:} \textcolor{red}{3, 7}
                  \item[k)] \text{37,0mm} \textcolor{red}{3} \textcolor{blue}{alg. sig.:} \textcolor{red}{3, 7, 0}
                  \item[l)] \text{473m} \textcolor{red}{3} \textcolor{blue}{alg. sig.:} \textcolor{red}{4, 7, 3}
              \end{itemize}
          \end{multicols}

    \item Relate o resultado das operações aritméticas indicadas utilizando o número correto de algarismos significativos. Considere que os valores são medições.
          \begin{enumerate}[align=left, labelsep=-0.5em]
              \item[a)] 4,30 \(\times\) 0,31

                    \textcolor{blue}{
                        Multiplicamos os números \(\textcolor{red}{4,30 \times 0,31 = 1,333}\). O número com a menor quantidade de algarismos significativos é \(\textcolor{red}{0,31}\), que tem \(\textcolor{red}{2}\) algarismos significativos. Arredondamos o resultado para \(\textcolor{red}{2}\) algarismos significativos, obtendo \(\textcolor{red}{1,3}\).
                    }

              \item[b)] \(20,41 + 1,322 + 83,0\)

                    \textcolor{blue}{
                        Somamos os números \(\textcolor{red}{20,41 + 1,322 + 83,0 = 104,732}\). O número com a menor quantidade de algarismos significativos é \(\textcolor{red}{83,0}\), que tem \(\textcolor{red}{3}\) algarismos significativos. Arredondamos o resultado para \(\textcolor{red}{3}\) algarismos significativos, obtendo \(\textcolor{red}{105}\).
                    }

              \item[c)] 5,61 \(\div\) 17,32

                    \textcolor{blue}{
                        Dividimos os números \(\textcolor{red}{5,61 \div 17,32 = 0,3239}\). O número com a menor quantidade de algarismos significativos é \(\textcolor{red}{5,61}\), que tem \(\textcolor{red}{3}\) algarismos significativos. Arredondamos o resultado para \(\textcolor{red}{3}\) algarismos significativos, obtendo \(\textcolor{red}{0,324}\).
                    }

              \item[d)] \(7,354 - 2,8\)

                    \textcolor{blue}{
                        Subtraímos os números \(\textcolor{red}{7,354 - 2,8 = 4,554}\). O número com a menor quantidade de algarismos significativos é \(\textcolor{red}{2,8}\), que tem \(\textcolor{red}{2}\) algarismos significativos. Arredondamos o resultado para \(\textcolor{red}{2}\) algarismos significativos, obtendo \(\textcolor{red}{4,6}\).
                    }

              \item[e)] 1,024 \(\div\) 4

                    \textcolor{blue}{
                        Dividimos os números \(\textcolor{red}{1,024 \div 4 = 0,256}\). O número com a menor quantidade de algarismos significativos é \(\textcolor{red}{4}\), que tem \(\textcolor{red}{1}\) algarismo significativo. Arredondamos o resultado para \(\textcolor{red}{1}\) algarismo significativo, obtendo \(\textcolor{red}{0,3}\).
                    }

              \item[f)] 1,3589 \(\times\) 7,2

                    \textcolor{blue}{
                        Multiplicamos os números \(\textcolor{red}{1,3589 \times 7,2 = 9,78408}\). O número com a menor quantidade de algarismos significativos é \(\textcolor{red}{7,2}\), que tem \(\textcolor{red}{2}\) algarismos significativos. Arredondamos o resultado para \(\textcolor{red}{2}\) algarismos significativos, obtendo \(\textcolor{red}{9,8}\).
                    }

              \item[g)] \(25,32 - 2,8588\)

                    \textcolor{blue}{
                        Subtraímos os números \(\textcolor{red}{25,32 - 2,8588 = 22,4612}\). O número com a menor quantidade de algarismos significativos é \(\textcolor{red}{25,32}\), que tem \(\textcolor{red}{4}\) algarismos significativos. Arredondamos o resultado para \(\textcolor{red}{4}\) algarismos significativos, obtendo \(\textcolor{red}{22,46}\).
                    }

              \item[h)] \(0,0089 + 1,326\)

                    \textcolor{blue}{
                        Somamos os números \(\textcolor{red}{0,0089 + 1,326 = 1,3349}\). O número com a menor quantidade de algarismos significativos é \(\textcolor{red}{1,326}\), que tem \(\textcolor{red}{4}\) algarismos significativos. Arredondamos o resultado para \(\textcolor{red}{4}\) algarismos significativos, obtendo \(\textcolor{red}{1,335}\).
                    }
          \end{enumerate}

    \item Uma lata contém 18,2 litros de água. Se você despejar mais 0,2360 litros, o volume terá o número de algarismos significativos igual a quanto?

          \textcolor{blue}{
              Somamos os volumes \(\textcolor{red}{18,2 + 0,2360 = 18,436}\). O número com a menor quantidade de algarismos significativos é \(\textcolor{red}{18,2}\), que tem \(\textcolor{red}{3}\) algarismos significativos. Portanto, o volume final deve ser arredondado para \(\textcolor{red}{3}\) algarismos significativos, resultando em \(\textcolor{red}{18,4 \, \text{L}}\).
          }


    \item Sabendo que a densidade do clorofórmio é de 1,4832 g/mL a 20°C, qual seria o volume necessário para ser usado num procedimento extrativo que requer 59,59 g desse solvente? Expresse o resultado utilizando as regras para algarismos significativos.

          \textcolor{blue}{
              Utilizamos a fórmula da densidade \(\textcolor{red}{d = \frac{m}{V}}\) para calcular o volume:
              \[
                  \textcolor{red}{V = \frac{m}{d} = \frac{59,59}{1,4832} = 40,17}
              \]
              O número com a menor quantidade de algarismos significativos é \(\textcolor{red}{59,59}\), que tem \(\textcolor{red}{4}\) algarismos significativos. Arredondamos o resultado para \(\textcolor{red}{4}\) algarismos significativos, obtemos \textcolor{red}{40,18L}.
          }

    \item O diamante tem uma densidade de 3,513 g/cm\(^3\). A massa dos diamantes é medida frequentemente em quilates, sendo um quilate igual a 0,200 g. Qual é o volume (em centímetros cúbicos) de um diamante de 1,50 quilates? (responda usando o número correto de algarismos significativos).

          \textcolor{blue}{
              Primeiro, convertemos a massa do diamante de quilates para gramas:
              \[
                  \textcolor{red}{1,50 \, \text{ct} \times 0,200 \, \text{g/ct} = 0,300 \, \text{g}}
              \]
              Em seguida, utilizamos a fórmula da densidade \(\textcolor{red}{d = \frac{m}{V}}\) para calcular o volume:
              \[
                  \textcolor{red}{V = \frac{m}{d} = \frac{0,300}{3,513} = 0,0854}
              \]
              O número com a menor quantidade de algarismos significativos é \(\textcolor{red}{0,300}\), que tem \(\textcolor{red}{3}\) algarismos significativos. Arredondamos o resultado para \(\textcolor{red}{3}\) algarismos significativos, obtemos \(\textcolor{red}{0,0854 \, \text{cm}^3}\).
          }

    \item Um estudante, tendo medido o corredor de sua casa, encontrou os seguintes valores: Comprimento: 5,7 m; Largura: 1,25 m. Desejando determinar a área deste corredor com a maior precisão possível, o estudante multiplica os dois valores anteriores e registra o resultado com o número correto de algarismos, isto é, somente com os algarismos que sejam significativos. Como ele escreveu esse resultado?

          \textcolor{blue}{
              Para calcular a área, multiplicamos o comprimento pela largura:
              \[
                  \textcolor{red}{5,7 \, \text{m} \times 1,25 \, \text{m} = 7,125 \, \text{m}^2}
              \]
              O número com a menor quantidade de algarismos significativos é \(\textcolor{red}{5,7}\), que tem \(\textcolor{red}{2}\) algarismos significativos. Portanto, o resultado deve ser arredondado para \(\textcolor{red}{2}\) algarismos significativos, resultando em \(\textcolor{red}{7,1 \, \text{m}^2}\).
          }

    \item As distâncias moleculares são dadas em nanômetros ou picômetros. Entretanto, a unidade angstrom (Å) é por vezes utilizada, onde 1 Å = \(10^{-10}\) m. Se a distância entre o átomo de Pt e N na droga para quimioterapia anticâncer cisplatina é de 1,97 Å, qual é essa distância em nanômetros?

          \textcolor{blue}{
              Para converter a distância de angstroms \textcolor{red}{(\(\text{Å}\))} para nanômetros \textcolor{red}{(\(\text{nm}\))}, utilizamos a relação:
              \[
                  \textcolor{red}{1 \, \text{Å} = 10^{-1} \, \text{nm}}
              \]
              Portanto, a distância de \(\textcolor{red}{1,97 \, \text{Å}}\) em nanômetros é:
              \[
                  \textcolor{red}{1,97 \, \text{Å} \times 10^{-1} \, \text{nm/Å} = 0,197 \, \text{nm}}
              \]
              O número com a menor quantidade de algarismos significativos é \(\textcolor{red}{1,97}\), que tem \(\textcolor{red}{3}\) algarismos significativos. Portanto, o resultado final é \(\textcolor{red}{0,197 \, \text{nm}}\).
          }
    \item A densidade da água do mar é frequentemente expressa por unidades de quilogramas por metro cúbico. Se a densidade da água do mar é 1,025 g/cm\(^3\) a 15°C, qual é a sua densidade em quilogramas por metro cúbico (usar análise dimensional)?

          \textcolor{blue}{
              Para converter a densidade de \(\textcolor{red}{\text{g/cm}^3}\) para \(\textcolor{red}{\text{kg/m}^3}\), utilizamos fatores de conversão:
              \[
                  \textcolor{red}{1,025 \, \frac{\text{g}}{\text{cm}^3} \times \frac{1 \, \text{kg}}{10^3 \, \text{g}} \times \frac{1 \, \text{cm}^3}{10^{-6} \, \text{m}^3}}
              \]
              As unidades \(\textcolor{red}{\text{g}}\) e \(\textcolor{red}{\text{cm}^3}\) se cancelam, restando:
              \[
                  \textcolor{red}{\frac{1,025 \times 1 \, \text{kg} \times 1}{10^3 \times 10^{-6} \, \text{m}^3}}
              \]
              Resolvendo a expressão, obtemos:
              \[
                  \textcolor{red}{\frac{1,025}{10^{-3} \, \text{m}^3}}
              \]
              Sabemos que \textcolor{red}{\( \frac{1}{10^{-3}} = 10^3 \)}. Portanto:
              \[
                  \textcolor{red}{\frac{1,025}{10^{-3}} = 1,025 \times 10^3}
              \]
              A densidade em unidades do SI \textcolor{red}{(\(\text{kg/m}^3\))} é:
              \[
                  \textcolor{red}{1025 \, \text{kg/m}^3}
              \]
          }

    \item Para cada um dos seguintes conjuntos de valores experimentais, calcule a média aritmética e o desvio padrão.
          \begin{enumerate}[align=left, labelsep=-0.5em]
              \item[a)] 42,33; 42,28; 42,35; 42,30mL

                    \textcolor{blue}{
                        \text{Média aritmética:}
                        \[
                            \textcolor{red}{\bar{x} = \frac{42,33 + 42,28 + 42,35 + 42,30}{4} = \frac{169,26}{4} = 42,315 \, \text{mL}}
                        \]
                        \text{Desvio padrão:}
                        \[
                            \textcolor{red}{s = \sqrt{\frac{\sum_{i=1}^n (x_i - \bar{x})^2}{n - 1}}}
                        \]
                        \[
                            \textcolor{red}{s = \sqrt{\frac{(42,33 - 42,315)^2 + (42,28 - 42,315)^2 + (42,35 - 42,315)^2 + (42,30 - 42,315)^2}{4-1}}}
                        \]
                        Calculando os desvios:
                        \begin{multicols}{2}
                            \begin{itemize}
                                \item[] \(\textcolor{red}{(42,33 - 42,315)^2 = 0,000225}\)
                                \item[] \(\textcolor{red}{(42,28 - 42,315)^2 = 0,001225}\)
                                \item[] \(\textcolor{red}{(42,35 - 42,315)^2 = 0,001225}\)
                                \item[] \(\textcolor{red}{(42,30 - 42,315)^2 = 0,000225}\)
                            \end{itemize}
                        \end{multicols}
                        Somando os desvios:
                        \[
                            \textcolor{red}{0,000225 + 0,001225 + 0,001225 + 0,000225 = 0,0029}
                        \]
                        Calculando o desvio padrão:
                        \[
                            \textcolor{red}{s = \sqrt{\frac{0,0029}{3}} = \sqrt{0,0009667} \approx 0,0311 \, \text{mL}}
                        \]
                    }
              \item[b)] 0,032; 0,038; 0,036; 0,032; 0,034; 0,035g

                    \textcolor{blue}{
                        \text{Média aritmética:}
                        \[
                            \textcolor{red}{\bar{x} = \frac{0,032 + 0,038 + 0,036 + 0,032 + 0,034 + 0,035}{6} = \frac{0,207}{6} = 0,0345 \, \text{g}}
                        \]
                        \text{Desvio padrão:}
                        \[
                            \textcolor{red}{s = \sqrt{\frac{\sum_{i=1}^n (x_i - \bar{x})^2}{n - 1}}}
                        \]
                        $
                            \textcolor{red}{s = \sqrt{\frac{(0,032 - 0,0345)^2 + (0,038 - 0,0345)^2 + (0,036 - 0,0345)^2 + (0,032 - 0,0345)^2 + (0,034 - 0,0345)^2 + (0,035 - 0,0345)^2}{6-1}}}
                        $
                        \\[10pt]
                        Calculando os desvios:
                        \begin{multicols}{2}
                            \begin{itemize}
                                \item[] \(\textcolor{red}{(0,032 - 0,0345)^2 = 0,00000625}\)
                                \item[] \(\textcolor{red}{(0,038 - 0,0345)^2 = 0,00001225}\)
                                \item[] \(\textcolor{red}{(0,036 - 0,0345)^2 = 0,00000225}\)
                                \item[] \(\textcolor{red}{(0,032 - 0,0345)^2 = 0,00000625}\)
                                \item[] \(\textcolor{red}{(0,034 - 0,0345)^2 = 0,00000025}\)
                                \item[] \(\textcolor{red}{(0,035 - 0,0345)^2 = 0,00000025}\)
                            \end{itemize}
                        \end{multicols}
                        Somando os desvios:
                        \[
                            \textcolor{red}{0,00000625 + 0,00001225 + 0,00000225 + 0,00000625 + 0,00000025 + 0,00000025 = 0,0000275}
                        \]
                        Calculando o desvio padrão:
                        \[
                            \textcolor{red}{s = \sqrt{\frac{0,0000275}{5}} = \sqrt{0,0000055} \approx 0,00235 \, \text{g}}
                        \]
                    }
          \end{enumerate}
    \item Faça os arredondamentos abaixo para 2 casas decimais:
          \begin{enumerate}[align=left, labelsep=-0.5em]
              \item[a)] 15,4852 \textcolor{red}{15,49} \textcolor{blue}{alg. sig.:} \textcolor{red}{1, 5, 4, 9} \\
                    \textcolor{blue}{O terceiro dígito decimal \textcolor{red}{(5)} é igual ou maior que 5, então arredondamos para cima.}

              \item[b)] 25,3270 \textcolor{red}{25,33} \textcolor{blue}{alg. sig.:} \textcolor{red}{2, 5, 3, 3} \\
                    \textcolor{blue}{O terceiro dígito decimal \textcolor{red}{(7)} é maior que \textcolor{red}{5}, então arredondamos para cima.}

              \item[c)] 18,0300 \textcolor{red}{18,03} \textcolor{blue}{alg. sig.:} \textcolor{red}{1, 8, 0, 3} \\
                    \textcolor{blue}{O terceiro dígito decimal \textcolor{red}{(0)} é menor que \textcolor{red}{5}, então mantemos o valor.}

              \item[d)] 15,992 \textcolor{red}{15,99} \textcolor{blue}{alg. sig.:} \textcolor{red}{1, 5, 9, 9} \\
                    \textcolor{blue}{O terceiro dígito decimal \textcolor{red}{(2)} é menor que \textcolor{red}{5}, então mantemos o valor.}

              \item[e)] 7,5999 \textcolor{red}{7,60} \textcolor{blue}{alg. sig.:} \textcolor{red}{7, 6, 0} \\
                    \textcolor{blue}{O terceiro dígito decimal \textcolor{red}{(9)} é maior que \textcolor{red}{5}, então arredondamos para cima.}

              \item[f)] 8,3299 \textcolor{red}{8,33} \textcolor{blue}{alg. sig.:} \textcolor{red}{8, 3, 3} \\
                    \textcolor{blue}{O terceiro dígito decimal \textcolor{red}{(9)} é maior que \textcolor{red}{5}, então arredondamos para cima.}

              \item[g)] 15,0005  \textcolor{red}{15,00} \textcolor{blue}{alg. sig.:} \textcolor{red}{1, 5, 0, 0} \\
                    \textcolor{blue}{O terceiro dígito decimal \textcolor{red}{(0)} é menor que \textcolor{red}{5}, então mantemos o valor.}

              \item[h)] 35,92106 \textcolor{red}{35,92} \textcolor{blue}{alg. sig.:} \textcolor{red}{3, 5, 9, 2} \\
                    \textcolor{blue}{O terceiro dígito decimal \textcolor{red}{(1)} é menor que \textcolor{red}{5}, então mantemos o valor.}

              \item[i)] 0,890501 \textcolor{red}{0,89} \textcolor{blue}{alg. sig.:} \textcolor{red}{8, 9} \\
                    \textcolor{blue}{O terceiro dígito decimal \textcolor{red}{(0)} é menor que \textcolor{red}{5}, então mantemos o valor.}
          \end{enumerate}
\end{enumerate}
\end{document}