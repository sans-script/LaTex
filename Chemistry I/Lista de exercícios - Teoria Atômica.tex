\documentclass[a4paper, 12pt]{article}
\usepackage[utf8]{inputenc}            % Codificação do documento
\usepackage[T1]{fontenc}               % Encodings para caracteres especiais
\usepackage[brazil]{babel}             % Idioma do documento
\usepackage{geometry}                  % Para ajustar margens
\usepackage{graphicx}                  % Para incluir imagens
\usepackage{amsmath, amsfonts, amsthm, amssymb} % Para símbolos matemáticos
\usepackage{fancyhdr}                  % Para cabeçalhos e rodapés
\usepackage[dvipsnames]{xcolor}        % Para cores
\usepackage{thmtools}                  % Para definições de teoremas
\usepackage{titlesec}                  % Para personalizar títulos de seções
\usepackage{booktabs}                  % Adicione isso no preâmbulo
\usepackage{tabularx}                  % Para tabelas com largura ajustável
\usepackage{mathptmx}                  % Para usar Times New Roman
\usepackage{enumitem} % Para personalizar a lista
\usepackage{cancel} % Para cancelar termos em equações
\usepackage[version=4]{mhchem} % Para escrever fórmulas químicas
\usepackage{chemfig} % Para desenhar moléculas
\usepackage{colortbl} % Para colorir células de tabelas
\usepackage{multicol} % Para dividir o texto em colunas
\usepackage{tikz} % Para desenhar gráficos
\usepackage{tikz}
\usepackage{pgf-spectra}
\usepackage{siunitx}


% Configuração das margens
\geometry{top=2cm, bottom=2cm, left=1cm, right=1cm}

\titleformat{\section}
{\normalfont\Large\bfseries}{\thesection}{0.6em}{}

\titleformat{\subsection}
{\normalfont\large\bfseries}{\thesubsection}{0.3em}{}

\setcounter{secnumdepth}{0}  % Remove a numeração de seções
\def\checkmark{\tikz\fill[scale=0.4](0,.35) -- (.25,0) -- (1,.7) -- (.25,.15) -- cycle;}
\definecolor{indigo}{rgb}{0.29, 0.0, 0.51}

% Configura cabeçalhos e rodapés
\pagestyle{fancy}
\fancyhf{} % Limpa cabeçalhos e rodapés
\fancyhead[R]{Exercícios de Fixação} % Nome à direita no cabeçalho
\fancyhead[L]{Alexandre Santos}
\fancyfoot[R]{\thepage} % Número da página à direita no rodapé

\begin{document}

\subsection*{Natureza Ondulatória da Luz}

\begin{enumerate}
	\item Quando determinadas quantidades de gás xenônio são adicionadas às luzes de néon, sua cor se torna azul-esverdeada. Se o comprimento de onda dessa luz é 480 nm, qual é a frequência?
	      
	      \textcolor{blue}{
		      Para calcular a frequência da luz, utilizamos a relação entre a velocidade da luz \textcolor{red}{(\(c\))}, o comprimento de onda \textcolor{red}{(\(\lambda\))} e a frequência \textcolor{red}{(\(f\))}:
		      \[
			      \textcolor{red}{c = \lambda \times f}
		      \]
		      Sabemos que a velocidade da luz no vácuo é \textcolor{red}{\(c = 3,00 \times 10^8 \, \text{m/s}\)} e o comprimento de onda é \textcolor{red}{\(\lambda = 480 \, \text{nm}\)}. Representado por \textcolor{red}{\(480 \times 10^{-9} \, \text{m}\)}. Portanto, a frequência \textcolor{red}{\(f\)} é dada por:
		      \[
			      \textcolor{red}{f = \frac{c}{\lambda} = \frac{3,00 \times 10^8 \, \text{m/s}}{480 \times 10^{-9} \, \text{m}} = 6,25 \times 10^{14} \, \text{Hz}}
		      \]
		      A frequência da luz azul-esverdeada é \textcolor{red}{\(6,25 \times 10^{14} \, \text{Hz}\)}.
	      }
	      
	\item Um laser infravermelho para uso de uma rede de comunicações de fibra ótica emite um comprimento de onda de 1,2 µm. Qual é a energia de um fóton para esta radiação?
	      
	      \textcolor{blue}{
		      Para calcular a energia de um fóton, utilizamos a equação de Planck:
		      \[
			      \textcolor{red}{E = h \times f}
		      \]
		      Onde \textcolor{red}{\(E\)} é a energia do fóton, \textcolor{red}{\(h\)} é a constante de Planck \textcolor{red}{(\(h = 6,626 \times 10^{-34} \, \text{J} \cdot \text{s}\))} e \textcolor{red}{\(f\)} é a frequência da radiação.      
		      Primeiro, calculamos a frequência \textcolor{red}{\(f\)} a partir do comprimento de onda \textcolor{red}{\(\lambda = 1,2 \, \text{µm} = 1,2 \times 10^{-6} \, \text{m}\)}, utilizando a relação:
		      \[
			      \textcolor{red}{f = \frac{c}{\lambda}}
		      \]
		      Onde \textcolor{red}{\(c = 3,00 \times 10^8 \, \text{m/s}\)} é a velocidade da luz. Substituindo os valores:
		      \[
			      \textcolor{red}{f = \frac{3,00 \times 10^8 \, \text{m/s}}{1,2 \times 10^{-6} \, \text{m}} = 2,50 \times 10^{14} \, \text{Hz}}
		      \]
		      Agora, calculamos a energia do fóton:
		      \[
			      \textcolor{red}{E = h \times f = 6,626 \times 10^{-34} \, \text{J} \cdot \text{s} \times 2,50 \times 10^{14} \, \text{Hz} = 1,66 \times 10^{-19} \, \text{J}}
		      \]
		      Portanto, a energia de um fóton para esta radiação é \textcolor{red}{\(1,66 \times 10^{-19} \, \text{J}\)}.
	      }
	      
	\item Uma dada forma de radiação eletromagnética tem uma frequência de \(8,11 \times 10^{14} \, \text{s}^{-1}\).
	      \begin{enumerate}
		      \item[a)] Qual o seu comprimento de onda em nanômetro?
		            
		            \textcolor{blue}{
			            Para calcular o comprimento de onda \textcolor{red}{(\(\lambda\))}, utilizamos a relação entre a velocidade da luz \textcolor{red}{(\(c\))}, a frequência \textcolor{red}{(\(f\))} e o comprimento de onda:
			            \[
				            \textcolor{red}{\lambda = \frac{c}{f}}
			            \]
			            Onde \textcolor{red}{\(c = 3,00 \times 10^8 \, \text{m/s}\)} é a velocidade da luz e \textcolor{red}{\(f = 8,11 \times 10^{14} \, \text{s}^{-1}\)} é a frequência. Substituindo os valores:
			            \[
				            \textcolor{red}{\lambda = \frac{3,00 \times 10^8 \, \text{m/s}}{8,11 \times 10^{14} \, \text{s}^{-1}} = 3,699 \times 10^{-7} \, \text{m}}
			            \]
			            Convertendo o comprimento de onda para nanômetros \textcolor{red}{(1 nm = \(10^{-9}\) m)}:
			            \[
				            \textcolor{red}{\lambda = 3,699 \times 10^{-7} \times 10^9 = 3,69 \times 10^2 \, \text{ nm}}
			            \]
		            }
		      \item[b)] Qual a energia (em Joule) de um quantum dessa radiação?
		            
		            \textcolor{blue}{
			            A energia de um quantum (fóton) é dada pela equação de Planck:
			            \[
				            \textcolor{red}{E = h \times f}
			            \]
			            Onde \textcolor{red}{\(h = 6,626 \times 10^{-34} \, \text{J} \cdot \text{s}\)} é a constante de Planck e \textcolor{red}{\(f = 8,11 \times 10^{14} \, \text{s}^{-1}\)} é a frequência. Substituindo os valores:
			            \[
				            \textcolor{red}{E = 6,626 \times 10^{-34} \, \text{J} \cdot \text{s} \times 8,11 \times 10^{14} \, \text{s}^{-1} = 5,376 \times 10^{-19} \, \text{J}}
			            \]
			            Arredondando para duas casas decimais, a energia de um quantum dessa radiação é \textcolor{red}{\(5,38 \times 10^{-19} \, \text{J}\)}.
		            }
	      \end{enumerate}
	      
	\item Uma análise espectral cuidadosa mostra que a luz amarela das lâmpadas de sódio (usadas nos postes de rua) é uma mistura de fótons de dois comprimentos de onda, \textcolor{red}{589,0 nm} e \textcolor{red}{589,6 nm}. Qual é a diferença de energia (em Joule) entre os fótons com estes comprimentos de onda?
	      
	      \textcolor{blue}{
		      Para calcular a diferença de energia entre os fótons, primeiro determinamos a energia de cada fóton usando a equação de Planck:
		      \[
			      \textcolor{red}{E = \frac{h \cdot c}{\lambda}}
		      \]
		      Onde:
		      \begin{itemize}
			      \item[] \textcolor{red}{\(h = 6,626 \times 10^{-34} \, \text{J} \cdot \text{s}\)} é a constante de Planck.
			      \item[] \textcolor{red}{\(c = 3,00 \times 10^8 \, \text{m/s}\)} é a velocidade da luz.
			      \item[] \textcolor{red}{\(\lambda\)} é o comprimento de onda.
		      \end{itemize}
		      Para o comprimento de onda de \textcolor{red}{589,0 nm}:
		      \[
			      \textcolor{red}{\lambda_1 = 589,0 \, \text{nm} = 589,0 \times 10^{-9} \, \text{m}}
		      \]
		      \[
			      \textcolor{red}{E_1 = \frac{6,626 \times 10^{-34} \times 3,00 \times 10^8}{589,0 \times 10^{-9}} = 3,375 \times 10^{-19} \, \text{J}}
		      \]
		      Para o comprimento de onda de \textcolor{red}{589,6 nm}:
		      \[
			      \textcolor{red}{\lambda_2 = 589,6 \, \text{nm} = 589,6 \times 10^{-9} \, \text{m}}
		      \]
		      \[
			      \textcolor{red}{E_2 = \frac{6,626 \times 10^{-34} \times 3,00 \times 10^8}{589,6 \times 10^{-9}} = 3,371 \times 10^{-19} \, \text{J}}
		      \]
		      A diferença de energia entre os fótons é:
		      \[
			      \textcolor{red}{\Delta E = E_1 - E_2 = 3,375 \times 10^{-19} \, \text{J} - 3,370 \times 10^{-19} \, \text{J} = 4,00 \times 10^{-22} \, \text{J}}
		      \]
		      Portanto, a diferença de energia entre os fótons com comprimentos de onda de \textcolor{red}{589,0 nm} e \textcolor{red}{589,6 nm} é \textcolor{red}{\(4,00 \times 10^{-22} \, \text{J}\)}.
	      }
	      \pagebreak
	\item Calcule a frequência em Hertz, a energia em joules de um fóton de raios X que tem comprimento de onda de \textcolor{red}{2,70 Å}.
	      
	      \textcolor{blue}{
		      Para calcular a frequência e a energia do fóton, utilizamos as seguintes relações:
		      \begin{itemize}
			      \item[] A frequência \textcolor{red}{(\(f\))} é dada por:
			            \[
				            \textcolor{red}{f = \frac{c}{\lambda}}
			            \]
			      \item[] A energia \textcolor{red}{(\(E\))} de um fóton é dada pela equação de Planck:
			            \[
				            \textcolor{red}{E = h \cdot f}
			            \]
			            Onde:
			            \begin{itemize}
				            \item[] \textcolor{red}{\(c = 3,00 \times 10^8 \, \text{m/s}\)} é a velocidade da luz.
				            \item[] \textcolor{red}{\(h = 6,626 \times 10^{-34} \, \text{J} \cdot \text{s}\)} é a constante de Planck.
				            \item[] \textcolor{red}{\(\lambda\)} é o comprimento de onda.
			            \end{itemize}
		      \end{itemize}
		      Primeiro, convertemos o comprimento de onda de angstroms \textcolor{red}{(Å)} para metros \textcolor{red}{(m)}:
		      \[
			      \textcolor{red}{\lambda = 2,70 \, \text{Å} = 2,70 \times 10^{-10} \, \text{m}}
		      \]
		      \text{Cálculo da frequência:}
		      \[
			      \textcolor{red}{f = \frac{c}{\lambda} = \frac{3,00 \times 10^8 \, \text{m/s}}{2,70 \times 10^{-10} \, \text{m}} = 1,11 \times 10^{18} \, \text{Hz}}
		      \]
		      \text{Cálculo da energia:}
		      \[
			      \textcolor{red}{E = h \cdot f = 6,626 \times 10^{-34} \, \text{J} \cdot \text{s} \times 1,11 \times 10^{18} \, \text{Hz} = 7,36 \times 10^{-16} \, \text{J}}
		      \]
		      Portanto:
		      \begin{itemize}
			      \item[] A frequência do fóton de raios X é \textcolor{red}{\(1,11 \times 10^{18} \, \text{Hz}\)}.
			      \item[] A energia do fóton de raios X é \textcolor{red}{\(7,36 \times 10^{-16} \, \text{J}\)}.
		      \end{itemize}
	      }
	      
	\item Analise as seguintes informações sobre a radiação eletromagnética e determine se são verdadeiras ou falsas. Se forem falsas, corrija-as.
	      \begin{enumerate}
		      \item[\textcolor{green}{a)}] \textcolor{green}{(V) Numa superfície metálica nenhum elétron é ejetado até que a radiação tenha frequência igual ou acima de um determinado valor, característico da ligação do elétron com o metal.}
		      \item[] \textcolor{blue}{
			            Justificativa: A afirmação é verdadeira, pois descreve corretamente o efeito fotoelétrico. Segundo esse fenômeno, a radiação incidente só pode ejetar elétrons de uma superfície metálica se sua frequência for maior ou igual à frequência de limiar, que depende da energia necessária para liberar o elétron do metal.
		            }
		      \item[\textcolor{red}{b)}] \textcolor{red}{(F) A energia de um fóton é diretamente proporcional ao comprimento de onda da radiação.}
		      \item[] \textcolor{blue}{
			            Justificativa: A afirmação é falsa. A energia de um fóton é inversamente proporcional ao comprimento de onda, como descrito pela fórmula \( E = \frac{h \cdot c}{\lambda} \), onde \( h \) é a constante de Planck, \( c \) a velocidade da luz e \( \lambda \) o comprimento de onda. Isso significa que quanto maior o comprimento de onda, menor a energia do fóton.
		            }
		            \pagebreak
		      \item[\textcolor{red}{c)}] \textcolor{red}{(F) Num espectro eletromagnético a faixa de frequência das radiações infravermelha e da ultravioleta é na ordem de \(10^{14} \, \text{s}^{-1}\) e \(10^{16} \, \text{s}^{-1}\), respectivamente, portanto a radiação ultravioleta é de menor energia.}
		      \item[] \textcolor{blue}{
			            Justificativa: A afirmação é falsa. A radiação ultravioleta tem frequências mais altas do que a radiação infravermelha, o que implica em maior energia. De acordo com a fórmula \( E = h \cdot f \), onde \( f \) é a frequência, a radiação ultravioleta possui maior energia do que a infravermelha, pois sua frequência é maior.
		            }
	      \end{enumerate}
	      
	\item Calcule e compare a energia de um fóton de comprimento de onda de 3,3 µm com um de comprimento de onda de 0,154 nm.
	      
	      \textcolor{blue}{
		      Para calcular a energia de um fóton, utilizamos a equação de Planck:
		      \[
			      \textcolor{red}{E = \frac{h \cdot c}{\lambda}}
		      \]
		      Onde:
		      \begin{itemize}
			      \item[] \textcolor{red}{\(h = 6,626 \times 10^{-34} \, \text{J} \cdot \text{s}\)} é a constante de Planck.
			      \item[] \textcolor{red}{\(c = 3,00 \times 10^8 \, \text{m/s}\)} é a velocidade da luz.
			      \item[] \textcolor{red}{\(\lambda\)} é o comprimento de onda.
		      \end{itemize}
		      \text{Fóton 1: Comprimento de onda de \textcolor{red}{3,3 µm}}
		      Primeiro, convertemos o comprimento de onda de micrômetros \textcolor{red}{(µm)} para metros \textcolor{red}{(m)}:
		      \[
			      \textcolor{red}{\lambda_1 = 3,3 \, \text{µm} = 3,3 \times 10^{-6} \, \text{m}}
		      \]
		      Agora, calculamos a energia do fóton:
		      \[
			      \textcolor{red}{E_1 = \frac{6,626 \times 10^{-34} \times 3,00 \times 10^8}{3,3 \times 10^{-6}} = 6,02 \times 10^{-20} \, \text{J}}
		      \]
		      \text{Fóton 2: Comprimento de onda de \textcolor{red}{ 0,154 nm}}
		      Primeiro, convertemos o comprimento de onda de nanômetros \textcolor{red}{(nm)} para metros \textcolor{red}{(m)}:
		      \[
			      \textcolor{red}{\lambda_2 = 0,154 \, \text{nm} = 0,154 \times 10^{-9} \, \text{m}}
		      \]
		      Agora, calculamos a energia do fóton:
		      \[
			      \textcolor{red}{E_2 = \frac{6,626 \times 10^{-34} \times 3,00 \times 10^8}{0,154 \times 10^{-9}} = 1,29 \times 10^{-15} \, \text{J}}
		      \]
		      \text{Comparação das energias:}
		      \begin{itemize}
			      \item[] A energia do fóton com comprimento de onda de \textcolor{red}{3,3 µm} é \textcolor{red}{\(6,02 \times 10^{-20} \, \text{J}\)}.
			      \item[] A energia do fóton com comprimento de onda de \textcolor{red}{0,154 nm} é \textcolor{red}{\(1,29 \times 10^{-15} \, \text{J}\)}.
		      \end{itemize}
		      \[
			      \textcolor{red}{\frac{E_2}{E_1} = \frac{1,29 \times 10^{-15}}{6,02 \times 10^{-20}} = 2,14 \times 10^4}
		      \]
		      Portanto, o fóton com comprimento de onda de \textcolor{red}{0,154 nm} tem uma energia \textcolor{red}{\(2,14 \times 10^4\)} vezes maior que o fóton com comprimento de onda de \textcolor{red}{3,3 µm}.
	      }
	      \pagebreak
	      
	\item Calcule o comprimento de onda (em nanômetro) do fóton emitido quando o elétron do átomo de hidrogênio decai do nível \(n = 5\) para o nível \(n = 2\).
	      
	      \textcolor{blue}{
	      Para calcular o comprimento de onda do fóton emitido, utilizamos a fórmula de Rydberg para o átomo de hidrogênio:
	      \[
		      \textcolor{red}{\frac{1}{\lambda} = R_H \left( \frac{1}{n_f^2} - \frac{1}{n_i^2} \right)}
	      \]
	      Onde:
	      \begin{itemize}
		      \item[] \textcolor{red}{\(\lambda\)} é o comprimento de onda do fóton emitido.
		      \item[] \textcolor{red}{\(R_H = 1,097 \times 10^7 \, \text{m}^{-1}\)} é a constante de Rydberg para o hidrogênio.
		      \item[] \textcolor{red}{\(n_i = 5\)} é o nível inicial.
		      \item[] \textcolor{red}{\(n_f = 2\)} é o nível final.
	      \end{itemize}
	      Substituindo os valores na fórmula:
	      \[
		      \textcolor{red}{\frac{1}{\lambda} = 1,097 \times 10^7 \left( \frac{1}{2^2} - \frac{1}{5^2} \right)}
	      \]
	      \[
		      \textcolor{red}{\frac{1}{\lambda} = 1,097 \times 10^7 \left( \frac{1}{4} - \frac{1}{25} \right)}
	      \]
	      \[
		      \textcolor{red}{\frac{1}{\lambda} = 1,097 \times 10^7 \left( 0,25 - 0,04 \right)}
	      \]
	      \[
		      \textcolor{red}{\frac{1}{\lambda} = 1,097 \times 10^7 \times 0,21}
	      \]
	      \[
		      \textcolor{red}{\frac{1}{\lambda} = 2,3037 \times 10^6 \, \text{m}^{-1}}
	      \]
	      Agora, calculamos o comprimento de onda \(\lambda\):
	      \[
		      \textcolor{red}{\lambda = \frac{1}{2,3037 \times 10^6} = 4,34 \times 10^{-7} \, \text{m}}
	      \]
	      Convertendo o comprimento de onda para nanômetros {(1 nm = \(10^{-9}\) m)}:
	      \[
		      \textcolor{red}{\lambda = 4,34 \times 10^{-7} \, \text{m} \times 10^9 \, \text{nm/m} = 434 \, \text{nm}}
	      \]
	      Apresentando o resultado em notação científica:
	      \[
		      \textcolor{red}{\lambda = 4,34 \times 10^{2} \, \text{nm}}
	      \]
	      Portanto, o comprimento de onda do fóton emitido é \textcolor{red}{\(434 \, \text{nm}\)} ou \textcolor{red}{\(4,34 \times 10^{2} \, \text{nm}\)}.
	      }
	      
	      
	      \pagebreak
	\item As cores de luz exibidas na queima de fogos de artifício dependem de certas substâncias utilizadas na sua fabricação. Sabe-se que a frequência da luz emitida pela combustão do níquel é \(6,0 \times 10^{14} \, \text{Hz}\) e que a velocidade da luz é \(3 \times 10^{8} \, \text{m/s}\). Com base nesses dados e no espectro visível fornecido pela figura a seguir, diga qual a cor da luz dos fogos de artifício que contêm compostos de níquel.
	      
	      \textcolor{blue}{
		      Para determinar a cor da luz emitida pelo níquel, primeiro calculamos seu comprimento de onda utilizando a relação entre a velocidade da luz \textcolor{red}{\(c\)}, o comprimento de onda \textcolor{red}{\(\lambda\)} e a frequência \textcolor{red}{\(f\)}:
		      \[
			      \textcolor{red}{c = \lambda \cdot f}
		      \]
		      Isolando o comprimento de onda:
		      \[
			      \textcolor{red}{\lambda = \frac{c}{f}}
		      \]
		      Substituindo os valores:
		      \[
			      \textcolor{red}{\lambda = \frac{3,0 \times 10^8 \, \text{m/s}}{6,0 \times 10^{14} \, \text{Hz}}}
		      \]
		      \[
			      \textcolor{red}{\lambda = 5,0 \times 10^{-7} \, \text{m}}
		      \]
		      Convertendo para nanômetros \textcolor{red}{{(\(1 \, \text{nm} = 10^{-9} \, \text{m}\))}}:
		      \[
			      \textcolor{red}{\lambda = 5,0 \times 10^{-7} \, \text{m} \times 10^9 \, \text{nm/m}}
		      \]
		      \[
			      \textcolor{red}{\lambda = 500 \, \text{nm}}
		      \]
		      Consultando o espectro visível, a luz com comprimento de onda em torno de \textcolor{red}{\(500 \, \text{nm}\)} corresponde à cor \textcolor{green}{verde}.
	      }
	      
	      \begin{tikzpicture}
		      \node[
		      below=4em,
		      align=center,
		      label=above:{\text{Espectro Visível}}
		      ] (FULL_VISIBLE_LIGHT) {
		      \pgfspectra\\
		      {\num{400}\, \text{nm} \hfill \num{450}\, \text{nm} \hfill \num{500}\, \text{nm} \hfill \num{550}\, \text{nm} \hfill \num{580}\, \text{nm} \hfill \num{600}\, \text{nm} \hfill \num{650}\, \text{nm}}
		      };
	      \end{tikzpicture}
	      
	      \pagebreak   
	\item Um átomo de hidrogênio está em um estado excitado com \(n = 2\), com uma energia \(E_2 = -3,4 \, \text{eV}\). Ocorre uma transição para o estado \(n = 1\), com energia \(E_1 = -13,6 \, \text{eV}\), e um fóton é emitido. Qual a frequência da radiação emitida em Hz? (Dados: \(1 \, \text{eV} = 1,6 \times 10^{-19} \, \text{J}\); \(h = 6,63 \times 10^{-34} \, \text{Js}\).)
	      
	      \textcolor{blue}{
		      Para calcular a frequência da radiação emitida, utilizamos a diferença de energia entre os estados \textcolor{red}{\(n = 2\) }e \textcolor{red}{\(n = 1\)} e a relação de Planck:
		      \[
			      \textcolor{red}{E = h \cdot f}
		      \]
		      Onde:
		      \begin{itemize}
			      \item[] \textcolor{red}{\(E\)} é a energia do fóton emitido.
			      \item[] \textcolor{red}{\(h = 6,63 \times 10^{-34} \, \text{Js}\)} é a constante de Planck.
			      \item[] \textcolor{red}{\(f\)} é a frequência da radiação emitida.
		      \end{itemize}
		      A diferença de energia \textcolor{red}{(\(\Delta E\))} entre os estados \textcolor{red}{\(n = 2\) }e \textcolor{red}{\(n = 1\)} é:
		      \[
			      \textcolor{red}{\Delta E = E_1 - E_2 = -13,6 \, \text{eV} - (-3,4 \, \text{eV}) = -10,2 \, \text{eV}}
		      \]
		      Como a energia do fóton emitido é positiva, consideramos o valor absoluto:
		      \[
			      \textcolor{red}{\Delta E = 10,2 \, \text{eV}}
		      \]
		      Convertemos a energia de elétron-volts \textcolor{red}{(eV)} para joules \textcolor{red}{(J)} usando a conversão \textcolor{red}{\(1 \, \text{eV} = 1,6 \times 10^{-19} \, \text{J}\)}:
		      \[
			      \textcolor{red}{\Delta E = 10,2 \, \text{eV} \times 1,6 \times 10^{-19} \, \text{J/eV} = 1,632 \times 10^{-18} \, \text{J}}
		      \]
		      Agora, calculamos a frequência da radiação emitida usando a relação de Planck:
		      \[
			      \textcolor{red}{f = \frac{\Delta E}{h} = \frac{1,632 \times 10^{-18} \, \text{J}}{6,63 \times 10^{-34} \, \text{Js}} = 2,46 \times 10^{15} \, \text{Hz}}
		      \]
		      Portanto, a frequência da radiação emitida é \textcolor{red}{\(2,46 \times 10^{15} \, \text{Hz}\)}.
	      }
	      
	\item Calcule o comprimento de onda de De Broglie, em nanômetros, associado a uma bola de futebol, com massa de 400 g que se desloca a uma velocidade de 10 m/s.
	      
	      \textcolor{blue}{
		      Para calcular o comprimento de onda de De Broglie, utilizamos a fórmula:
		      \[
			      \textcolor{red}{\lambda = \frac{h}{mv}}
		      \]
		      Onde:
		      \begin{itemize}
			      \item[] \textcolor{red}{\(h = 6,626 \times 10^{-34} \, \text{J} \cdot \text{s}\)} é a constante de Planck;
			      \item[] \textcolor{red}{\(m = 400 \, \text{g} = 0,400 \, \text{kg}\)} é a massa da bola de futebol;
			      \item[] \textcolor{red}{\(v = 10 \, \text{m/s}\)} é a velocidade da bola.
		      \end{itemize}
		      Substituindo os valores:
		      \[
			      \textcolor{red}{\lambda = \frac{6,626 \times 10^{-34}}{0,400 \times 10} = \frac{6,626 \times 10^{-34}}{4}}
		      \]
		      \[
			      \textcolor{red}{\lambda = 1,6565 \times 10^{-34} \, \text{m}}
		      \]
		      Convertendo para nanômetros \textcolor{red}{{(\(1 \, \text{nm} = 10^{-9} \, \text{m}\))}}:
		      \[
			      \textcolor{red}{\lambda = 1,6565 \times 10^{-34} \, \text{m} \times 10^9 \, \text{nm/m} = 1,6565 \times 10^{-25} \, \text{nm}}
		      \]
		      Portanto, o comprimento de onda de De Broglie associado à bola de futebol é \textcolor{red}{\(1,66 \times 10^{-25} \, \text{nm}\)}.
	      }
	      \pagebreak
	      
	\item Um elétron se move a velocidade igual a \(4,7 \times 10^{6} \, \text{m/s}\). Determine o comprimento de onda de De Broglie, em nanômetro, para esse elétron. Obs.: apresentar todas as unidades durante as operações.
	      
	      \textcolor{blue}{
		      Para determinar o comprimento de onda de De Broglie, utilizamos a fórmula:
		      \[
			      \textcolor{red}{\lambda = \frac{h}{mv}}
		      \]
		      Onde:
		      \begin{itemize}
			      \item[] \textcolor{red}{\(h = 6,626 \times 10^{-34} \, \text{J} \cdot \text{s}\)} é a constante de Planck;
			      \item[] \textcolor{red}{\(m = 9,11 \times 10^{-31} \, \text{kg}\)} é a massa do elétron;
			      \item[] \textcolor{red}{\(v = 4,7 \times 10^{6} \, \text{m/s}\)} é a velocidade do elétron.
		      \end{itemize}
		      Substituindo os valores na fórmula:
		      \[
			      \textcolor{red}{\lambda = \frac{6,626 \times 10^{-34} \, \text{J} \cdot \text{s}}{9,11 \times 10^{-31} \, \text{kg} \times 4,7 \times 10^{6} \, \text{m/s}}}
		      \]
		      Calculando o denominador:
		      \[
			      \textcolor{red}{9,11 \times 10^{-31} \, \text{kg} \times 4,7 \times 10^{6} \, \text{m/s} = 4,2817 \times 10^{-24} \, \text{kg} \cdot \text{m/s}}
		      \]
		      Assim:
		      \[
			      \textcolor{red}{\lambda = \frac{6,626 \times 10^{-34} \, \text{J} \cdot \text{s}}{4,2817 \times 10^{-24} \, \text{kg} \cdot \text{m/s}} = 1,548 \times 10^{-10} \, \text{m}}
		      \]
		      Convertendo para nanômetros \textcolor{red}{{(\(1 \, \text{nm} = 10^{-9} \, \text{m}\))}}:
		      \[
			      \textcolor{red}{\lambda = 1,548 \times 10^{-10} \, \text{m} \times 10^{9} \, \text{nm} = 1,548 \times 10^{-1} \text{nm}}
		      \]
		      Portanto, o comprimento de onda de De Broglie para esse elétron é \textcolor{red}{\(0,1548 \, \text{nm}\)} ou \textcolor{red}{\(1,548 \times 10^{-1} \text{nm}\)}. 
	      }
	      
\end{enumerate}

\subsection*{Números Quânticos}

\begin{enumerate}
	\setcounter{enumi}{12}
	\item Um orbital tem números quânticos de \(n = 4\), \(l = 2\), \(m_l = -1\). Que tipo de orbital é esse?
	      
	      \textcolor{blue}{
		      Para identificar o tipo de orbital, analisamos os números quânticos fornecidos:
		      \begin{itemize}
			      \item[] \textcolor{red}{\(n = 4\)} é o nível de energia (camada).
			      \item[] \textcolor{red}{\(l = 2\)} é o subnível \(d\).
			      \item[] \textcolor{red}{\(m_l = -1\)} é a orientação do orbital.
		      \end{itemize}
		      O orbital é um \textcolor{red}{\(4d\)}, com \textcolor{red}{\(m_l = -1\)}, uma das cinco orientações possíveis para um orbital \textcolor{red}{\(d\)}. A distribuição dos elétrons no subnível \textcolor{red}{\(d\)} é feita da seguinte forma:    
		      \begin{itemize}
			      \item[] \text{Quando há \textcolor{red}{2} elétrons no subnível \textcolor{red}{\(d\)}, a configuração é \textcolor{red}{\(4d^2\)}.}
			      \item[] \text{Quando há \textcolor{red}{7} elétrons, a configuração é \textcolor{red}{\(4d^7\)}.}
		      \end{itemize}
	      }
	      \[
    \textcolor{red}{
        \begin{array}{c c c}
            \begin{array}{c}
                \begin{tabular}{|m{0.5cm}|m{0.5cm}|m{0.5cm}|m{0.5cm}|m{0.5cm}|}
                    \hline
                    $\uparrow$ & $\uparrow$ &  &  & \\
                    \hline
                \end{tabular} \\
                \begin{tabular}{m{0.5cm} m{0.5cm} m{0.5cm} m{0.5cm} m{0.5cm}}
                    $-2$ & $-1$ & \text{ 0 } & $+1$ & $+2$
                \end{tabular}
            \end{array}
             & 
            \quad \text{\textcolor{blue}{ou}} \quad
             & 
            \begin{array}{c}
                \begin{tabular}{|m{0.5cm}|m{0.5cm}|m{0.5cm}|m{0.5cm}|m{0.5cm}|}
                    \hline
                    $\uparrow \downarrow$ & $\uparrow\downarrow$ & $\uparrow$ & $\uparrow$ & $\uparrow$ \\
                    \hline
                \end{tabular} \\
                \begin{tabular}{m{0.5cm} m{0.5cm} m{0.5cm} m{0.5cm} m{0.5cm}}
                    $-2$ & $-1$ & \text{ 0 } & $+1$ & $+2$
                \end{tabular}
            \end{array}
        \end{array}
    }
\]
	      
	      \textcolor{blue}{\text{Portanto, o orbital pode ter configuração \textcolor{red}{\(4d^2\)} ou \textcolor{red}{\(4d^7\)}.}}
	      \pagebreak
	\item Qual é o número máximo de elétrons em um átomo que podem ter os seguintes números quânticos:
	      \begin{enumerate}
		      \item[a)] \(n = 4\), \(l = 3\), \(m_l = -3\), \(m_s = -1/2\).
		            \setlength{\columnsep}{0.1cm}
		            \begin{multicols}{2}
			            \begin{tikzpicture}[scale=1.5, y={(0cm,-1cm)}, line cap=rect]
        \foreach\y in {1,...,8}
        {%
        \ifnum\y > 1
            \draw[very thick, red]    (0.5*\y+1.25,0.5*\y-0.75) arc (-135:45:{0.125*sqrt(2)});
        \fi
        \draw[very thick,->, red]   (0.5*\y+1.5,0.5*\y-0.5) -- (0.5,\y+0.5);
        \ifnum \y < 8
            \draw[very thick,->, red] (0.5,\y+0.5) arc (225:45:{0.125*sqrt(2)}) --
            (0.5*\y+1.75,0.5*\y-0.25);
        \fi
        }
        \foreach\y in {1,...,7}
            {%
                \pgfmathtruncatemacro\maxx{6-abs(4.5-\y)}
                \foreach[count=\x]\i in {s,p,d,f}
                    {%
                        \ifnum \x < \maxx
                            \pgfmathtruncatemacro\ne{2*(2*\x-1)}
                            \node[fill=white, text=red, inner sep=1pt] at (\x,\y) {$\y\i^{\ne}$};
                        \fi
                    }
            }
\end{tikzpicture}
			            
			            \columnbreak
			            \small 
			            \textcolor{blue}{
				            \textcolor{blue}{Primeiro, analisamos os números quânticos fornecidos:}
				            \begin{itemize}
					            \item \textcolor{red}{\(n = 4\)} é o número quântico principal, indicando a quarta camada de energia.
					            \item \textcolor{red}{\(l = 3\)} corresponde ao subnível \textcolor{red}{\(f\)}.
					            \item \textcolor{red}{\(m_l = -3\)} é uma das orientações possíveis.
					            \item \textcolor{red}{\(m_s = -1/2\)} representa o spin do elétron.
				            \end{itemize}
			            }
			            
			            \vspace{10pt}
			            
			            \text{\textcolor{blue}{Fazemos a distribuição dos elétrons}:}
			            \textcolor{red}{
				            1s\textsuperscript{2} \quad 2s\textsuperscript{2} \quad 2p\textsuperscript{6} \quad 3s\textsuperscript{2} \quad 3p\textsuperscript{6} \quad 4s\textsuperscript{2} \quad 3d\textsuperscript{10} 4p\textsuperscript{6} \quad 5s\textsuperscript{2} \quad 4d\textsuperscript{10} \quad 5p\textsuperscript{6} \quad 6s\textsuperscript{2}
			            }
			            
			            \textcolor{blue}{Até agora foram distribuídos \textcolor{red}{56} elétrons. Agora, é necessário preencher o subnível \textcolor{red}{4f}.}
			            \\[10pt]
			            \textcolor{blue}{Expandindo o subnível \textcolor{red}{4f}:
			            }
			            
			            \[
    \textcolor{red}{
        \begin{array}{c}
            \begin{tabular}{|m{0.5cm}|m{0.5cm}|m{0.5cm}|m{0.5cm}|m{0.5cm}|m{0.5cm}|m{0.5cm}|}
                \hline
                $\uparrow \textcolor{green}{\downarrow}$ & $\uparrow$ & $\uparrow$ & $\uparrow$ & $\uparrow$ & $\uparrow$ & $\uparrow$ \\
                \hline
            \end{tabular} \\ 
            \begin{tabular}{m{0.5cm} m{0.5cm} m{0.5cm} m{0.5cm} m{0.5cm} m{0.5cm} m{0.5cm}}
                $-3$ & $-2$ & $-1$ & \text{ 0 } & $+1$ & $+2$ & $+3$
            \end{tabular}
        \end{array}
    }
\]
			            
			            \textcolor{blue}{Começamos a partir do \textcolor{red}{57º} elétron. Os primeiros \textcolor{red}{7} elétrons são distribuídos com spin \textcolor{red}{\(\uparrow\)} nos orbitais do subnível \textcolor{red}{4f}. Para encontrar o elétron com os números quânticos desejados \textcolor{red}{(\(n = 4\), \(l = 3\), \(m_l = -3\), \(m_s = -1/2\))}, adicionamos um elétron no orbital \textcolor{red}{\(m_l = -3\)} com spin \textcolor{red}{\(\downarrow\)}.}
			            \textcolor{blue}{Portanto, o número máximo de elétrons que esse átomo pode ter é \textcolor{red}{64}.}
		            \end{multicols}
		            
		      \item[b)] \(n = 2\), \(l = 1\), \(m_l = 0\), \(m_s = -1/2\).
		            \\[10pt]
		            \small 
		            \textcolor{blue}{Primeiro, analisamos os números quânticos fornecidos:}
		            \begin{multicols}{2}
			            \textcolor{blue}{
				            \begin{itemize}
					            \item \textcolor{red}{\(n = 2\)} é o número quântico principal.
					            \item \textcolor{red}{\(l = 1\)} corresponde ao subnível \textcolor{red}{\(p\)}.
				            \end{itemize}
			            }
			            \columnbreak
			            \textcolor{blue}{
				            \begin{itemize}
					            \item \textcolor{red}{\(m_l = 0\)} é uma das orientações possíveis.
					            \item \textcolor{red}{\(m_s = -1/2\)} representa o spin do elétron.
				            \end{itemize}
			            }
		            \end{multicols}
		            
		            
		            \vspace{10pt}
		            
		            \text{\textcolor{blue}{Fazemos a distribuição dos elétrons}:}
		            \textcolor{red}{
			            1s\textsuperscript{2} \quad 2s\textsuperscript{2}
		            }
		            \\[10pt] 
		            \textcolor{blue}{Até agora foram distribuídos \textcolor{red}{4} elétrons. Agora, é necessário preencher o subnível \textcolor{red}{2p}.}
		            \\[10pt]
		            \textcolor{blue}{Expandindo o subnível \textcolor{red}{2p}:}
		            
		            \input{orbital_2p.tex}
		            
		            \textcolor{blue}{Começamos a partir do \textcolor{red}{5º} elétron. Os primeiros \textcolor{red}{3} elétrons são distribuídos com spin \textcolor{red}{\(\uparrow\)}. O próximo elétron é distribuído com spin \textcolor{red}{\(\downarrow\)}. Para encontrar o elétron com os números quânticos desejados \textcolor{red}{(\(n = 2\), \(l = 1\), \(m_l = 0\), \(m_s = -1/2\))}, adicionamos mais um elétron no orbital \textcolor{red}{\(m_l = 0\)} com spin \textcolor{red}{\(\downarrow\)}.}
		            \textcolor{blue}{Portanto, o número máximo de elétrons que esse átomo pode ter é \textcolor{red}{9}.}
		            
	      \end{enumerate}
	      
	\item Escreva um conjunto completo de números quânticos (\(n\), \(l\), \(m_l\), \(m_s\)) permitidos pela teoria quântica para cada um dos seguintes orbitais:
	      \begin{enumerate}
		      \item[a)] \(2p^2\)
		            
		            \textcolor{blue}{No orbital \textcolor{red}{\(2p\)}:}
		            \textcolor{blue}{
			            \begin{itemize}
				            \item \textcolor{blue}{\( \textcolor{red}{n = 2} \) é o número quântico principal.}
				            \item \textcolor{blue}{\( \textcolor{red}{l = 1} \) é o subnível \( \textcolor{red}{p} \).}
				            \item \textcolor{blue}{Valores permitidos de \( \textcolor{red}{m_l} \): \( \textcolor{red}{-1} \), \( \textcolor{red}{0} \) e \( \textcolor{red}{+1} \).}
				            \item \textcolor{blue}{Para cada orbital, \( \textcolor{red}{m_s = +1/2} \) ou \( \textcolor{red}{m_s = -1/2} \).}
			            \end{itemize}}
		            
		            \textcolor{blue}{Para a configuração \( \textcolor{red}{2p^2} \), temos dois elétrons. Seguindo a regra de Hund, eles serão distribuídos em orbitais diferentes.}
		            
		            \[
    \textcolor{red}{
        \begin{array}{c}
            \begin{tabular}{|m{0.5cm}|m{0.5cm}|m{0.5cm}|}
                \hline
                $\uparrow $ & $\uparrow $ &  \\
                \hline
            \end{tabular} \\ 
            \begin{tabular}{m{0.5cm} m{0.5cm} m{0.5cm}}
                $-1$ & \text{ 0 } & $+1$
            \end{tabular}
        \end{array}
    }
\]
		            
		            \textcolor{blue}{Na tabela acima, o orbital com \( \textcolor{red}{m_l = -1} \) contém um elétron (com \( \textcolor{red}{m_s = +1/2} \)) e o orbital com \( \textcolor{red}{m_l = 0} \) também, enquanto o orbital \( \textcolor{red}{m_l = +1} \) permanece vazio.}\\[2mm]
		            
		            \textcolor{blue}{Logo, os números quânticos para o conjunto completo de números quânticos permitidos pela teoria quântica são:}
		            \[
			            \textcolor{red}{n = 2}, \quad 
			            \textcolor{red}{l = 1}, \quad 
			            \textcolor{red}{m_l = 0}, \quad 
			            \textcolor{red}{m_s = +1/2}.
		            \]
		            
		      \item[b)] \(4f^{12}\)
		            
		            \textcolor{blue}{No orbital \textcolor{red}{\(4f\)}:}
		            \textcolor{blue}{
			            \begin{itemize}
				            \item \textcolor{blue}{\( \textcolor{red}{n = 4} \) é o número quântico principal.}
				            \item \textcolor{blue}{\( \textcolor{red}{l = 3} \) é o subnível \( \textcolor{red}{f} \).}
				            \item \textcolor{blue}{Valores permitidos de \( \textcolor{red}{m_l} \): \( \textcolor{red}{-3} \), \( \textcolor{red}{-2} \), \( \textcolor{red}{-1} \), \( \textcolor{red}{0} \), \( \textcolor{red}{+1} \), \( \textcolor{red}{+2} \), \( \textcolor{red}{+3} \).}
				            \item \textcolor{blue}{Para cada orbital, \( \textcolor{red}{m_s = +1/2} \) ou \( \textcolor{red}{m_s = -1/2} \).}
			            \end{itemize}
		            }
		            
		            \textcolor{blue}{Para a configuração \( \textcolor{red}{4f^{12}} \), temos doze elétrons. Seguindo a regra de Hund, eles serão distribuídos nos orbitais disponíveis.}
		            
		            \input{orbital_4f12.tex}
		            
		            \textcolor{blue}{Na tabela acima, os orbitais com \( \textcolor{red}{m_l = -3} \), \( \textcolor{red}{m_l = -2} \), \( \textcolor{red}{m_l = -1} \), \( \textcolor{red}{m_l = 0} \) e \( \textcolor{red}{m_l = +1} \) contêm dois elétrons (um com \( \textcolor{red}{m_s = +1/2} \) e outro com \( \textcolor{red}{m_s = -1/2} \)), enquanto os orbitais \( \textcolor{red}{m_l = +2} \) e \( \textcolor{red}{m_l = +3} \) contêm apenas um elétron com \( \textcolor{red}{m_s = +1/2} \).}\\[2mm]
		            
		            
		            \textcolor{blue}{Logo, os números quânticos para o conjunto completo de números quânticos permitidos pela teoria quântica são:}
		            \[
			            \textcolor{red}{n = 4}, \quad 
			            \textcolor{red}{l = 3}, \quad 
			            \textcolor{red}{m_l = +1}, \quad 
			            \textcolor{red}{m_s = -1/2}.
		            \]
	      \end{enumerate}
	      \pagebreak
	\item Qual o número máximo de elétrons no orbital que podem ser identificados a cada um dos seguintes conjuntos de número quânticos:
	      \begin{enumerate}
		      \item[a)] \(n = 6\), \(l = 1\), \(m_l = -1\)
		            \\[10pt]
		            \textcolor{blue}{Primeiramente, analisamos os números quânticos fornecidos:}
		            \textcolor{blue}{
			            \begin{itemize}
				            \item[] \textcolor{red}{\(n = 6\)} indica que o elétron pertence ao sexto nível de energia (camada).
				            \item[] \textcolor{red}{\(l = 1\)} corresponde ao subnível \textcolor{red}{\(p\)}.
				            \item[] \textcolor{red}{\(m_l = -1\)} representa uma das três orientações possíveis para um orbital \textcolor{red}{\(p\)}.
			            \end{itemize}}
		            \textcolor{blue}{\\ No subnível \textcolor{red}{\(p\)}, existem três orbitais \textcolor{red}{\((m_l = -1, 0, +1)\)}, e cada um pode acomodar no máximo \textcolor{red}{2} elétrons (um com \textcolor{red}{\(m_s = +1/2\)} e outro com \textcolor{red}{\(m_s = -1/2\)}).}
		            \\[10pt]
		            \textcolor{blue}{Portanto, para satisfazer os números quânticos listados, a configuração eletrônica é \textcolor{red}{\(6p^1\)} ou \textcolor{red}{\(6p^4\)}.}
		            \\
		            \[ 
    \textcolor{red}{
        \begin{array}{c c c}
            \begin{array}{c}
                \begin{tabular}{|m{0.5cm}|m{0.5cm}|m{0.5cm}|}
                    \hline
                    $\uparrow$ &  &  \\
                    \hline
                \end{tabular} \\
                \begin{tabular}{m{0.5cm} m{0.5cm} m{0.5cm}}
                    $-1$ & \text{ 0 } & $+1$
                \end{tabular}
            \end{array}
             & 
            \quad \text{\textcolor{blue}{ou}} \quad
             & 
            \begin{array}{c}
                \begin{tabular}{|m{0.5cm}|m{0.5cm}|m{0.5cm}|}
                    \hline
                    $\uparrow \downarrow$ & $\uparrow$ &  \\
                    \hline
                \end{tabular} \\
                \begin{tabular}{m{0.5cm} m{0.5cm} m{0.5cm}}
                    $-1$ & \text{ 0 } & $+1$
                \end{tabular}
            \end{array}
        \end{array}
    }
\]
		            
		            
		      \item[b)] \(n = 3\), \(l = 3\), \(m_l = -3\)
		            \\[10pt]
		            \textcolor{blue}{Primeiramente, analisamos os números quânticos fornecidos:}
		            \textcolor{blue}{
			            \begin{itemize}
				            \item[] \textcolor{red}{\(n = 3\)} indica que o elétron pertence ao terceiro nível de energia (camada).
				            \item[] \textcolor{red}{\(l = 3\)} corresponde a um subnível \textcolor{red}{\(f\)}.
				            \item[] \textcolor{red}{\(m_l = -3\)} representa uma das orientações possíveis para um orbital \textcolor{red}{\(f\)}.
			            \end{itemize}}
		            \textcolor{blue}{\\No entanto, no nível \textcolor{red}{\(n = 3\)}, o valor de \textcolor{red}{\(l\)} não pode ser \textcolor{red}{\(3\)}, pois, pelo princípio da mecânica quântica, o valor de \textcolor{red}{\(l\)} deve ser restrito ao intervalo \textcolor{red}{\(0 \leq l \leq n-1\)}. Isso significa que, para \textcolor{red}{\(n = 3\)}, o valor de \textcolor{red}{\(l\)} pode ser \textcolor{red}{\(0\)}, \textcolor{red}{\(1\)} ou \textcolor{red}{\(2\)} (os subníveis \textcolor{red}{\(s\)}, \textcolor{red}{\(p\)} e \textcolor{red}{\(d\)}, respectivamente). Isso ocorre porque, para cada valor de \textcolor{red}{\(n\)}, \textcolor{red}{\(l\)} é limitado a um intervalo que vai de \textcolor{red}{\(0\)} até \textcolor{red}{\(n-1\)}.}
		            \\[10pt]
		            \textcolor{blue}{Portanto, o conjunto de números quânticos fornecido \textcolor{red}{(\( n = 3, \ l = 3, \ m_l = -3\))} \textcolor{red}{não existe}.}
		            \[
			            \textcolor{red}{\nexists}
		            \]
		            
		            
		            
	      \end{enumerate}
	      
	\item Um elétron num certo átomo está no nível quântico \(n = 2\). Indique os valores possíveis de \(l\) e \(m_l\).
	      \\[10pt]
	      \textcolor{blue}{Analisando o número quântico \textcolor{red}{\(n = 2\)}:}
	      \textcolor{blue}{
		      \begin{itemize}
			      \item[] \textcolor{red}{\(n = 2\)} indica que o elétron está no segundo nível de energia (camada).
		      \end{itemize}
	      }
	      \textcolor{blue}{Para \textcolor{red}{\(n = 2\)}, os valores possíveis de \textcolor{red}{\(l\)} são:}
	      \[
		      \textcolor{blue}{\textcolor{red}{l = 0, \ \text{subnível \(s\)}} \text{ e } \textcolor{red}{l = 1, \text{ subnível \(p\)}}}
	      \]
	      \textcolor{blue}{Como \textcolor{red}{\(l\)} deve ser um número inteiro no intervalo \textcolor{red}{\(0 \leq l \leq n-1\)}, para \textcolor{red}{\(n = 2\)} o valor de \textcolor{red}{\(l\)} pode ser \textcolor{red}{0} ou \textcolor{red}{1}.}
	      \\[10pt]
	      \textcolor{blue}{Para cada valor de \textcolor{red}{\(l\)}, os valores possíveis de \textcolor{red}{\(m_l\)} são:}
	      \[
		      \textcolor{red}{l = 0} \quad \textcolor{red}{\Rightarrow} \quad \textcolor{red}{m_l = 0}
	      \]
	      \[
		      \textcolor{red}{l = 1} \quad \textcolor{red}{\Rightarrow} \quad \textcolor{red}{m_l = -1, \ 0,\ +1}
	      \]
	      
	      
	\item Indique qual (ais) dos seguintes conjuntos dos números quânticos para um átomo é (são) inaceitável (eis) e explique por quê.
	      \begin{enumerate}
		      \item[a)] \(3, 0, 0, +1/2\)
		            \\[10pt]
		            \textcolor{blue}{
			            Para verificar a validade do conjunto de números quânticos \(\textcolor{red}{(n = 3,\ l = 0,\ m_l = 0,\ m_s = +1/2)}\), analisamos cada valor individualmente:
			            \begin{itemize}
				            \item[] \textcolor{red}{\(n = 3\)} é válido, pois \(\textcolor{red}{n}\) deve ser um número inteiro positivo \(\textcolor{red}{(n \geq 1)}\).
				            \item[] \textcolor{red}{\(l = 0\)} é válido, pois para \(\textcolor{red}{n = 3}\), os valores possíveis de \(\textcolor{red}{l}\) são \(\textcolor{red}{0, 1, 2}\).
				            \item[] \textcolor{red}{\(m_l = 0\)} é válido, pois quando \(\textcolor{red}{l = 0}\), o único valor permitido para \(\textcolor{red}{m_l}\) é \(\textcolor{red}{0}\).
				            \item[] \textcolor{red}{\(m_s = +1/2\)} é válido, pois \(\textcolor{red}{m_s}\) pode ser \(\textcolor{red}{+1/2}\) ou \(\textcolor{red}{-1/2}\).
			            \end{itemize}
			            \textcolor{blue}{Portanto, o conjunto de números quânticos \(\textcolor{red}{(3,\ 0,\ 0,\ +1/2)}\) é \textcolor{green}{válido}.}
		            }
		            
		      \item[b)] \(2, 2, 1, +1/2\)
		            \\[10pt]
		            \textcolor{blue}{
			            Para verificar a validade do conjunto de números quânticos \(\textcolor{red}{(n = 2,\ l = 2,\ m_l = 1,\ m_s = +1/2)}\), analisamos cada valor individualmente:
			            \begin{itemize}
				            \item[] \textcolor{red}{\(n = 2\)} é válido, pois \(\textcolor{red}{n}\) deve ser um número inteiro positivo \(\textcolor{red}{(n \geq 1)}\).
				            \item[] \textcolor{red}{\(l = 2\)} é inválido, pois para \(\textcolor{red}{n = 2}\), os valores possíveis de \(\textcolor{red}{l}\) são apenas \(\textcolor{red}{0, 1}\), e \(\textcolor{red}{l = 2}\) não é permitido.
				            \item[] \textcolor{red}{\(m_l = 1\)} não pode ser analisado, pois \(\textcolor{red}{l = 2}\) já é inválido para \(\textcolor{red}{n = 2}\).
				            \item[] \textcolor{red}{\(m_s = +1/2\)} é válido, pois \(\textcolor{red}{m_s}\) pode ser \(\textcolor{red}{+1/2}\) ou \(\textcolor{red}{-1/2}\).
			            \end{itemize}
			            \textcolor{blue}{Portanto, o conjunto de números quânticos \(\textcolor{red}{(2,\ 2,\ 1,\ +1/2)}\) é \textcolor{red}{inválido}.}
		            }
		            
		      \item[c)] \(4, 3, -2, +1/2\)
		            \\[10pt]
		            \textcolor{blue}{
			            Para verificar a validade do conjunto de números quânticos \(\textcolor{red}{(n = 4,\ l = 3,\ m_l = -2,\ m_s = +1/2)}\), analisamos cada valor individualmente:
			            \begin{itemize}
				            \item[] \textcolor{red}{\(n = 4\)} é válido, pois \(\textcolor{red}{n}\) deve ser um número inteiro positivo \(\textcolor{red}{(n \geq 1)}\).
				            \item[] \textcolor{red}{\(l = 3\)} é válido, pois para \(\textcolor{red}{n = 4}\), os valores possíveis de \(\textcolor{red}{l}\) são \(\textcolor{red}{0, 1, 2, 3}\).
				            \item[] \textcolor{red}{\(m_l = -2\)} é válido, pois para \(\textcolor{red}{l = 3}\), os valores possíveis de \(\textcolor{red}{m_l}\) são \(\textcolor{red}{-3, -2, -1, 0, +1, +2, +3}\).
				            \item[] \textcolor{red}{\(m_s = +1/2\)} é válido, pois \(\textcolor{red}{m_s}\) pode ser \(\textcolor{red}{+1/2}\) ou \(\textcolor{red}{-1/2}\).
			            \end{itemize}
			            \textcolor{blue}{Portanto, o conjunto de números quânticos \(\textcolor{red}{(4,\ 3,\ -2,\ +1/2)}\) é \textcolor{green}{válido}.}
		            }
	      \end{enumerate}
	      \pagebreak
	\item Qual é a designação (notação) para o subnível \(n = 5\) e \(l = 1\)? Quantos orbitais existem nesse subnível e indique os valores de \(m_l\) para cada um desses orbitais.
	      \\[10pt]
	      \textcolor{blue}{
		      Para identificar a notação do subnível e determinar seus orbitais, analisamos os números quânticos fornecidos:
		      \begin{itemize}
			      \item[] \textcolor{red}{\(n = 5\)} representa a quinta camada de energia.
			      \item[] \textcolor{red}{\(l = 1\)} corresponde ao subnível \(\textcolor{red}{p}\), pois a relação entre \textcolor{red}{\(l\)} e os subníveis é:
			            \begin{itemize}
				            \item[] \(\textcolor{red}{l = 0} \Rightarrow\) subnível \(\textcolor{red}{s}\).
				            \item[] \(\textcolor{red}{l = 1} \Rightarrow\) subnível \(\textcolor{red}{p}\).
				            \item[] \(\textcolor{red}{l = 2} \Rightarrow\) subnível \(\textcolor{red}{d}\).
				            \item[] \(\textcolor{red}{l = 3} \Rightarrow\) subnível \(\textcolor{red}{f}\).
			            \end{itemize}
		      \end{itemize}
		      \textcolor{blue}{Portanto, a notação do subnível é \(\textcolor{red}{5p}\).} \\[2mm]		      
		      \textcolor{blue}{Para \(\textcolor{red}{l = 1}\), o número de orbitais é dado pela equação \textcolor{red}{\(2l + 1\)}:}
		      \[
			      \textcolor{red}{2(1) + 1 = 3}
		      \]
		      \textcolor{blue}{Assim, o subnível \(\textcolor{red}{5p}\) contém \(\textcolor{red}{3}\) orbitais.} \\[2mm]		      
		      \textcolor{blue}{Os valores possíveis de \(\textcolor{red}{m_l}\) representam as diferentes orientações dos orbitais no espaço. Para o subnível \(\textcolor{red}{p}\), existem três orbitais, que possuem os seguintes valores:}
		      \[
			      \textcolor{red}{m_l = -1, 0, +1.}
		      \]
	      }
	\item Em relação aos números quânticos responda:
	      \begin{enumerate}
		      \item[a)] A notação da subcamada e o número de orbitais para: \(\{n = 6,\ l = 1\}\), \(\{n = 5,\ l = 3\}\), \(\{n = 4,\ l = 2\}\).
		      \item[] \textcolor{blue}{
			            Para cada conjunto de números quânticos:
			            \begin{itemize}
				            \item[] Para \textcolor{red}{\(\{n = 6,\ l = 1\}\)}:
				                  \begin{itemize}
					                  \item[] \textcolor{red}{\(n = 6\)} representa a sexta camada de energia.
					                  \item[] \textcolor{red}{\(l = 1\)} corresponde ao subnível \(\textcolor{red}{p}\).
					                  \item[] Assim, a notação é \(\textcolor{red}{6p}\) e o número de orbitais é \(\textcolor{red}{2(1)+1 = 3}\).
					                  \item[] Os valores de \(\textcolor{red}{m_l}\) são: \(\textcolor{red}{-1,\ 0,\ +1}\).
				                  \end{itemize}
				            \item[] Para \textcolor{red}{\(\{n = 5,\ l = 3\}\)}:
				                  \begin{itemize}
					                  \item[] \textcolor{red}{\(n = 5\)} representa a quinta camada de energia.
					                  \item[] \textcolor{red}{\(l = 3\)} corresponde ao subnível \(\textcolor{red}{f}\).
					                  \item[] Assim, a notação é \(\textcolor{red}{5f}\) e o número de orbitais é \(\textcolor{red}{2(3)+1 = 7}\).
					                  \item[] Os valores de \(\textcolor{red}{m_l}\) são: \(\textcolor{red}{-3,\ -2,\ -1,\ 0,\ +1,\ +2,\ +3}\).
				                  \end{itemize}
				            \item[] Para \textcolor{red}{\(\{n = 4,\ l = 2\}\)}:
				                  \begin{itemize}
					                  \item[] \textcolor{red}{\(n = 4\)} representa a quarta camada de energia.
					                  \item[] \textcolor{red}{\(l = 2\)} corresponde ao subnível \(\textcolor{red}{d}\).
					                  \item[] Assim, a notação é \(\textcolor{red}{4d}\) e o número de orbitais é \(\textcolor{red}{2(2)+1 = 5}\).
					                  \item[] Os valores de \(\textcolor{red}{m_l}\) são: \(\textcolor{red}{-2,\ -1,\ 0,\ +1,\ +2}\).
				                  \end{itemize}
			            \end{itemize}
		            }
		            \pagebreak       
		      \item[b)] A ordem crescente de energia dos conjuntos de número quânticos de (a).
		            \\[10pt]
		            \textcolor{blue}{
			            A energia dos subníveis pode ser aproximada pela soma \textcolor{red}{\(n + l\)}. Assim, calculamos:
			            \begin{itemize}
				            \item[] Para \(\textcolor{red}{\{n = 4, l = 2\}}\): \(\textcolor{red}{n + l = 4 + 2 = 6}\).
				            \item[] Para \(\textcolor{red}{\{n = 6, l = 1\}}\): \(\textcolor{red}{n + l = 6 + 1 = 7}\).
				            \item[] Para \(\textcolor{red}{\{n = 5, l = 3\}}\): \(\textcolor{red}{n + l = 5 + 3 = 8}\).
			            \end{itemize}
			            \textcolor{blue}{Portanto, em ordem crescente de energia, temos:}
			            \[
				            \textcolor{red}{(n = 4,\ l = 2) \, (4d) < (n = 6,\ l = 1) \, (6p) < (n = 5,\ l = 3) \, (5f).}
			            \]
		            }
	      \end{enumerate}
	      
	\item Com relação aos números quânticos responda:
	      \begin{enumerate}
		      \item[a)] O número de elétrons no orbital \(p\) do terceiro nível do elemento de número atômico 16.
		            \\[10pt]
		            \textcolor{blue}{
			            O elemento de número atômico \(\textcolor{red}{16}\) é o enxofre \(\textcolor{red}{(S)}\), cuja configuração eletrônica é:
			            \[
				            \textcolor{red}{1s^2 \ 2s^2 \ 2p^6 \ 3s^2  \ 3p^4}
			            \]
			            \textcolor{blue}{O orbital \textcolor{red}{\(p\)} do terceiro nível corresponde ao subnível \(\textcolor{red}{3p}\).} \\[2mm]
			            \textcolor{blue}{Observamos que \(\textcolor{red}{3p}\) contém \(\textcolor{red}{4}\) elétrons. Portanto, a resposta é:}
			            \[
				            \textcolor{red}{4 \text{ $e^{-}$}}
			            \]
		            }
		      \item[b)] O conjunto dos quatro números quânticos para o último elétron de um átomo neutro, cuja configuração é: \(1s^2 \ 2s^2 \ 2p^6 \ 3s^2 \ 3p^6 \ 4s^2\).
		            \\[10pt]
		            \textcolor{blue}{
			            A configuração eletrônica fornecida termina em \(\textcolor{red}{4s^2}\), indicando que o último elétron ocupa o orbital \(\textcolor{red}{4s}\). \\[2mm]
			            \textcolor{blue}{Os números quânticos desse elétron são:}
			            \begin{itemize}
				            \item[] \textcolor{red}{\(n = 4\)} representa a quarta camada de energia.
				            \item[] \textcolor{red}{\(l = 0\)} indica que se trata de um orbital \(\textcolor{red}{s}\).
				            \item[] \textcolor{red}{\(m_l = 0\)}, pois no subnível \(\textcolor{red}{s}\), só existe um orbital possível.
				            \item[] \textcolor{red}{\(m_s = -1/2\)}, pois é o segundo elétron a ocupar esse orbital.
			            \end{itemize}
			            \textcolor{blue}{Portanto, o conjunto dos quatro números quânticos para o último elétron é:}
			            \[
				            \textcolor{red}{n = 4, \ l = 0, \ m_l = 0, \ m_s = -1/2}
			            \]
		            }
	      \end{enumerate}
	      \pagebreak
	\item Quais são os quatro números quânticos do último elétron representado, seguindo a regra de Hund, ao efetuar a representação gráfica de 9 elétrons no subnível \(4f\)?
	      \\[10pt]
	      \textcolor{blue}{
		      O subnível \(\textcolor{red}{4f}\) possui \(\textcolor{red}{7}\) orbitais, correspondendo aos seguintes valores de \(\textcolor{red}{m_l}\):
		      \[
			      \textcolor{red}{m_l = -3, \ -2, \ -1, \  0, \ +1, \ +2, \ +3.}
		      \]
		      \\
		      \textcolor{blue}{Cada orbital pode conter no máximo \(\textcolor{red}{2}\) elétrons, um com spin \(\textcolor{red}{+1/2}\) e outro com \(\textcolor{red}{-1/2}\).} \\[2mm]
		      \textcolor{blue}{Seguindo a regra de Hund, ao distribuir \(\textcolor{red}{9}\) elétrons no subnível \(\textcolor{red}{4f}\), primeiro ocupamos os \(\textcolor{red}{7}\) orbitais com elétrons de spins paralelos \(\textcolor{red}{(+1/2)}\), e depois começamos a emparelhar os orbitais já preenchidos.} \\[2mm]
		      \textcolor{blue}{O nono elétron será o primeiro a emparelhar, ou seja, ocupará um orbital já preenchido com spin \(\textcolor{red}{+1/2}\), mas agora com spin \(\textcolor{red}{-1/2}\).} \\[2mm]
		      \[
    \textcolor{red}{
        \begin{array}{c}
            \begin{tabular}{|m{0.5cm}|m{0.5cm}|m{0.5cm}|m{0.5cm}|m{0.5cm}|m{0.5cm}|m{0.5cm}|}
                \hline
                $\uparrow \downarrow$ & $\uparrow \textcolor{green}{\downarrow}$ & $\uparrow$ & $\uparrow$ & $\uparrow$ & $\uparrow$ & $\uparrow$ \\
                \hline
            \end{tabular} \\
            \begin{tabular}{m{0.5cm} m{0.5cm} m{0.5cm} m{0.5cm} m{0.5cm} m{0.5cm} m{0.5cm}}
                $-3$ & $-2$ & $-1$ & \text{ 0 } & $+1$ & $+2$ & $+3$
            \end{tabular}
        \end{array}
    }
\]
		      \\
		      \textcolor{blue}{Assim, os números quânticos do último elétron representado são:}
		      \[
			      \textcolor{red}{n = 4, \ l = 3, \ m_l = -2, \ m_s = -1/2.}
		      \]
	      }
	      
	\item Qual o conjunto dos quatro números quânticos para o elétron diferenciador do íon \(^{20}\text{Ca}^{+2}\)?
	      \\[10pt]
	      \textcolor{blue}{
		      O elemento \(\textcolor{red}{^{20}\text{Ca}}\) (Cálcio) possui número atômico \(\textcolor{red}{20}\), e sua configuração eletrônica no estado fundamental é:
		      \[
			      \textcolor{red}{1s^2 \ 2s^2 \ 2p^6 \ 3s^2 \ 3p^6 \ 4s^2}
		      \]
		      \textcolor{blue}{O íon \(\textcolor{red}{Ca^{+2}}\) resulta da remoção de dois elétrons da camada mais externa, ou seja, os elétrons do subnível \(\textcolor{red}{4s}\) são removidos. Assim, a configuração eletrônica do \(\textcolor{red}{Ca^{+2}}\) fica:}
		      \[
			      \textcolor{red}{1s^2 \ 2s^2 \ 2p^6 \ 3s^2 \ 3p^6}
		      \]
		      \textcolor{blue}{O elétron diferenciador agora é o último elétron adicionado ao estado fundamental neutro, ou seja, o último elétron de \(\textcolor{red}{3p^6}\).} \\[2mm]		      
		      \textcolor{blue}{Os números quânticos desse elétron são:}
		      \begin{itemize}
			      \item[] \textcolor{red}{\(n = 3\)} representa a terceira camada de energia.
			      \item[] \textcolor{red}{\(l = 1\)} indica que o elétron está em um orbital \(\textcolor{red}{p}\).
			      \item[] \textcolor{red}{\(m_l = 0\)}, pois o elétron pode estar em qualquer um dos três orbitais \textcolor{red}{\(p\)} \(\textcolor{red}{(m_l = -1, \ 0, \ +1)}\), e por convenção, escolhemos \(\textcolor{red}{-1}\).
			      \item[] \textcolor{red}{\(m_s = -1/2\)}, pois é o segundo elétron a ocupar esse orbital.
		      \end{itemize}
		      \input{orbital_3p6.tex}
		      \textcolor{blue}{Portanto, o conjunto dos quatro números quânticos para o elétron diferenciador do \(\textcolor{red}{Ca^{+2}}\) é:}
		      \[
			      \textcolor{red}{n = 3, \ l = 1, \ m_l = -1, \ m_s = -1/2.}
		      \]
	      }
	      
\end{enumerate}

\end{document}