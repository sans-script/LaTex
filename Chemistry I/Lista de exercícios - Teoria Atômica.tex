\documentclass[a4paper, 12pt]{article}
\usepackage[utf8]{inputenc}            % Codificação do documento
\usepackage[T1]{fontenc}               % Encodings para caracteres especiais
\usepackage[brazil]{babel}             % Idioma do documento
\usepackage{geometry}                  % Para ajustar margens
\usepackage{graphicx}                  % Para incluir imagens
\usepackage{amsmath, amsfonts, amsthm, amssymb} % Para símbolos matemáticos
\usepackage{fancyhdr}                  % Para cabeçalhos e rodapés
\usepackage[dvipsnames]{xcolor}        % Para cores
\usepackage{thmtools}                  % Para definições de teoremas
\usepackage{titlesec}                  % Para personalizar títulos de seções
\usepackage{booktabs}                  % Adicione isso no preâmbulo
\usepackage{tabularx}                  % Para tabelas com largura ajustável
\usepackage{mathptmx}                  % Para usar Times New Roman
\usepackage{enumitem} % Para personalizar a lista
\usepackage{cancel} % Para cancelar termos em equações
\usepackage[version=4]{mhchem} % Para escrever fórmulas químicas
\usepackage{chemfig} % Para desenhar moléculas
\usepackage{colortbl} % Para colorir células de tabelas
\usepackage{multicol} % Para dividir o texto em colunas
\usepackage{tikz} % Para desenhar gráficos
\usepackage{tikz}
\usepackage{pgf-spectra}
\usepackage{siunitx}


% Configuração das margens
\geometry{top=2cm, bottom=2cm, left=1cm, right=1cm}

\titleformat{\section}
{\normalfont\Large\bfseries}{\thesection}{0.6em}{}

\titleformat{\subsection}
{\normalfont\large\bfseries}{\thesubsection}{0.3em}{}

\setcounter{secnumdepth}{0}  % Remove a numeração de seções
\def\checkmark{\tikz\fill[scale=0.4](0,.35) -- (.25,0) -- (1,.7) -- (.25,.15) -- cycle;}
\definecolor{indigo}{rgb}{0.29, 0.0, 0.51}

% Configura cabeçalhos e rodapés
\pagestyle{fancy}
\fancyhf{} % Limpa cabeçalhos e rodapés
\fancyhead[R]{Exercícios de Fixação} % Nome à direita no cabeçalho
\fancyhead[L]{Alexandre Santos}
\fancyfoot[R]{\thepage} % Número da página à direita no rodapé
\begin{document}

\subsection*{Natureza Ondulatória da Luz}

\begin{enumerate}
	\item Quando determinadas quantidades de gás xenônio são adicionadas às luzes de néon, sua cor se torna azul-esverdeada. Se o comprimento de onda dessa luz é 480 nm, qual é a frequência?
	      
	      \textcolor{blue}{
		      Para calcular a frequência da luz, utilizamos a relação entre a velocidade da luz \textcolor{red}{(\(c\))}, o comprimento de onda \textcolor{red}{(\(\lambda\))} e a frequência \textcolor{red}{(\(f\))}:
		      \[
			      \textcolor{red}{c = \lambda \times f}
		      \]
		      Sabemos que a velocidade da luz no vácuo é \textcolor{red}{\(c = 3,00 \times 10^8 \, \text{m/s}\)} e o comprimento de onda é \textcolor{red}{\(\lambda = 480 \, \text{nm}\)}. Representado por \textcolor{red}{\(480 \times 10^{-9} \, \text{m}\)}. Portanto, a frequência \textcolor{red}{\(f\)} é dada por:
		      \[
			      \textcolor{red}{f = \frac{c}{\lambda} = \frac{3,00 \times 10^8 \, \text{m/s}}{480 \times 10^{-9} \, \text{m}} = 6,25 \times 10^{14} \, \text{Hz}}
		      \]
		      A frequência da luz azul-esverdeada é \textcolor{red}{\(6,25 \times 10^{14} \, \text{Hz}\)}.
	      }
	      
	\item Um laser infravermelho para uso de uma rede de comunicações de fibra ótica emite um comprimento de onda de 1,2 µm. Qual é a energia de um fóton para esta radiação?
	      
	      \textcolor{blue}{
		      Para calcular a energia de um fóton, utilizamos a equação de Planck:
		      \[
			      \textcolor{red}{E = h \times f}
		      \]
		      Onde \textcolor{red}{\(E\)} é a energia do fóton, \textcolor{red}{\(h\)} é a constante de Planck \textcolor{red}{(\(h = 6,626 \times 10^{-34} \, \text{J} \cdot \text{s}\))} e \textcolor{red}{\(f\)} é a frequência da radiação.      
		      Primeiro, calculamos a frequência \textcolor{red}{\(f\)} a partir do comprimento de onda \textcolor{red}{\(\lambda = 1,2 \, \text{µm} = 1,2 \times 10^{-6} \, \text{m}\)}, utilizando a relação:
		      \[
			      \textcolor{red}{f = \frac{c}{\lambda}}
		      \]
		      Onde \textcolor{red}{\(c = 3,00 \times 10^8 \, \text{m/s}\)} é a velocidade da luz. Substituindo os valores:
		      \[
			      \textcolor{red}{f = \frac{3,00 \times 10^8 \, \text{m/s}}{1,2 \times 10^{-6} \, \text{m}} = 2,50 \times 10^{14} \, \text{Hz}}
		      \]
		      Agora, calculamos a energia do fóton:
		      \[
			      \textcolor{red}{E = h \times f = 6,626 \times 10^{-34} \, \text{J} \cdot \text{s} \times 2,50 \times 10^{14} \, \text{Hz} = 1,66 \times 10^{-19} \, \text{J}}
		      \]
		      Portanto, a energia de um fóton para esta radiação é \textcolor{red}{\(1,66 \times 10^{-19} \, \text{J}\)}.
	      }
	      
	\item Uma dada forma de radiação eletromagnética tem uma frequência de \(8,11 \times 10^{14} \, \text{s}^{-1}\).
	      \begin{enumerate}
		      \item[a)] Qual o seu comprimento de onda em nanômetro?
		            
		            \textcolor{blue}{
			            Para calcular o comprimento de onda \textcolor{red}{(\(\lambda\))}, utilizamos a relação entre a velocidade da luz \textcolor{red}{(\(c\))}, a frequência \textcolor{red}{(\(f\))} e o comprimento de onda:
			            \[
				            \textcolor{red}{\lambda = \frac{c}{f}}
			            \]
			            Onde \textcolor{red}{\(c = 3,00 \times 10^8 \, \text{m/s}\)} é a velocidade da luz e \textcolor{red}{\(f = 8,11 \times 10^{14} \, \text{s}^{-1}\)} é a frequência. Substituindo os valores:
			            \[
				            \textcolor{red}{\lambda = \frac{3,00 \times 10^8 \, \text{m/s}}{8,11 \times 10^{14} \, \text{s}^{-1}} = 3,699 \times 10^{-7} \, \text{m}}
			            \]
			            Convertendo o comprimento de onda para nanômetros \textcolor{red}{(1 nm = \(10^{-9}\) m)}:
			            \[
				            \textcolor{red}{\lambda = 3,699 \times 10^{-7} \times 10^9 = 3,69 \times 10^2 \, \text{ nm}}
			            \]
		            }
		      \item[b)] Qual a energia (em Joule) de um quantum dessa radiação?
		            
		            \textcolor{blue}{
			            A energia de um quantum (fóton) é dada pela equação de Planck:
			            \[
				            \textcolor{red}{E = h \times f}
			            \]
			            Onde \textcolor{red}{\(h = 6,626 \times 10^{-34} \, \text{J} \cdot \text{s}\)} é a constante de Planck e \textcolor{red}{\(f = 8,11 \times 10^{14} \, \text{s}^{-1}\)} é a frequência. Substituindo os valores:
			            \[
				            \textcolor{red}{E = 6,626 \times 10^{-34} \, \text{J} \cdot \text{s} \times 8,11 \times 10^{14} \, \text{s}^{-1} = 5,376 \times 10^{-19} \, \text{J}}
			            \]
			            Arredondando para duas casas decimais, a energia de um quantum dessa radiação é \textcolor{red}{\(5,38 \times 10^{-19} \, \text{J}\)}.
		            }
	      \end{enumerate}
	      
	\item Uma análise espectral cuidadosa mostra que a luz amarela das lâmpadas de sódio (usadas nos postes de rua) é uma mistura de fótons de dois comprimentos de onda, \textcolor{red}{589,0 nm} e \textcolor{red}{589,6 nm}. Qual é a diferença de energia (em Joule) entre os fótons com estes comprimentos de onda?
	      
	      \textcolor{blue}{
		      Para calcular a diferença de energia entre os fótons, primeiro determinamos a energia de cada fóton usando a equação de Planck:
		      \[
			      \textcolor{red}{E = \frac{h \cdot c}{\lambda}}
		      \]
		      Onde:
		      \begin{itemize}
			      \item[] \textcolor{red}{\(h = 6,626 \times 10^{-34} \, \text{J} \cdot \text{s}\)} é a constante de Planck.
			      \item[] \textcolor{red}{\(c = 3,00 \times 10^8 \, \text{m/s}\)} é a velocidade da luz.
			      \item[] \textcolor{red}{\(\lambda\)} é o comprimento de onda.
		      \end{itemize}
		      Para o comprimento de onda de \textcolor{red}{589,0 nm}:
		      \[
			      \textcolor{red}{\lambda_1 = 589,0 \, \text{nm} = 589,0 \times 10^{-9} \, \text{m}}
		      \]
		      \[
			      \textcolor{red}{E_1 = \frac{6,626 \times 10^{-34} \times 3,00 \times 10^8}{589,0 \times 10^{-9}} = 3,375 \times 10^{-19} \, \text{J}}
		      \]
		      Para o comprimento de onda de \textcolor{red}{589,6 nm}:
		      \[
			      \textcolor{red}{\lambda_2 = 589,6 \, \text{nm} = 589,6 \times 10^{-9} \, \text{m}}
		      \]
		      \[
			      \textcolor{red}{E_2 = \frac{6,626 \times 10^{-34} \times 3,00 \times 10^8}{589,6 \times 10^{-9}} = 3,371 \times 10^{-19} \, \text{J}}
		      \]
		      A diferença de energia entre os fótons é:
		      \[
			      \textcolor{red}{\Delta E = E_1 - E_2 = 3,375 \times 10^{-19} \, \text{J} - 3,370 \times 10^{-19} \, \text{J} = 4,00 \times 10^{-22} \, \text{J}}
		      \]
		      Portanto, a diferença de energia entre os fótons com comprimentos de onda de \textcolor{red}{589,0 nm} e \textcolor{red}{589,6 nm} é \textcolor{red}{\(4,00 \times 10^{-22} \, \text{J}\)}.
	      }
	      \pagebreak
	\item Calcule a frequência em Hertz, a energia em joules de um fóton de raios X que tem comprimento de onda de \textcolor{red}{2,70 Å}.
	      
	      \textcolor{blue}{
		      Para calcular a frequência e a energia do fóton, utilizamos as seguintes relações:
		      \begin{itemize}
			      \item[] A frequência \textcolor{red}{(\(f\))} é dada por:
			            \[
				            \textcolor{red}{f = \frac{c}{\lambda}}
			            \]
			      \item[] A energia \textcolor{red}{(\(E\))} de um fóton é dada pela equação de Planck:
			            \[
				            \textcolor{red}{E = h \cdot f}
			            \]
			            Onde:
			            \begin{itemize}
				            \item[] \textcolor{red}{\(c = 3,00 \times 10^8 \, \text{m/s}\)} é a velocidade da luz.
				            \item[] \textcolor{red}{\(h = 6,626 \times 10^{-34} \, \text{J} \cdot \text{s}\)} é a constante de Planck.
				            \item[] \textcolor{red}{\(\lambda\)} é o comprimento de onda.
			            \end{itemize}
		      \end{itemize}
		      Primeiro, convertemos o comprimento de onda de angstroms \textcolor{red}{(Å)} para metros \textcolor{red}{(m)}:
		      \[
			      \textcolor{red}{\lambda = 2,70 \, \text{Å} = 2,70 \times 10^{-10} \, \text{m}}
		      \]
		      \text{Cálculo da frequência:}
		      \[
			      \textcolor{red}{f = \frac{c}{\lambda} = \frac{3,00 \times 10^8 \, \text{m/s}}{2,70 \times 10^{-10} \, \text{m}} = 1,11 \times 10^{18} \, \text{Hz}}
		      \]
		      \text{Cálculo da energia:}
		      \[
			      \textcolor{red}{E = h \cdot f = 6,626 \times 10^{-34} \, \text{J} \cdot \text{s} \times 1,11 \times 10^{18} \, \text{Hz} = 7,36 \times 10^{-16} \, \text{J}}
		      \]
		      Portanto:
		      \begin{itemize}
			      \item[] A frequência do fóton de raios X é \textcolor{red}{\(1,11 \times 10^{18} \, \text{Hz}\)}.
			      \item[] A energia do fóton de raios X é \textcolor{red}{\(7,36 \times 10^{-16} \, \text{J}\)}.
		      \end{itemize}
	      }
	      
	\item Analise as seguintes informações sobre a radiação eletromagnética e determine se são verdadeiras ou falsas. Se forem falsas, corrija-as.
	      \begin{enumerate}
		      \item[\textcolor{green}{a)}] \textcolor{green}{(V) Numa superfície metálica nenhum elétron é ejetado até que a radiação tenha frequência igual ou acima de um determinado valor, característico da ligação do elétron com o metal.}
		      \item[] \textcolor{blue}{
			            Justificativa: A afirmação é verdadeira, pois descreve corretamente o efeito fotoelétrico. Segundo esse fenômeno, a radiação incidente só pode ejetar elétrons de uma superfície metálica se sua frequência for maior ou igual à frequência de limiar, que depende da energia necessária para liberar o elétron do metal.
		            }
		      \item[\textcolor{red}{b)}] \textcolor{red}{(F) A energia de um fóton é diretamente proporcional ao comprimento de onda da radiação.}
		      \item[] \textcolor{blue}{
			            Justificativa: A afirmação é falsa. A energia de um fóton é inversamente proporcional ao comprimento de onda, como descrito pela fórmula \( E = \frac{h \cdot c}{\lambda} \), onde \( h \) é a constante de Planck, \( c \) a velocidade da luz e \( \lambda \) o comprimento de onda. Isso significa que quanto maior o comprimento de onda, menor a energia do fóton.
		            }
		            \pagebreak
		      \item[\textcolor{red}{c)}] \textcolor{red}{(F) Num espectro eletromagnético a faixa de frequência das radiações infravermelha e da ultravioleta é na ordem de \(10^{14} \, \text{s}^{-1}\) e \(10^{16} \, \text{s}^{-1}\), respectivamente, portanto a radiação ultravioleta é de menor energia.}
		      \item[] \textcolor{blue}{
			            Justificativa: A afirmação é falsa. A radiação ultravioleta tem frequências mais altas do que a radiação infravermelha, o que implica em maior energia. De acordo com a fórmula \( E = h \cdot f \), onde \( f \) é a frequência, a radiação ultravioleta possui maior energia do que a infravermelha, pois sua frequência é maior.
		            }
	      \end{enumerate}
	      
	\item Calcule e compare a energia de um fóton de comprimento de onda de 3,3 µm com um de comprimento de onda de 0,154 nm.
	      
	      \textcolor{blue}{
		      Para calcular a energia de um fóton, utilizamos a equação de Planck:
		      \[
			      \textcolor{red}{E = \frac{h \cdot c}{\lambda}}
		      \]
		      Onde:
		      \begin{itemize}
			      \item[] \textcolor{red}{\(h = 6,626 \times 10^{-34} \, \text{J} \cdot \text{s}\)} é a constante de Planck.
			      \item[] \textcolor{red}{\(c = 3,00 \times 10^8 \, \text{m/s}\)} é a velocidade da luz.
			      \item[] \textcolor{red}{\(\lambda\)} é o comprimento de onda.
		      \end{itemize}
		      \text{Fóton 1: Comprimento de onda de \textcolor{red}{3,3 µm}}
		      Primeiro, convertemos o comprimento de onda de micrômetros \textcolor{red}{(µm)} para metros \textcolor{red}{(m)}:
		      \[
			      \textcolor{red}{\lambda_1 = 3,3 \, \text{µm} = 3,3 \times 10^{-6} \, \text{m}}
		      \]
		      Agora, calculamos a energia do fóton:
		      \[
			      \textcolor{red}{E_1 = \frac{6,626 \times 10^{-34} \times 3,00 \times 10^8}{3,3 \times 10^{-6}} = 6,02 \times 10^{-20} \, \text{J}}
		      \]
		      \text{Fóton 2: Comprimento de onda de \textcolor{red}{ 0,154 nm}}
		      Primeiro, convertemos o comprimento de onda de nanômetros \textcolor{red}{(nm)} para metros \textcolor{red}{(m)}:
		      \[
			      \textcolor{red}{\lambda_2 = 0,154 \, \text{nm} = 0,154 \times 10^{-9} \, \text{m}}
		      \]
		      Agora, calculamos a energia do fóton:
		      \[
			      \textcolor{red}{E_2 = \frac{6,626 \times 10^{-34} \times 3,00 \times 10^8}{0,154 \times 10^{-9}} = 1,29 \times 10^{-15} \, \text{J}}
		      \]
		      \text{Comparação das energias:}
		      \begin{itemize}
			      \item[] A energia do fóton com comprimento de onda de \textcolor{red}{3,3 µm} é \textcolor{red}{\(6,02 \times 10^{-20} \, \text{J}\)}.
			      \item[] A energia do fóton com comprimento de onda de \textcolor{red}{0,154 nm} é \textcolor{red}{\(1,29 \times 10^{-15} \, \text{J}\)}.
		      \end{itemize}
		      \[
			      \textcolor{red}{\frac{E_2}{E_1} = \frac{1,29 \times 10^{-15}}{6,02 \times 10^{-20}} = 2,14 \times 10^4}
		      \]
		      Portanto, o fóton com comprimento de onda de \textcolor{red}{0,154 nm} tem uma energia \textcolor{red}{\(2,14 \times 10^4\)} vezes maior que o fóton com comprimento de onda de \textcolor{red}{3,3 µm}.
	      }
	      \pagebreak
	      
	\item Calcule o comprimento de onda (em nanômetro) do fóton emitido quando o elétron do átomo de hidrogênio decai do nível \(n = 5\) para o nível \(n = 2\).
	      
	      \textcolor{blue}{
	      Para calcular o comprimento de onda do fóton emitido, utilizamos a fórmula de Rydberg para o átomo de hidrogênio:
	      \[
		      \textcolor{red}{\frac{1}{\lambda} = R_H \left( \frac{1}{n_f^2} - \frac{1}{n_i^2} \right)}
	      \]
	      Onde:
	      \begin{itemize}
		      \item[] \textcolor{red}{\(\lambda\)} é o comprimento de onda do fóton emitido.
		      \item[] \textcolor{red}{\(R_H = 1,097 \times 10^7 \, \text{m}^{-1}\)} é a constante de Rydberg para o hidrogênio.
		      \item[] \textcolor{red}{\(n_i = 5\)} é o nível inicial.
		      \item[] \textcolor{red}{\(n_f = 2\)} é o nível final.
	      \end{itemize}
	      Substituindo os valores na fórmula:
	      \[
		      \textcolor{red}{\frac{1}{\lambda} = 1,097 \times 10^7 \left( \frac{1}{2^2} - \frac{1}{5^2} \right)}
	      \]
	      \[
		      \textcolor{red}{\frac{1}{\lambda} = 1,097 \times 10^7 \left( \frac{1}{4} - \frac{1}{25} \right)}
	      \]
	      \[
		      \textcolor{red}{\frac{1}{\lambda} = 1,097 \times 10^7 \left( 0,25 - 0,04 \right)}
	      \]
	      \[
		      \textcolor{red}{\frac{1}{\lambda} = 1,097 \times 10^7 \times 0,21}
	      \]
	      \[
		      \textcolor{red}{\frac{1}{\lambda} = 2,3037 \times 10^6 \, \text{m}^{-1}}
	      \]
	      Agora, calculamos o comprimento de onda \(\lambda\):
	      \[
		      \textcolor{red}{\lambda = \frac{1}{2,3037 \times 10^6} = 4,34 \times 10^{-7} \, \text{m}}
	      \]
	      Convertendo o comprimento de onda para nanômetros {(1 nm = \(10^{-9}\) m)}:
	      \[
		      \textcolor{red}{\lambda = 4,34 \times 10^{-7} \, \text{m} \times 10^9 \, \text{nm/m} = 434 \, \text{nm}}
	      \]
	      Apresentando o resultado em notação científica:
	      \[
		      \textcolor{red}{\lambda = 4,34 \times 10^{2} \, \text{nm}}
	      \]
	      Portanto, o comprimento de onda do fóton emitido é \textcolor{red}{\(434 \, \text{nm}\)} ou \textcolor{red}{\(4,34 \times 10^{2} \, \text{nm}\)}.
	      }
	      
	      
	      \pagebreak
	\item As cores de luz exibidas na queima de fogos de artifício dependem de certas substâncias utilizadas na sua fabricação. Sabe-se que a frequência da luz emitida pela combustão do níquel é \(6,0 \times 10^{14} \, \text{Hz}\) e que a velocidade da luz é \(3 \times 10^{8} \, \text{m/s}\). Com base nesses dados e no espectro visível fornecido pela figura a seguir, diga qual a cor da luz dos fogos de artifício que contêm compostos de níquel.
	      
	      \textcolor{blue}{
		      Para determinar a cor da luz emitida pelo níquel, primeiro calculamos seu comprimento de onda utilizando a relação entre a velocidade da luz \textcolor{red}{\(c\)}, o comprimento de onda \textcolor{red}{\(\lambda\)} e a frequência \textcolor{red}{\(f\)}:
		      \[
			      \textcolor{red}{c = \lambda \cdot f}
		      \]
		      Isolando o comprimento de onda:
		      \[
			      \textcolor{red}{\lambda = \frac{c}{f}}
		      \]
		      Substituindo os valores:
		      \[
			      \textcolor{red}{\lambda = \frac{3,0 \times 10^8 \, \text{m/s}}{6,0 \times 10^{14} \, \text{Hz}}}
		      \]
		      \[
			      \textcolor{red}{\lambda = 5,0 \times 10^{-7} \, \text{m}}
		      \]
		      Convertendo para nanômetros \textcolor{red}{{(\(1 \, \text{nm} = 10^{-9} \, \text{m}\))}}:
		      \[
			      \textcolor{red}{\lambda = 5,0 \times 10^{-7} \, \text{m} \times 10^9 \, \text{nm/m}}
		      \]
		      \[
			      \textcolor{red}{\lambda = 500 \, \text{nm}}
		      \]
		      Consultando o espectro visível, a luz com comprimento de onda em torno de \textcolor{red}{\(500 \, \text{nm}\)} corresponde à cor \textcolor{green}{verde}.
	      }
	      
	      \begin{tikzpicture}
		      \node[
		      below=4em,
		      align=center,
		      label=above:{\text{Espectro Visível}}
		      ] (FULL_VISIBLE_LIGHT) {
		      \pgfspectra\\
		      {\num{400}\, \text{nm} \hfill \num{450}\, \text{nm} \hfill \num{500}\, \text{nm} \hfill \num{550}\, \text{nm} \hfill \num{580}\, \text{nm} \hfill \num{600}\, \text{nm} \hfill \num{650}\, \text{nm}}
		      };
	      \end{tikzpicture}
	      
	      \pagebreak   
	\item Um átomo de hidrogênio está em um estado excitado com \(n = 2\), com uma energia \(E_2 = -3,4 \, \text{eV}\). Ocorre uma transição para o estado \(n = 1\), com energia \(E_1 = -13,6 \, \text{eV}\), e um fóton é emitido. Qual a frequência da radiação emitida em Hz? (Dados: \(1 \, \text{eV} = 1,6 \times 10^{-19} \, \text{J}\); \(h = 6,63 \times 10^{-34} \, \text{Js}\).)
	      
	      \textcolor{blue}{
		      Para calcular a frequência da radiação emitida, utilizamos a diferença de energia entre os estados \textcolor{red}{\(n = 2\) }e \textcolor{red}{\(n = 1\)} e a relação de Planck:
		      \[
			      \textcolor{red}{E = h \cdot f}
		      \]
		      Onde:
		      \begin{itemize}
			      \item[] \textcolor{red}{\(E\)} é a energia do fóton emitido.
			      \item[] \textcolor{red}{\(h = 6,63 \times 10^{-34} \, \text{Js}\)} é a constante de Planck.
			      \item[] \textcolor{red}{\(f\)} é a frequência da radiação emitida.
		      \end{itemize}
		      A diferença de energia \textcolor{red}{(\(\Delta E\))} entre os estados \textcolor{red}{\(n = 2\) }e \textcolor{red}{\(n = 1\)} é:
		      \[
			      \textcolor{red}{\Delta E = E_1 - E_2 = -13,6 \, \text{eV} - (-3,4 \, \text{eV}) = -10,2 \, \text{eV}}
		      \]
		      Como a energia do fóton emitido é positiva, consideramos o valor absoluto:
		      \[
			      \textcolor{red}{\Delta E = 10,2 \, \text{eV}}
		      \]
		      Convertemos a energia de elétron-volts \textcolor{red}{(eV)} para joules \textcolor{red}{(J)} usando a conversão \textcolor{red}{\(1 \, \text{eV} = 1,6 \times 10^{-19} \, \text{J}\)}:
		      \[
			      \textcolor{red}{\Delta E = 10,2 \, \text{eV} \times 1,6 \times 10^{-19} \, \text{J/eV} = 1,632 \times 10^{-18} \, \text{J}}
		      \]
		      Agora, calculamos a frequência da radiação emitida usando a relação de Planck:
		      \[
			      \textcolor{red}{f = \frac{\Delta E}{h} = \frac{1,632 \times 10^{-18} \, \text{J}}{6,63 \times 10^{-34} \, \text{Js}} = 2,46 \times 10^{15} \, \text{Hz}}
		      \]
		      Portanto, a frequência da radiação emitida é \textcolor{red}{\(2,46 \times 10^{15} \, \text{Hz}\)}.
	      }
	      
	\item Calcule o comprimento de onda de De Broglie, em nanômetros, associado a uma bola de futebol, com massa de 400 g que se desloca a uma velocidade de 10 m/s.
	      
	      \textcolor{blue}{
		      Para calcular o comprimento de onda de De Broglie, utilizamos a fórmula:
		      \[
			      \textcolor{red}{\lambda = \frac{h}{mv}}
		      \]
		      Onde:
		      \begin{itemize}
			      \item[] \textcolor{red}{\(h = 6,626 \times 10^{-34} \, \text{J} \cdot \text{s}\)} é a constante de Planck;
			      \item[] \textcolor{red}{\(m = 400 \, \text{g} = 0,400 \, \text{kg}\)} é a massa da bola de futebol;
			      \item[] \textcolor{red}{\(v = 10 \, \text{m/s}\)} é a velocidade da bola.
		      \end{itemize}
		      Substituindo os valores:
		      \[
			      \textcolor{red}{\lambda = \frac{6,626 \times 10^{-34}}{0,400 \times 10} = \frac{6,626 \times 10^{-34}}{4}}
		      \]
		      \[
			      \textcolor{red}{\lambda = 1,6565 \times 10^{-34} \, \text{m}}
		      \]
		      Convertendo para nanômetros \textcolor{red}{{(\(1 \, \text{nm} = 10^{-9} \, \text{m}\))}}:
		      \[
			      \textcolor{red}{\lambda = 1,6565 \times 10^{-34} \, \text{m} \times 10^9 \, \text{nm/m} = 1,6565 \times 10^{-25} \, \text{nm}}
		      \]
		      Portanto, o comprimento de onda de De Broglie associado à bola de futebol é \textcolor{red}{\(1,66 \times 10^{-25} \, \text{nm}\)}.
	      }
	      \pagebreak
	      
	\item Um elétron se move a velocidade igual a \(4,7 \times 10^{6} \, \text{m/s}\). Determine o comprimento de onda de De Broglie, em nanômetro, para esse elétron. Obs.: apresentar todas as unidades durante as operações.
	      
	      \textcolor{blue}{
		      Para determinar o comprimento de onda de De Broglie, utilizamos a fórmula:
		      \[
			      \textcolor{red}{\lambda = \frac{h}{mv}}
		      \]
		      Onde:
		      \begin{itemize}
			      \item[] \textcolor{red}{\(h = 6,626 \times 10^{-34} \, \text{J} \cdot \text{s}\)} é a constante de Planck;
			      \item[] \textcolor{red}{\(m = 9,11 \times 10^{-31} \, \text{kg}\)} é a massa do elétron;
			      \item[] \textcolor{red}{\(v = 4,7 \times 10^{6} \, \text{m/s}\)} é a velocidade do elétron.
		      \end{itemize}
		      Substituindo os valores na fórmula:
		      \[
			      \textcolor{red}{\lambda = \frac{6,626 \times 10^{-34} \, \text{J} \cdot \text{s}}{9,11 \times 10^{-31} \, \text{kg} \times 4,7 \times 10^{6} \, \text{m/s}}}
		      \]
		      Calculando o denominador:
		      \[
			      \textcolor{red}{9,11 \times 10^{-31} \, \text{kg} \times 4,7 \times 10^{6} \, \text{m/s} = 4,2817 \times 10^{-24} \, \text{kg} \cdot \text{m/s}}
		      \]
		      Assim:
		      \[
			      \textcolor{red}{\lambda = \frac{6,626 \times 10^{-34} \, \text{J} \cdot \text{s}}{4,2817 \times 10^{-24} \, \text{kg} \cdot \text{m/s}} = 1,548 \times 10^{-10} \, \text{m}}
		      \]
		      Convertendo para nanômetros \textcolor{red}{{(\(1 \, \text{nm} = 10^{-9} \, \text{m}\))}}:
		      \[
			      \textcolor{red}{\lambda = 1,548 \times 10^{-10} \, \text{m} \times 10^{9} \, \text{nm} = 1,548 \times 10^{-1} \text{nm}}
		      \]
		      Portanto, o comprimento de onda de De Broglie para esse elétron é \textcolor{red}{\(0,1548 \, \text{nm}\)} ou \textcolor{red}{\(1,548 \times 10^{-1} \text{nm}\)}. 
	      }
	      
\end{enumerate}

\subsection*{Números Quânticos}

\begin{enumerate}
	\setcounter{enumi}{12}
	\item Um orbital tem números quânticos de \(n = 4\), \(l = 2\), \(m_l = -1\). Que tipo de orbital é esse?
	      
	\item Qual é o número máximo de elétrons em um átomo que podem ter os seguintes números quânticos:
	      \begin{enumerate}
		      \item[a)] \(n = 4\), \(l = 3\), \(m_l = -3\), \(m_s = -1/2\).
		      \item[b)] \(n = 2\), \(l = 1\), \(m_l = 0\), \(m_s = -1/2\).
	      \end{enumerate}
	      
	\item Escreva um conjunto completo de números quânticos (\(n\), \(l\), \(m_l\), \(m_s\)) permitidos pela teoria quântica para cada um dos seguintes orbitais:
	      \begin{enumerate}
		      \item[a)] \(2p^2\)
		      \item[b)] \(4f^{12}\)
	      \end{enumerate}
	      
	\item Qual o número máximo de elétrons no orbital que podem ser identificados a cada um dos seguintes conjuntos de número quânticos:
	      \begin{enumerate}
		      \item[a)] \(n = 6\), \(l = 1\), \(m_l = -1\)
		      \item[b)] \(n = 3\), \(l = 3\), \(m_l = -3\)
	      \end{enumerate}
	      
	\item Um elétron num certo átomo está no nível quântico \(n = 2\). Indique os valores possíveis de \(l\) e \(m_l\).
	      
	\item Indique qual (ais) dos seguintes conjuntos dos números quânticos para um átomo é (são) inaceitável (eis) e explique por quê.
	      \begin{enumerate}
		      \item[a)] \((3, 0, 0, +1/2)\)
		      \item[b)] \((2, 2, 1, +1/2)\)
		      \item[c)] \((4, 3, -2, +1/2)\)
	      \end{enumerate}
	      
	\item Qual é a designação (notação) para o subnível \(n = 5\) e \(l = 1\)? Quantos orbitais existem nesse subnível e indique os valores de \(m_l\) para cada um desses orbitais.
	      
	\item Em relação aos números quânticos responda:
	      \begin{enumerate}
		      \item[a)] A notação da subcamada e o número de orbitais para: \(\{n = 6, l = 1\}\), \(\{n = 5, l = 3\}\), \(\{n = 4, l = 2\}\).
		      \item[b)] A ordem crescente de energia dos conjuntos de número quânticos de (a).
	      \end{enumerate}
	      
	\item Com relação aos números quânticos responda:
	      \begin{enumerate}
		      \item[a)] O número de elétrons no orbital \(p\) do terceiro nível do elemento de número atômico 16.
		      \item[b)] O conjunto dos quatro números quânticos para o último elétron de um átomo neutro, cuja configuração é: \(1s^2 2s^2 2p^6 3s^2 3p^6 4s^2\).
	      \end{enumerate}
	      
	\item Quais são os quatro números quânticos do último elétron representado, seguindo a regra de Hund, ao efetuar a representação gráfica de 9 elétrons no subnível \(4f\)?
	      
	\item Qual o conjunto dos quatro números quânticos para o elétron diferenciador do íon \(^{20}\text{Ca}^{+2}\)?
\end{enumerate}

\end{document}