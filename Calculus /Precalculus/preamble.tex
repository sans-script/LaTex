\usepackage[utf8]{inputenc}            % Codificação do documento
\usepackage[T1]{fontenc}               % Encodings para caracteres especiais
\usepackage[brazil]{babel}             % Idioma do documento
\usepackage{geometry}                  % Para ajustar margens
\usepackage{graphicx}                  % Para incluir imagens
\usepackage{amsmath, amsfonts, amsthm, amssymb} % Para símbolos matemáticos
\usepackage{fancyhdr}                  % Para cabeçalhos e rodapés
\usepackage[dvipsnames]{xcolor}        % Para cores
\usepackage{thmtools}                  % Para definições de teoremas
\usepackage{titlesec}                  % Para personalizar títulos de seções
\usepackage{booktabs}                  % Adicione isso no preâmbulo
\usepackage{tabularx}                  % Para tabelas com largura ajustável
\usepackage{mathptmx}                  % Para usar Times New Roman
\usepackage{enumitem} % Para personalizar a lista


% Configuração das margens
\geometry{top=2cm, bottom=2cm, left=2cm, right=2cm}

\titleformat{\section}
{\normalfont\Large\bfseries}{\thesection}{0.6em}{}

\titleformat{\subsection}
{\normalfont\large\bfseries}{\thesubsection}{0.3em}{}

\setcounter{secnumdepth}{0}  % Remove a numeração de seções


% Configura cabeçalhos e rodapés
\pagestyle{fancy}
\fancyhf{} % Limpa cabeçalhos e rodapés
\fancyhead[R]{Pré-Cálculo} % Nome à direita no cabeçalho
\fancyhead[L]{Alexandre Santos}
\fancyfoot[R]{\thepage} % Número da página à direita no rodapé