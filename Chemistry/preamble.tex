\usepackage[utf8]{inputenc}            % Codificação do documento
\usepackage[T1]{fontenc}               % Encodings para caracteres especiais
\usepackage[brazil]{babel}             % Idioma do documento
\usepackage{geometry}                  % Para ajustar margens
\usepackage{graphicx}                  % Para incluir imagens
\usepackage{amsmath, amsfonts, amsthm, amssymb} % Para símbolos matemáticos
\usepackage{fancyhdr}                  % Para cabeçalhos e rodapés
\usepackage[dvipsnames]{xcolor}        % Para cores
\usepackage{thmtools}                  % Para definições de teoremas
\usepackage{titlesec}                  % Para personalizar títulos de seções
\usepackage{booktabs}                  % Para tabelas
\usepackage{tabularx}                  % Para tabelas com largura ajustável
\usepackage{mathptmx}                  % Para usar Times New Roman
\usepackage{enumitem}                  % Para personalizar a lista
\usepackage{cancel}                    % Para cancelar termos em equações
\usepackage[version=4]{mhchem}        % Para escrever fórmulas químicas
\usepackage{chemfig}                   % Para desenhar moléculas
\usepackage{colortbl}                  % Para colorir células de tabelas
\usepackage{multicol}                  % Para dividir o texto em colunas
\usepackage{tikz}                      % Para desenhar gráficos
\usepackage{pgf-spectra}               % Para espectros
\usepackage{siunitx}                   % Para formatação de unidades
\usepackage{array}                     % Para manipulação de arrays
\usepackage{cellspace}                 % Para espaçamento em células
\renewcommand{\arraystretch}{1.5}      % Ajusta a altura das células
\usepackage{modiagram}                 % Para diagramas

% Configuração das margens
\geometry{top=2cm, bottom=2cm, left=1cm, right=1cm}

\titleformat{\section}
{\normalfont\Large\bfseries}{\thesection}{0.6em}{}

\titleformat{\subsection}
{\normalfont\large\bfseries}{\thesubsection}{0.3em}{}

\setcounter{secnumdepth}{0}  % Remove a numeração de seções
\def\checkmark{\tikz\fill[scale=0.4](0,.35) -- (.25,0) -- (1,.7) -- (.25,.15) -- cycle;}
\definecolor{indigo}{rgb}{0.29, 0.0, 0.51}

% Configura cabeçalhos e rodapés
\pagestyle{fancy}
\fancyhf{} % Limpa cabeçalhos e rodapés
\fancyhead[L]{Alexandre Santos}
\fancyfoot[R]{\thepage} % Número da página à direita no rodapé