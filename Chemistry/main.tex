\documentclass[a4paper, 12pt]{article}
\usepackage[utf8]{inputenc}            % Codificação do documento
\usepackage[T1]{fontenc}               % Encodings para caracteres especiais
\usepackage[brazil]{babel}             % Idioma do documento
\usepackage{geometry}                  % Para ajustar margens
\usepackage{graphicx}                  % Para incluir imagens
\usepackage{amsmath, amsfonts, amsthm, amssymb} % Para símbolos matemáticos
\usepackage{fancyhdr}                  % Para cabeçalhos e rodapés
\usepackage[dvipsnames]{xcolor}        % Para cores
\usepackage{thmtools}                  % Para definições de teoremas
\usepackage{titlesec}                  % Para personalizar títulos de seções
\usepackage{booktabs}                  % Adicione isso no preâmbulo
\usepackage{tabularx}                  % Para tabelas com largura ajustável
\usepackage{mathptmx}                  % Para usar Times New Roman
\usepackage{enumitem} % Para personalizar a lista
\usepackage{cancel} % Para cancelar termos em equações
\usepackage[version=4]{mhchem} % Para escrever fórmulas químicas
\usepackage{chemfig} % Para desenhar moléculas
\usepackage{colortbl} % Para colorir células de tabelas
\usepackage{multicol} % Para dividir o texto em colunas
\usepackage{tikz} % Para desenhar gráficos
\usepackage{tikz}
\usepackage{pgf-spectra}
\usepackage{siunitx}


% Configuração das margens
\geometry{top=2cm, bottom=2cm, left=1cm, right=1cm}

\titleformat{\section}
{\normalfont\Large\bfseries}{\thesection}{0.6em}{}

\titleformat{\subsection}
{\normalfont\large\bfseries}{\thesubsection}{0.3em}{}

\setcounter{secnumdepth}{0}  % Remove a numeração de seções
\def\checkmark{\tikz\fill[scale=0.4](0,.35) -- (.25,0) -- (1,.7) -- (.25,.15) -- cycle;}
\definecolor{indigo}{rgb}{0.29, 0.0, 0.51}

% Configura cabeçalhos e rodapés
\pagestyle{fancy}
\fancyhf{} % Limpa cabeçalhos e rodapés
\fancyhead[R]{Exercícios de Fixação} % Nome à direita no cabeçalho
\fancyhead[L]{Alexandre Santos}
\fancyfoot[R]{\thepage} % Número da página à direita no rodapé

\begin{document}
\section*{Exercícios}
\begin{enumerate}
    % ======= Questão 1 ========= %

    \item Apresenta-se duas reações químicas representadas pelas equações a seguir:
          \begin{itemize}
              \item[] Equação I:
                    $\text{AgNO}_{3} (aq) + \text{KCl}(aq) \rightarrow \text{AgCl}(s) + \text{KNO}
                        _{3} (aq)$
              \item[] Equação II:
                    $\text{HCl}(aq) + \text{Ca(OH)}_{2}(aq) \rightarrow \text{CaCl}_{2}(aq)
                        + \text{H}_{2}\text{O}(l)$
          \end{itemize}
          \begin{enumerate}[align=left, labelsep=-0.5em]
              \item[a)] Apresente as duas equações balanceadas.
                    %RESPOSTA%
                    \\[10pt]
                    \textcolor{blue}{\textbf{Equação I:}}
                    \[
                        \textcolor{red}{\text{AgNO}_3 (aq) + \text{KCl} (aq) \rightarrow \text{AgCl} (s) + \text{KNO}_3 (aq)}
                    \]
                    \textcolor{blue}{Nessa reação, todos os coeficientes já estão balanceados. Portanto, a equação já está correta.}
                    \\[10pt]
                    \textcolor{blue}{\textbf{Equação II:}}
                    \\[10pt]
                    \textcolor{blue}{Primeiro, escrevemos a equação molecular inicial:}
                    \[
                        \textcolor{red}{\text{HCl} (aq) + \text{Ca(OH)}_2 (aq) \rightarrow \text{CaCl}_2 (aq) + \text{H}_2\text{O} (l)}
                    \]
                    \textcolor{blue}{Agora, verificamos o balanceamento dos átomos. No lado dos reagentes, temos 1 átomo de cloro \(\textcolor{red}{(\text{Cl})}\) no \(\textcolor{red}{\text{HCl}}\) e 2 átomos de cloro \(\textcolor{red}{(\text{Cl})}\) no \(\textcolor{red}{\text{CaCl}_2}\). Para balancear o cloro, colocamos o coeficiente 2 na frente de \(\textcolor{red}{\text{HCl}}\):}
                    \[
                        \textcolor{red}{2 \, \text{HCl} (aq) + \text{Ca(OH)}_2 (aq) \rightarrow \text{CaCl}_2 (aq) + \text{H}_2\text{O} (l)}
                    \]
                    \textcolor{blue}{Agora, verificamos o balanceamento do hidrogênio. No lado dos reagentes, temos 2 átomos de hidrogênio \(\textcolor{red}{(\text{H})}\) no \(\textcolor{red}{2 \, \text{HCl}}\) e 2 átomos de hidrogênio \(\textcolor{red}{(\text{H})}\) no \(\textcolor{red}{\text{Ca(OH)}_2}\). No lado dos produtos, temos apenas 2 átomos de hidrogênio \(\textcolor{red}{(\text{H})}\) no \(\textcolor{red}{\text{H}_2\text{O}}\). Para balancear, colocamos o coeficiente 2 na frente de \(\textcolor{red}{\text{H}_2\text{O}}\):}
                    \[
                        \textcolor{red}{2 \, \text{HCl} (aq) + \text{Ca(OH)}_2 (aq) \rightarrow \text{CaCl}_2 (aq) + 2 \, \text{H}_2\text{O} (l)}
                    \]
                    \textcolor{blue}{Por fim, verificamos o balanceamento dos átomos de oxigênio. No lado dos reagentes, temos 2 átomos de oxigênio \(\textcolor{red}{(\text{O})}\) no \(\textcolor{red}{\text{Ca(OH)}_2}\), e no lado dos produtos, temos 2 átomos de oxigênio \(\textcolor{red}{(\text{O})}\) no \(\textcolor{red}{2 \, \text{H}_2\text{O}}\). Portanto, a equação está balanceada.}
                    \\[10pt]
                    \textcolor{blue}{\textbf{Equação balanceada final:}}
                    \[
                        \textcolor{red}{2 \, \text{HCl} (aq) + \text{Ca(OH)}_2 (aq) \rightarrow \text{CaCl}_2 (aq) + 2 \, \text{H}_2\text{O} (l)}
                    \]
                    %RESPOSTA%
              \item[b)] Escreva sobre as características das duas reações químicas.
                    %RESPOSTA%
                    \\[10pt] \textcolor{blue}{\textbf{Equação I:} Trata-se de uma reação de
                        dupla troca ou precipitação, onde o nitrato de prata \textcolor{red}{($\text{AgNO}
                                _{3}$)} reage com o cloreto de potássio \textcolor{red}{($\text{KCl}$)}
                        formando o cloreto de prata \textcolor{red}{($\text{AgCl}$)}, que é um
                        precipitado, e o nitrato de potássio \textcolor{red}{($\text{KNO}_{3}$)},
                        que permanece em solução aquosa.} \\[10pt] \textcolor{blue}{\textbf{Equação
                            II:} É uma reação ácido-base (neutralização), onde o ácido clorídrico
                        \textcolor{red}{($\text{HCl}$)} reage com o hidróxido de cálcio \textcolor{red}{($\text{Ca(OH)}
                                _{2}$)} formando o cloreto de cálcio \textcolor{red}{($\text{CaCl}_{2}$)}
                        e água \textcolor{red}{($\text{H}_{2}\text{O}$)}, liberando calor durante
                        o processo.}
                    %RESPOSTA%
              \item[c)] Escreva a equação iônica completa para ambas as reações.
                    \\[10pt] \textcolor{blue}{\textbf{Equação iônica completa - Reação I:}}
                    \[
                        \textcolor{red}{\text{Ag}^+ (aq) + \text{NO}_3^- (aq) + \text{K}^+ (aq)
                            + \text{Cl}^- (aq) \rightarrow \text{AgCl} (s) + \text{K}^+ (aq) +
                            \text{NO}_3^- (aq)}
                    \]
                    \\[10pt] \textcolor{blue}{\textbf{Equação iônica completa - Reação II:}}
                    \[
                        \textcolor{red}{2 \, \text{H}^+ (aq) + 2 \, \text{Cl}^- (aq) +
                            \text{Ca}^{2+} (aq) + 2 \, \text{OH}^- (aq) \rightarrow \text{Ca}^{2+}
                            (aq) + 2 \, \text{Cl}^- (aq) + 2 \, \text{H}_2\text{O} (l)}
                    \]
                    %RESPOSTA%
              \item[d)] Escreva a equação iônica líquida para as duas reações.
                    %RESPOSTA%
                    \\[10pt]
                    \textcolor{blue}{\textbf{Equação iônica líquida - Reação I:}}
                    \[
                        \textcolor{red}{\text{Ag}^+ (aq) + \cancel{\text{NO}_3^- (aq)} + \cancel{\text{K}^+ (aq)}
                            + \text{Cl}^- (aq) \rightarrow \text{AgCl} (s) + \cancel{\text{K}^+ (aq)} +
                            \cancel{\text{NO}_3^- (aq)}}
                    \]
                    \[
                        \textcolor{red}{\text{Ag}^+ (aq) + \text{Cl}^- (aq) \rightarrow
                            \text{AgCl} (s)}
                    \]
                    \\[10pt]
                    \textcolor{blue}{\textbf{Equação iônica líquida - Reação II:}}
                    \[
                        \textcolor{red}{2 \, \text{H}^+ (aq) + 2 \, \cancel{\text{Cl}^- (aq)} + \cancel{\text{Ca}^{2+} (aq)} + 2 \, \text{OH}^- (aq) \rightarrow \cancel{\text{Ca}^{2+}} (aq) + \cancel{2 \, \text{Cl}^- (aq)} + 2 \, \text{H}_2\text{O} (l)}
                    \]
                    \[
                        \textcolor{red}{2 \, \text{H}^+ (aq) + 2 \, \text{OH}^- (aq)
                            \rightarrow 2 \, \text{H}_2\text{O} (l)}
                    \]
                    %RESPOSTA%
          \end{enumerate}

          % ======= Questão 2 ========= %

    \item Escreva uma equação iônica líquida, balanceada para as seguintes reações:
          \begin{enumerate}[align=left, labelsep=-0.5em]
              \item[a)] Entre soluções aquosas de cloreto de bário e sulfato de sódio para formar sulfato de bário e cloreto de sódio.
                    %RESPOSTA%
                    \\[10pt]
                    \textcolor{blue}{Primeiro, escrevemos a equação molecular:}
                    \[
                        \textcolor{red}{\text{BaCl}_2 (aq) + \text{Na}_2\text{SO}_4 (aq) \rightarrow \text{BaSO}_4 (s) + \text{NaCl} (aq)}
                    \]
                    \textcolor{blue}{Agora, balanceamos a equação. No lado dos reagentes, temos 1 átomo de bário, 2 átomos de cloro, 2 átomos de sódio, 1 átomo de enxofre e 4 átomos de oxigênio. No lado dos produtos, temos 1 átomo de bário, 1 átomo de enxofre, 4 átomos de oxigênio, 1 átomo de sódio e 1 átomo de cloro. Para balancear, colocamos o número 2 na frente do \(\textcolor{red}{\text{NaCl}}\):}
                    \[
                        \textcolor{red}{\text{BaCl}_2 (aq) + \text{Na}_2\text{SO}_4 (aq) \rightarrow \text{BaSO}_4 (s) + 2 \, \text{NaCl} (aq)}
                    \]
                    \textcolor{blue}{Agora, dissociamos os compostos que se ionizam em solução aquosa:}
                    \[
                        \textcolor{red}{\text{Ba}^{2+} (aq) + 2 \, \text{Cl}^- (aq) + 2 \, \text{Na}^+ (aq) + \text{SO}_4^{2-} (aq) \rightarrow \text{BaSO}_4 (s) + 2 \, \text{Na}^+ (aq) + 2 \, \text{Cl}^- (aq)}
                    \]
                    \textcolor{blue}{Eliminamos os íons espectadores, que são os íons \(\textcolor{red}{\text{Na}^+}\) e \(\textcolor{red}{\text{Cl}^-}\). A equação iônica líquida é:}
                    \[
                        \textcolor{red}{\text{Ba}^{2+} (aq) + \cancel{2 \, \text{Cl}^- (aq)} + \cancel{2 \, \text{Na}^+ (aq)} + \text{SO}_4^{2-} (aq) \rightarrow \text{BaSO}_4 (s) + \cancel{2 \, \text{Na}^+ (aq)} + \cancel{2 \, \text{Cl}^- (aq)}}
                    \]
                    \[
                        \textcolor{red}{\text{Ba}^{2+} (aq) + \text{SO}_4^{2-} (aq) \rightarrow \text{BaSO}_4 (s)}
                    \]
                    %RESPOSTA%

              \item[b)] Entre as soluções de nitrato de chumbo (II) e de cloreto de potássio.
                    %RESPOSTA%
                    \\[10pt]
                    \textcolor{blue}{Primeiro, escrevemos a equação molecular:}
                    \[
                        \textcolor{red}{\text{Pb(NO}_3)_2 (aq) + \text{KCl} (aq) \rightarrow \text{PbCl}_2 (s) + \text{KNO}_3 (aq)}
                    \]
                    \textcolor{blue}{Agora, balanceamos a equação. No lado dos reagentes, temos 1 átomo de chumbo, 2 átomos de nitrogênio, 6 átomos de oxigênio, 1 átomo de potássio e 1 átomo de cloro. No lado dos produtos, temos 1 átomo de chumbo, 2 átomos de cloro, 1 átomo de potássio, 1 átomo de nitrogênio e 3 átomos de oxigênio. Para balancear, colocamos o número 2 na frente de \(\textcolor{red}{\text{KCl}}\) e \(\textcolor{red}{\text{KNO}_3}\):}
                    \[
                        \textcolor{red}{\text{Pb(NO}_3)_2 (aq) + 2 \, \text{KCl} (aq) \rightarrow \text{PbCl}_2 (s) + 2 \, \text{KNO}_3 (aq)}
                    \]
                    \textcolor{blue}{Agora, dissociamos os compostos que se ionizam em solução aquosa:}
                    \[
                        \textcolor{red}{\text{Pb}^{2+} (aq) + 2 \, \text{NO}_3^- (aq) + 2 \, \text{K}^+ (aq) + 2 \, \text{Cl}^- (aq) \rightarrow \text{PbCl}_2 (s) + 2 \, \text{K}^+ (aq) + 2 \, \text{NO}_3^- (aq)}
                    \]
                    \textcolor{blue}{Eliminamos os íons espectadores, que são os íons \(\textcolor{red}{2 \,\text{NO}_3^-}\) e \(\textcolor{red}{2 \,\text{K}^+}\). A equação iônica líquida é:}
                    \[
                        \textcolor{red}{\text{Pb}^{2+} (aq) + \cancel{2 \, \text{NO}_3^- (aq)} + \cancel{2 \, \text{K}^+ (aq)} + 2 \, \text{Cl}^- (aq) \rightarrow \text{PbCl}_2 (s) + \cancel{2 \, \text{K}^+ (aq)} + \cancel{2 \, \text{NO}_3^- (aq)}}
                    \]
                    \[
                        \textcolor{red}{\text{Pb}^{2+} (aq) + 2 \, \text{Cl}^- (aq) \rightarrow \text{PbCl}_2 (s)}
                    \]
                    %RESPOSTA%
          \end{enumerate}
          % ======= Questão 3 ========= %

    \item Escreva uma equação balanceada para a reação que ocorre quando o carbonato
          de níquel (II) é tratado com o ácido sulfúrico.
          %RESPOSTA%
          \\[10pt]
          \textcolor{blue}{Primeiro, escrevemos a equação molecular:}
          \[
              \textcolor{red}{\text{NiCO}_3 (s) + \text{H}_2\text{SO}_4 (aq) \rightarrow \text{NiSO}_4 (aq) + \text{H}_2\text{O} (l) + \text{CO}_2 (g)}
          \]
          \textcolor{blue}{Agora, vamos balancear a equação. Nos reagentes, temos 1 átomo de níquel, 1 átomo de enxofre e 7 átomos de oxigênio (3 do carbonato e 4 do ácido sulfúrico), além de 2 átomos de hidrogênio. No lado dos produtos, temos 1 átomo de níquel, 1 átomo de enxofre, 7 átomos de oxigênio (4 do sulfato, 1 da água e 2 do dióxido de carbono) e 2 átomos de hidrogênio. Como tudo está balanceado, a equação já está balanceada.}

          \[
              \textcolor{red}{\text{NiCO}_3 (s) + \text{H}_2\text{SO}_4 (aq) \rightarrow \text{NiSO}_4 (aq) + \text{H}_2\text{O} (l) + \text{CO}_2 (g)}
          \]
          %RESPOSTA%


          % ======= Questão 4 ========= %
    \item A amônia, $\text{NH}_{3}$, é um dos produtos químicos mais importantes
          em economias industriais. É usada não apenas diretamente como fertilizante,
          mas também é a matéria prima para a produção de ácido nítrico. Como base, reage
          com os ácidos, como o ácido clorídrico. Pede-se:
          \begin{enumerate}[align=left, labelsep=-0.5em]
              \item[a)] Equação líquida, balanceada, para essa reação.
                    %RESPOSTA%
                    \\[10pt]
                    \textcolor{blue}{Primeiro, escrevemos a equação molecular:}
                    \[
                        \textcolor{red}{\text{NH}_3 (aq) + \text{HCl} (aq) \rightarrow \text{NH}_4\text{Cl} (aq)}
                    \]
                    \textcolor{blue}{A equação já está balanceada, pois o número de átomos de cada elemento é o mesmo dos dois lados da equação. Temos 1 átomo de Nitrogênio \textcolor{red}{(\(\text{N}\))}, 3 átomos de Hidrogênio \textcolor{red}{(\(\text{H}\))} do lado esquerdo e 1 átomo de Nitrogênio \textcolor{red}{(\(\text{N}\))} e 4 átomos de Hidrogênio \textcolor{red}{(\(\text{H}\))} no lado direito. O Cloro \textcolor{red}{(\(\text{Cl}\))} também está equilibrado com 1 átomo dos dois lados.}
                    \textcolor{blue}{Agora, dissociamos os compostos que se ionizam em solução aquosa:}
                    \[
                        \textcolor{red}{\text{NH}_3 (aq) + \text{H}^+ (aq) + \text{Cl}^- (aq) \rightarrow \text{NH}_4^+ (aq) + \text{Cl}^- (aq)}
                    \]
                    \textcolor{blue}{Eliminamos os íons espectadores, que são os íons \(\textcolor{red}{\text{Cl}^-}\). A equação iônica líquida é:}
                    \[
                        \textcolor{red}{\text{NH}_3 (aq) + \text{H}^+ (aq) \rightarrow \text{NH}_4^+ (aq)}
                    \]
                    %RESPOSTA%
              \item[b)] Fundamentar o motivo pelo qual $\text{NH}_{3}$ é classificado como
                    uma base.
                    %RESPOSTA%
                    \\[10pt]
                    \textcolor{blue}{A amônia \textcolor{red}{(\(\text{NH}_3\))} é classificada como uma base de Brønsted-Lowry porque ela tem a capacidade de aceitar um íon hidrogênio \textcolor{red}{(\(\text{H}^+\))} de um ácido. Em solução aquosa, o \textcolor{red}{(\(\text{NH}_3\))} reage com a água, formando o íon amônio \textcolor{red}{(\(\text{NH}_4^+\))} e liberando \textcolor{red}{(\(\text{OH}^-\))} (íon hidróxido), característica típica das bases. Além disso, \textcolor{red}{(\(\text{NH}_3\))} também é uma base de Lewis, pois possui um par de elétrons não compartilhados no átomo de nitrogênio, permitindo-lhe doar esse par para formar uma ligação com o íon hidrogênio \textcolor{red}{(\(\text{H}^+\))}.}
                    %RESPOSTA%

          \end{enumerate}



          % ======= Questão 5 ========= %

    \item Problemas estomacais causados pelo excesso de ácido estomacal, dentre
          eles o ácido clorídrico, podem ser tratados com antiácidos, que são bases
          simples como o hidróxido de magnésio, numa reação de neutralização. Esboce
          a equação molecular representativa dessa reação de neutralização e a
          equação iônica líquida.
          %RESPOSTA%
          \\[10pt]
          \textcolor{blue}{Primeiro, escrevemos a equação molecular da reação de neutralização entre o hidróxido de magnésio \textcolor{red}{(\(\text{Mg(OH)}_2\))} e o ácido clorídrico \textcolor{red}{(\(\text{HCl}\))}:}
          \[
              \textcolor{red}{\text{Mg(OH)}_2 (aq) + 2\text{HCl} (aq) \rightarrow \text{MgCl}_2 (aq) + 2\text{H}_2\text{O} (l)}
          \]
          \textcolor{blue}{Agora, dissociamos os compostos que se ionizam em solução aquosa:}
          \[
              \textcolor{red}{\text{Mg(OH)}_2 (aq) + 2\text{H}^+ (aq) + 2\text{Cl}^- (aq) \rightarrow \text{Mg}^{2+} (aq) + 2\text{Cl}^- (aq) + 2\text{H}_2\text{O} (l)}
          \]
          \textcolor{blue}{Eliminamos os íons espectadores, que são os íons \(\textcolor{red}{\text{Cl}^-}\). A equação iônica líquida é:}
          \[
              \textcolor{red}{\text{Mg(OH)}_2 (aq) + 2\text{H}^+ (aq) \rightarrow \text{Mg}^{2+} (aq) + 2\text{H}_2\text{O} (l)}
          \]
          %RESPOSTA%


          % ======= Questão 6 ========= %

    \item Dadas as seguintes misturas de reagentes, estabeleça os produtos de
          cada uma das reações e classifique-as em: reações de neutralização, reação
          de oxidação-redução, reação com formação de gás e reação de precipitação.
          Justifique sua resposta.
          \begin{enumerate}[align=left, labelsep=-0.5em]
              \item[a)] $\text{Ba(OH)}_{2}(s) + 2\text{HNO}_{3}(aq) \rightarrow$

              \item[b)] $\text{Zn}(s) + \text{CuSO}_{4}(aq) \rightarrow$

              \item[c)] $\text{NaSO}_{3}(aq) + 2\text{HCl}(aq) \rightarrow$

              \item[d)] $\text{Pb(NO}_{3})_{2}(aq) + 2\text{KI}(aq) \rightarrow$
          \end{enumerate}
          %RESPOSTA%
    \item[\textcolor{blue}{a)}] \textcolor{blue}{Esta é uma \textbf{reação de neutralização}, pois envolve a reação entre uma base, o \textcolor{red}{\(\text{Ba(OH)}_{2}\)}, e um ácido, o \textcolor{red}{\(\text{HNO}_{3}\)}, formando um sal e água. A equação molecular é:}
          \[
              \text{Ba(OH)}_{2}(s) + 2\text{HNO}_{3}(aq) \rightarrow \textcolor{red}{\text{Ba(NO}_{3})_{2}(aq) + 2\textcolor{red}{\text{H}_2\text{O}}(l)}
          \]
          \textcolor{blue}{Neste caso, a reação é uma \textbf{reação de neutralização}, pois ocorre a troca de prótons (\(\textcolor{red}{\text{H}^+}\)) entre o ácido e a base.}

    \item[\textcolor{blue}{b)}] \textcolor{blue}{Esta é uma \textbf{reação de oxidação-redução}, pois o \textcolor{red}{\(\text{Zn}\)} sofre oxidação (perdendo elétrons) e o \textcolor{red}{\(\text{Cu}^{2+}\)} sofre redução (ganhando elétrons). A equação molecular é:}
          \[
              \text{Zn}(s) + \text{CuSO}_{4}(aq) \rightarrow \textcolor{red}{\text{ZnSO}_{4}(aq) + \text{Cu}(s)}
          \]
          \textcolor{blue}{A oxidação do \(\textcolor{red}{\text{Zn}}\) ocorre porque ele perde elétrons e se transforma em \(\textcolor{red}{\text{Zn}^{2+}}\), enquanto o \(\textcolor{red}{\text{Cu}^{2+}}\) é reduzido a \(\textcolor{red}{\text{Cu}(s)}\).}

    \item[\textcolor{blue}{c)}] \textcolor{blue}{Esta é uma \textbf{reação de formação de gás}, pois o \textcolor{red}{\(\text{NaSO}_3\)} reage com o \textcolor{red}{\(\text{HCl}\)}, liberando gás dióxido de enxofre \textcolor{red}{($\text{SO}_{2}$)}. A equação molecular é:}
          \[
              \text{NaSO}_{3}(aq) + 2\text{HCl}(aq) \rightarrow \textcolor{red}{\text{NaCl}(aq) + \text{SO}_2(g) + \text{H}_2\text{O}(l)}
          \]
          \textcolor{blue}{O gás dióxido de enxofre \textcolor{red}{($\text{SO}_{2}$)} é liberado durante a reação, caracterizando-a como uma \textbf{reação de formação de gás}.}

    \item[\textcolor{blue}{d)}] \textcolor{blue}{Esta é uma \textbf{reação de precipitação}, pois forma um sólido insolúvel, o \textcolor{red}{\(\text{PbI}_{2}\)}, a partir da reação de dois compostos solúveis. A equação molecular é:}
          \[
              \text{Pb(NO}_{3})_{2}(aq) + 2\text{KI}(aq) \rightarrow \textcolor{red}{\text{PbI}_{2}(s) + 2\text{KNO}_{3}(aq)}
          \]
          \textcolor{blue}{O \(\textcolor{red}{\text{PbI}_{2}}\) é um precipitado amarelo, formado como resultado da troca de íons entre os compostos solúveis, caracterizando a reação como uma \textbf{reação de precipitação}.}
          %RESPOSTA%
          % ======= Questão 7 ========= %

    \item Com base na estequiometria responda as seguintes questões:
          \begin{enumerate}[align=left, labelsep=-0.5em]
              \item[a)] Qual é a massa, em gramas, de \(2,50 \times 10^{-3}\) mol de sulfato de alumínio?
                    \\[10pt]
                    \textcolor{blue}{\textcolor{red}{\text{Al}} = 26,98 \, \text{g/mol}, \quad \textcolor{red}{\text{S}} = 32,07 \, \text{g/mol}, \quad \textcolor{red}{\text{O}} = 16,00 \, \text{g/mol}}
                    \\[10pt]
                    \textcolor{blue}{A massa molar do \textcolor{red}{\((\text{Al}_2(\text{SO}_4)_3\))} é:}
                    \[
                        \textcolor{blue}{\text{$MM$} = 2 \times 26,98 + 3 \times (32,07 + 4 \times 16,00) = 342,17 \, \text{g/mol}}
                    \]
                    \textcolor{blue}{Usando a fórmula:}
                    \[
                        \textcolor{blue}{\text{$M$} = 2,50 \times 10^{-3} \, \text{mol} \times 342,17 \, \text{g/mol} = 0,855 \, \text{g}}
                    \]
              \item[b)] Qual é a massa, em gramas, de \(7,70 \times 10^{20}\) moléculas de cafeína, \(\text{C}_{8}\text{H}_{10}\text{N}_{4}\text{O}_{2}\)?
                    \\[10pt]
                    \textcolor{blue}{\textcolor{red}{\text{C}} = 12,01 \, \text{g/mol}, \quad \textcolor{red}{\text{H}} = 1,01 \, \text{g/mol}, \quad \textcolor{red}{\text{N}} = 14,01 \, \text{g/mol}, \quad \textcolor{red}{\text{O}} = 16,00 \, \text{g/mol}}
                    \\[10pt]
                    \textcolor{blue}{A massa molar da cafeína \textcolor{red}{(\(\text{C}_{8}\text{H}_{10}\text{N}_{4}\text{O}_{2}\))} é:}
                    \[
                        \textcolor{blue}{\text{$MM$} = 8 \times 12,01 + 10 \times 1,01 + 4 \times 14,01 + 2 \times 16,00 = 194,19 \, \text{g/mol}}
                    \]
                    \textcolor{blue}{Número de mols de moléculas:}
                    \[
                        \textcolor{blue}{\text{$n$} = \frac{7,70 \times 10^{20}}{6,022 \times 10^{23} \, \text{mol}^{-1}} = 1,28 \times 10^{-3} \, \text{mol}}
                    \]
                    \textcolor{blue}{Massa em gramas:}
                    \[
                        \textcolor{blue}{\text{$M$} = 1,28 \times 10^{-3} \, \text{mol} \times 194,19 \, \text{g/mol} = 0,248 \, \text{g}}
                    \]
              \item[c)] Qual é a massa molar do colesterol se \(0,00105\) mol pesa \(0,406\) g?
                    \\[10pt]
                    \textcolor{blue}{Cálculo da massa molar:}
                    \[
                        \textcolor{blue}{\text{\text{$MM$}} = \frac{\text{Massa (\text{$M$})}}{\text{Número de mols (\text{$n$})}} = \frac{0,406 \, \text{g}}{0,00105 \, \text{mol}} = 386,67 \, \text{g/mol}}
                    \]
              \item[d)] Qual a massa, em gramas, de \(6,33\) mol de \(\text{NaHCO}_{3}\)?
                    \\[10pt]
                    \textcolor{blue}{\textcolor{red}{\text{Na}} = 22,99 \, \text{g/mol}, \quad \textcolor{red}{\text{H}} = 1,01 \, \text{g/mol}, \quad \textcolor{red}{\text{C}} = 12,01 \, \text{g/mol}, \quad \textcolor{red}{\text{O}} = 16,00 \, \text{g/mol}}
                    \\[10pt]
                    \textcolor{blue}{A massa molar do \textcolor{red}{\(\text{NaHCO}_{3}\)} é:}
                    \[
                        \textcolor{blue}{\text{$MM$} = 22,99 + 1,01 + 12,01 + 3 \times 16,00 = 84,01 \, \text{g/mol}}
                    \]
                    \textcolor{blue}{Massa em gramas:}
                    \[
                        \textcolor{blue}{\text{$M$} = 6,33 \, \text{mol} \times 84,01 \, \text{g/mol} = 531,78 \, \text{g}}
                    \]

              \item[e)] O número de moléculas em \(0,245\) mol de \(\text{CH}_{3}\text{OH}\).
                    \\[10pt]
                    \textcolor{blue}{Número de moléculas:}
                    \[
                        \textcolor{blue}{\text{Número de moléculas} = 0,245 \, \text{mol} \times 6,022 \times 10^{23} \, \text{mol}^{-1} = 1,475 \times 10^{23} \, \text{ moléculas}}
                    \]
              \item[f)] Número de átomos de H em \(0,585\) mol de \(\text{C}_{4}\text{H}_{4}\).
                    \\[10pt]
                    \textcolor{blue}{Número de átomos de \textcolor{red}{H}:}
                    \[
                        \textcolor{blue}{\text{Número de átomos de \textcolor{red}{H}} = 0,585 \, \text{mol} \times 4 \times 6,022 \times 10^{23} \, \text{mol}^{-1} = 1,41 \times 10^{24} \, \text{ átomos de \textcolor{red}{H}}}
                    \]
              \item[g)] Número de íons \(\text{NO}_{3}^{-}\) em \(4,88 \times 10^{-3}\) mol de \(\text{Al(NO}_{3})_{3}\).
                    \\[10pt]
                    \textcolor{blue}{Número de íons \textcolor{red}{\(\text{NO}_{3}^{-}\)}:}
                    \[
                        \textcolor{blue}{\text{Número de íons } \textcolor{red}{\text{NO}_{3}^{-}} = 4,88 \times 10^{-3} \, \text{mol} \times 3 \times 6,022 \times 10^{23} \, \text{mol}^{-1} = 8,82 \times 10^{21} \, \text{ íons}}
                    \]
              \item[h)] Quantos átomos de oxigênio existem em \(0,25\) mol de \(\text{Ca(NO}_{3})_{2}\)?
                    \\[10pt]
                    \textcolor{blue}{Número de átomos de \textcolor{red}{O}:}
                    \[
                        \textcolor{blue}{\text{Número de átomos de \textcolor{red}{O}} = 0,25 \, \text{mol} \times 6 \times 6,022 \times 10^{23} \, \text{mol}^{-1} = 9,03 \times 10^{23} \, \text{ átomos de \textcolor{red}{O}}}
                    \]
              \item[i)] Quantas moléculas de ácido nítrico existem em \(4,20\) g de \(\text{HNO}_{3}\)?
                    \\[10pt]
                    \textcolor{blue}{\textcolor{red}{\text{H}} = 1,01 \, \text{g/mol}, \quad \textcolor{red}{\text{N}} = 14,01 \, \text{g/mol}, \quad \textcolor{red}{\text{O}} = 16,00 \, \text{g/mol}}
                    \\[10pt]
                    \textcolor{blue}{A massa molar do \textcolor{red}{\(\text{HNO}_{3}\)} é:}
                    \[
                        \textcolor{blue}{\text{$MM$} = 1,01 + 14,01 + 3 \times 16,00 = 63,02 \, \text{g/mol}}
                    \]
                    \textcolor{blue}{Número de mols:}
                    \[
                        \textcolor{blue}{n = \frac{4,20 \, \text{g}}{63,02 \, \text{g/mol}} = 0,0666 \, \text{mol}}
                    \]
                    \textcolor{blue}{Número de moléculas:}
                    \[
                        \textcolor{blue}{\text{Número de moléculas} = 0,0666 \, \text{mol} \times 6,022 \times 10^{23} \, \text{mol}^{-1} = 4,01 \times 10^{22} \, \text{ moléculas}}
                    \]
          \end{enumerate}

          % ======= Questão 8 ========= %

    \item Considere a combustão do butano, o combustível de isqueiros
          descartáveis:
          \[
              2\text{C}_{4}\text{H}_{10}(l) + 13\text{O}_{2}(g) \rightarrow 8\text{CO}_{2}
              (g) + 10\text{H}_{2}\text{O}(g)
          \]
          Calcule a massa de $\text{CO}_{2}$ produzida quando $1,00$ grama de $\text{C}
              _{4}\text{H}_{10}$ é queimado.
          %RESPOSTA%
          \\[10pt]
          $\textcolor{blue}{\text{Massa molar de \textcolor{red}{\(\text{C}_4\text{H}_{10}\)}} = 4 \times 12,01 + 10 \times 1,008 = 58,12 \, \text{g/mol}}$
          \[
              \textcolor{blue}{\text{Número de mols de \textcolor{red}{\(\text{C}_4\text{H}_{10}\)}} = \frac{1,00 \, \text{g}}{58,12 \, \text{g/mol}} = 0,0172 \, \text{mol}}
          \]
          $\textcolor{blue}{\text{A equação química nos mostra que } \, 2 \, \text{mol de \textcolor{red}{\(\text{C}_4\text{H}_{10}\)}} \, \text{ produzem } \, 8 \, \text{mol de \textcolor{red}{\(\text{CO}_2\)}}.}$
          \\[10pt]
          $\textcolor{blue}{\text{Portanto, os mols de \textcolor{red}{\(\text{CO}_2\)}} \, \text{produzidos são:}}$
          \[
              \textcolor{blue}{\text{Mols de \textcolor{red}{\(\text{CO}_2\)}} = 0,0172 \, \text{mol} \times \frac{8}{2} = 0,0688 \, \text{mol}}
          \]
          $\textcolor{blue}{\text{Agora, calculamos a massa de \textcolor{red}{\(\text{CO}_2\)}}:}$
          \[
              \textcolor{blue}{\text{Massa molar de \textcolor{red}{\(\text{CO}_2\)}} = 12,01 + 2 \times 16,00 = 44,01 \, \text{g/mol}}
          \]
          \[
              \textcolor{blue}{\text{Massa de \textcolor{red}{\(\text{CO}_2\)}} = 0,0688 \, \text{mol} \times 44,01 \, \text{g/mol} = 3,03 \, \text{g}}
          \]
          %RESPOSTA%



          % ======= Questão 9 ========= %

    \item A equação balanceada a seguir representa a reação de decomposição térmica
          do $\text{KClO}_{3}$.
          \[
              2\text{KClO}_{3}(s) \xrightarrow[\Delta]{\text{MnO}_2}2\text{KCl}(s) + 3\text{O}
              _{2}(g)
          \]
          Determine, em litros, o volume de $\text{O}_{2}$ produzido pela
          decomposição térmica de $245,2$ g de $\text{KClO}_{3}$, nas CNTP,
          expressando o resultado com dois algarismos significativos. Massas atômicas:
          $\text{K}= 39$ u; $\text{Cl}= 35,5$ u; $\text{O}= 16$ u.
          % RESPOSTA
          \\[10pt]
          $\textcolor{blue}{\text{Massa molar de \textcolor{red}{\(\text{KClO}_3\)}} = 39 + 35,5 + (3 \times 16) = 122,5 \, \text{g/mol}}$
          \[
              \textcolor{blue}{\text{Número de mols de \textcolor{red}{\(\text{KClO}_3\)}} = \frac{245,2 \, \text{g}}{122,5 \, \text{g/mol}} = 2,00 \, \text{mol}}
          \]
          $\textcolor{blue}{\text{A equação química nos mostra que } \, 2 \, \text{mol de \textcolor{red}{\(\text{KClO}_3\)}} \, \text{ produzem } \, 3 \, \text{mol de \textcolor{red}{\(\text{O}_2\)}}.}$
          \\ \\
          $\textcolor{blue}{\text{Portanto, os mols de \textcolor{red}{\(\text{O}_2\)}} \, \text{produzidos são:}}$
          \[
              \textcolor{blue}{\text{Mols de \textcolor{red}{\(\text{O}_2\)}} = 2,00 \, \text{mol} \times \frac{3}{2} = 3,00 \, \text{mol}}
          \]

          $\textcolor{blue}{\text{Agora, utilizamos o volume molar nas CNTP:}}$
          \[
              \textcolor{blue}{\text{Volume de \textcolor{red}{\(\text{O}_2\)}} = 3,00 \, \text{mol} \times 22,4 \, \text{L/mol} = 67 \, \text{L}}
          \]
          % RESPOSTA


          % ======= Questão 10 ========= %

    \item Descargas elétricas provocam a transformação do oxigênio ($\text{O}_{2}$)
          em ozônio ($\text{O}_{3}$). Quantos litros de oxigênio, medidos nas condições
          normais de pressão e temperatura, são necessários para a obtenção de
          $48,0$ g de ozônio? (Massa molar: $\text{O}= 16,0$ g/mol)
          % RESPOSTA
          \[
              \textcolor{blue}{\text{Massa molar de} \, \textcolor{red}{\text{O}_3} = 3 \times 16,0 \, \text{g/mol} = 48,0 \, \text{g/mol}}
          \]

          \[
              \textcolor{blue}{\text{Número de mols de} \, \textcolor{red}{\text{O}_3} = \frac{48,0 \, \text{g}}{48,0 \, \text{g/mol}} = 1,00 \, \text{mol}}
          \]

          \textcolor{blue}{A equação da reação química balanceada para a formação de ozônio é:}

          \[
              \textcolor{blue}{\textcolor{red}{3\text{O}_2(g) \rightarrow 2\text{O}_3(g)}}
          \]

          \textcolor{blue}{A equação balanceada nos mostra que, para 3 mols de $\textcolor{red}{\text{O}_2}$, formam-se 2 mols de $\textcolor{red}{\text{O}_3}$. Ou seja, a proporção entre os mols de $\textcolor{red}{\text{O}_2}$ e $\textcolor{red}{\text{O}_3}$ é de 3:2.}

          \textcolor{blue}{Agora, para 1 mol de $\textcolor{red}{\text{O}_3}$, são necessários:}

          \[
              \textcolor{blue}{\frac{3}{2} \, \text{mol de} \, \textcolor{red}{\text{O}_2} = 1,50 \, \text{mol de} \, \textcolor{red}{\text{O}_2}}
          \]

          \textcolor{blue}{Portanto, para obter 1 mol de $\textcolor{red}{\text{O}_3}$, são necessários 1,50 mol de $\textcolor{red}{\text{O}_2}$.}

          \textcolor{blue}{Agora, calculamos o volume de $\textcolor{red}{\text{O}_2}$ necessário. Nas condições normais de pressão e temperatura (CNTP), 1 mol de gás ocupa 22,4 L. Assim, o volume de $\textcolor{red}{\text{O}_2}$ necessário para a formação de 1 mol de $\textcolor{red}{\text{O}_3}$ é dado por:}

          \[
              \textcolor{blue}{V_{\textcolor{red}{\text{O}_2}} = 1,50 \, \text{mol} \times 22,4 \, \text{L/mol} = 33,6 \, \text{L}}
          \]
          % RESPOSTA

          % ======= Questão 11 ========= %

    \item Num processo de obtenção de ferro a partir da hematita ($\text{Fe}_{2}\text{O}_{3}$), considere a equação não balanceada:
          \[
              \text{Fe}_{2}\text{O}_{3} + \text{C} \rightarrow \text{Fe} + \text{CO}
          \]
          Utilizando-se $4,8$ toneladas de minério e admitindo-se um rendimento de $80\%$ na reação, a quantidade de ferro produzida será de: Pesos atômicos: $\text{C}= 12$; $\text{O}= 16$; $\text{Fe}= 56$.
          %RESPOSTA%
          \\[10pt]
          \textcolor{blue}{Primeiro, escrevemos a equação molecular inicial:}
          \[
              \textcolor{red}{\text{Fe}_2\text{O}_3 (s) + \text{C} (s) \rightarrow \text{Fe} (s) + \text{CO} (g)}
          \]
          \textcolor{blue}{Agora, verificamos o balanceamento dos átomos. No lado dos reagentes, temos 2 átomos de ferro \(\textcolor{red}{(\text{Fe})}\) no \(\textcolor{red}{\text{Fe}_2\text{O}_3}\) e 1 átomo de ferro \(\textcolor{red}{(\text{Fe})}\) no lado dos produtos. Para balancear o ferro, colocamos o coeficiente 2 na frente de \(\textcolor{red}{\text{Fe}}\):}
          \[
              \textcolor{red}{\text{Fe}_2\text{O}_3 (s) + \text{C} (s) \rightarrow 2 \, \text{Fe} (s) + \text{CO} (g)}
          \]
          \textcolor{blue}{Agora, verificamos o balanceamento do oxigênio. No lado dos reagentes, temos 3 átomos de oxigênio \(\textcolor{red}{(\text{O})}\) no \(\textcolor{red}{\text{Fe}_2\text{O}_3}\), e no lado dos produtos, temos 1 átomo de oxigênio \(\textcolor{red}{(\text{O})}\) no \(\textcolor{red}{\text{CO}}\). Para balancear o oxigênio, colocamos o coeficiente 3 na frente de \(\textcolor{red}{\text{CO}}\):}
          \[
              \textcolor{red}{\text{Fe}_2\text{O}_3 (s) + \text{C} (s) \rightarrow 2 \, \text{Fe} (s) + 3 \, \text{CO} (g)}
          \]
          \textcolor{blue}{Por fim, verificamos o balanceamento do carbono. No lado dos reagentes, temos 1 átomo de carbono \(\textcolor{red}{(\text{C})}\) no \(\textcolor{red}{\text{C}}\), e no lado dos produtos, temos 3 átomos de carbono \(\textcolor{red}{(\text{C})}\) no \(\textcolor{red}{3 \, \text{CO}}\). Para balancear o carbono, colocamos o coeficiente 3 na frente de \(\textcolor{red}{\text{C}}\):}
          \[
              \textcolor{red}{\text{Fe}_2\text{O}_3 (s) + 3 \, \text{C} (s) \rightarrow 2 \, \text{Fe} (s) + 3 \, \text{CO} (g)}
          \]
          \textcolor{blue}{Agora, a equação está balanceada.}
          \\[10pt]
          \textcolor{blue}{\text{Massa molar do \(\textcolor{red}{\text{Fe}_2\text{O}_3}\):}}
          \[
              \textcolor{blue}{\text{Massa molar de} \, \textcolor{red}{\text{Fe}_2\text{O}_3} = 2 \times 56 + 3 \times 16 = 160 \, \text{g/mol}}
          \]
          \\[10pt]
          \textcolor{blue}{\text{Conversão da massa de minério para gramas:}}
          \[
              \textcolor{blue}{4,8 \, \text{toneladas} = 4,8 \times 10^6 \, \text{g}}
          \]
          \\[10pt]
          \textcolor{blue}{\text{Número de mols de \(\textcolor{red}{\text{Fe}_2\text{O}_3}\):}}
          \[
              \textcolor{blue}{\text{Número de mols de} \, \textcolor{red}{\text{Fe}_2\text{O}_3} = \frac{4,8 \times 10^6 \, \text{g}}{160 \, \text{g/mol}} = 3,0 \times 10^4 \, \text{mol}}
          \]
          \\[10pt]
          \textcolor{blue}{\text{Número de mols de \(\textcolor{red}{\text{Fe}}\) produzidos:}}
          \[
              \textcolor{blue}{\text{Número de mols de} \, \textcolor{red}{\text{Fe}} = 3,0 \times 10^4 \, \text{mol} \times 2 = 6,0 \times 10^4 \, \text{mol}}
          \]
          \\[10pt]
          \textcolor{blue}{\text{Massa de \(\textcolor{red}{\text{Fe}}\) produzida:}}
          \[
              \textcolor{blue}{\text{Massa molar de } \, \textcolor{red}{\text{Fe}} = 56 \, \text{g/mol}}
          \]
          \[
              \textcolor{blue}{\text{Massa de }  \, \textcolor{red}{\text{Fe}} = 6,0 \times 10^4 \, \text{mol} \times 56 \, \text{g/mol} = 3,36 \times 10^6 \, \text{g}}
          \]
          \\[10pt]
          \textcolor{blue}{\text{Ajuste para o rendimento de 80\%:}}
          \[
              \textcolor{blue}{\text{Massa real de } \, \textcolor{red}{\text{Fe}} = 3,36 \times 10^6 \, \text{g} \times 0,80 = 2,688 \times 10^6 \, \text{g}}
          \]
          \[
              \textcolor{blue}{\text{Massa real de } \, \textcolor{red}{\text{Fe}} = 2,688 \, \text{t}}
          \]
          %RESPOSTA%



          % ======= Questão 12 ========= %

    \item O propano, $\text{C}_{3}\text{H}_{8}$, é um combustível comum para fogão
          e aquecimento residencial. Qual a massa de $\text{O}_{2}$ consumida na
          combustão de $1,00$ g de propano?
          %RESPOSTA%
          \\[10pt]
          \textcolor{blue}{Primeiro, escrevemos a equação molecular inicial:}
          \[
              \textcolor{red}{\text{C}_3\text{H}_8 (g) + \text{O}_2 (g) \rightarrow \text{CO}_2 (g) + \text{H}_2\text{O} (g)}
          \]
          \textcolor{blue}{Agora, verificamos o balanceamento dos átomos. No lado dos reagentes, temos 3 átomos de carbono \(\textcolor{red}{(\text{C})}\) no \(\textcolor{red}{\text{C}_3\text{H}_8}\) e 1 átomo de carbono \(\textcolor{red}{(\text{C})}\) no lado dos produtos. Para balancear o carbono, colocamos o coeficiente 3 na frente de \(\textcolor{red}{\text{CO}_2}\):}
          \[
              \textcolor{red}{\text{C}_3\text{H}_8 (g) + \text{O}_2 (g) \rightarrow 3 \, \text{CO}_2 (g) + \text{H}_2\text{O} (g)}
          \]
          \textcolor{blue}{Agora, verificamos o balanceamento do hidrogênio. No lado dos reagentes, temos 8 átomos de hidrogênio \(\textcolor{red}{(\text{H})}\) no \(\textcolor{red}{\text{C}_3\text{H}_8}\), e no lado dos produtos, temos 2 átomos de hidrogênio \(\textcolor{red}{(\text{H})}\) no \(\textcolor{red}{\text{H}_2\text{O}}\). Para balancear o hidrogênio, colocamos o coeficiente 4 na frente de \(\textcolor{red}{\text{H}_2\text{O}}\):}
          \[
              \textcolor{red}{\text{C}_3\text{H}_8 (g) + \text{O}_2 (g) \rightarrow 3 \, \text{CO}_2 (g) + 4 \, \text{H}_2\text{O} (g)}
          \]
          \textcolor{blue}{Por fim, verificamos o balanceamento do oxigênio. No lado dos reagentes, temos 2 átomos de oxigênio \(\textcolor{red}{(\text{O})}\) no \(\textcolor{red}{\text{O}_2}\), e no lado dos produtos, temos 10 átomos de oxigênio \(\textcolor{red}{(\text{O})}\) (6 no \(\textcolor{red}{3 \, \text{CO}_2}\) e 4 no \(\textcolor{red}{4 \, \text{H}_2\text{O}}\)). Para balancear o oxigênio, colocamos o coeficiente 5 na frente de \(\textcolor{red}{\text{O}_2}\):}
          \[
              \textcolor{red}{\text{C}_3\text{H}_8 (g) + 5 \, \text{O}_2 (g) \rightarrow 3 \, \text{CO}_2 (g) + 4 \, \text{H}_2\text{O} (g)}
          \]
          \textcolor{blue}{Agora, a equação está balanceada.}
          \[
              \textcolor{red}{\text{C}_3\text{H}_8 (g) + 5 \, \text{O}_2 (g) \rightarrow 3 \, \text{CO}_2 (g) + 4 \, \text{H}_2\text{O} (g)}
          \]
          \textcolor{blue}{\text{Massa molar do propano \(\textcolor{red}{\text{(C}_3\text{H}_8)}\):}}
          \[
              \textcolor{blue}{\text{Massa molar de } \, \textcolor{red}{\text{C}_3\text{H}_8} = 3 \times 12,01 + 8 \times 1,008 = 44,10 \, \text{g/mol}}
          \]
          \textcolor{blue}{\text{Número de mols de propano em 1,00 g:}}
          \[
              \textcolor{blue}{\text{Número de mols de } \, \textcolor{red}{\text{C}_3\text{H}_8} = \frac{1,00 \, \text{g}}{44,10 \, \text{g/mol}} = 0,0227 \, \text{mol}}
          \]
          \textcolor{blue}{\text{Relação estequiométrica entre \(\textcolor{red}{\text{C}_3\text{H}_8}\) e \(\textcolor{red}{\text{O}_2}\):}}
          \[
              \textcolor{blue}{\text{A equação balanceada mostra que 1 mol de } \, \textcolor{red}{\text{C}_3\text{H}_8} \, \text{ reage com 5 mols de } \, \textcolor{red}{\text{O}_2}.}
          \]
          \textcolor{blue}{\text{Número de mols de \(\textcolor{red}{\text{O}_2}\) consumidos:}}
          \[
              \textcolor{blue}{\text{Número de mols de } \, \textcolor{red}{\text{O}_2} = 0,0227 \, \text{mol} \times 5 = 0,1135 \, \text{mol}}
          \]
          \textcolor{blue}{\text{Massa molar do \(\textcolor{red}{\text{O}_2}\):}}
          \[
              \textcolor{blue}{\text{Massa molar de } \, \textcolor{red}{\text{O}_2} = 2 \times 16,00 = 32,00 \, \text{g/mol}}
          \]
          \textcolor{blue}{\text{Massa de \(\textcolor{red}{\text{O}_2}\) consumida:}}
          \[
              \textcolor{blue}{\text{Massa de } \, \textcolor{red}{\text{O}_2} = 0,1135 \, \text{mol} \times 32,00 \, \text{g/mol} = 3,63 \, \text{g}}
          \]
          \textcolor{blue}{\text{Resposta final:}}
          \[
              \textcolor{blue}{\text{A massa de } \, \textcolor{red}{\text{O}_2} \, \text{consumida na combustão de 1,00 g de propano é de } \, 3,63 \, \text{g}}.
          \]
          %RESPOSTA%


          % ======= Questão 13 ========= %

    \item O hidróxido de sódio reage com dióxido de carbono como a seguir:
          \[
              2\text{NaOH}(s) + \text{CO}_{2}(g) \rightarrow \text{Na}_{2}\text{CO}_{3}
              (s) + \text{H}_{2}\text{O}(l)
          \]
          Qual reagente é o limitante quando $1,70$ mol de $\text{NaOH}$ reage com
          $1,00$ mol de $\text{CO}_{2}$? Qual quantidade de matéria de $\text{Na}_{2}
              \text{CO}_{3}$ pode ser produzida? Qual quantidade de matéria do reagente
          em excesso sobra após a reação se completar?
          %RESPOSTA%
          \\[10pt]
          \textcolor{blue}{Primeiro, escrevemos a equação molecular:}
          \[
              \textcolor{red}{2 \, \text{NaOH} (s) + \text{CO}_2 (g) \rightarrow \text{Na}_2\text{CO}_3 (s) + \text{H}_2\text{O} (l)}
          \]
          \textcolor{blue}{Agora, verificamos a quantidade de matéria de cada reagente. Temos 1,70 mol de \(\textcolor{red}{\text{NaOH}}\) e 1,00 mol de \(\textcolor{red}{\text{CO}_2}\).}

          \textcolor{blue}{A equação balanceada mostra que 2 mols de \(\textcolor{red}{\text{NaOH}}\) reagem com 1 mol de \(\textcolor{red}{\text{CO}_2}\). Portanto, a quantidade de \(\textcolor{red}{\text{NaOH}}\) necessária para reagir com 1,00 mol de \(\textcolor{red}{\text{CO}_2}\) é:}
          \[
              \textcolor{blue}{\text{Quantidade de } \textcolor{red}{\text{NaOH}} = 1,00 \, \text{mol} \times 2 = 2,00 \, \text{mol}}
          \]
          \textcolor{blue}{Como temos apenas 1,70 mol de \(\textcolor{red}{\text{NaOH}}\), o \(\textcolor{red}{\text{NaOH}}\) é o reagente limitante.}

          \textcolor{blue}{Agora, calculamos a quantidade de matéria de \(\textcolor{red}{\text{Na}_2\text{CO}_3}\) que pode ser produzida. A equação balanceada mostra que 2 mols de \(\textcolor{red}{\text{NaOH}}\) produzem 1 mol de \(\textcolor{red}{\text{Na}_2\text{CO}_3}\). Portanto, a quantidade de \(\textcolor{red}{\text{Na}_2\text{CO}_3}\) produzida é:}
          \[
              \textcolor{blue}{2 \, \text{mol de } \, \textcolor{red}{\text{NaOH}} \rightarrow 1 \, \text{mol de } \, \textcolor{red}{\text{Na}_2\text{CO}_3}}
          \]
          \[
              \textcolor{blue}{1,70 \, \text{mol de } \, \textcolor{red}{\text{NaOH}} \rightarrow x \, \text{mol de } \, \textcolor{red}{\text{Na}_2\text{CO}_3}}
          \]
          \textcolor{blue}{Resolvendo a regra de três:}
          \[
              \textcolor{blue}{2x = 1,70  \times 1}
          \]
          \[
              \textcolor{blue}{x = \frac{1,70 \times 1 }{2} = 0,85 \, \text{mol de } \, \textcolor{red}{\text{Na}_2\text{CO}_3}}
          \]

          \textcolor{blue}{Por fim, calculamos a quantidade de matéria do reagente em excesso \(\textcolor{red}{(\text{CO}_2)}\) que sobra após a reação. A quantidade de \(\textcolor{red}{\text{CO}_2}\) que reage é:}
          \[
              \textcolor{blue}{2 \, \text{mols de } \, \textcolor{red}{\text{NaOH}} \rightarrow 1 \, \text{mol de } \,\textcolor{red}{ \text{CO}_2}}
          \]
          \[
              \textcolor{blue}{ 1,70 \, \text{mols de } \, \textcolor{red}{\text{NaOH}} \rightarrow x \, \text{mols de } \, \textcolor{red}{\text{CO}_2}}
          \]
          \textcolor{blue}{Resolvendo a regra de três:}
          \[
              \textcolor{blue}{2x = 1,70  \times 1}
          \]
          \[
              \textcolor{blue}{x = \frac{1,70  \times 1}{2} = 0,85 \, \text{mol de } \, \textcolor{red}{\text{CO}_2}}
          \]

          \textcolor{blue}{Portanto, a quantidade de \(\textcolor{red}{\text{CO}_2}\) em excesso é:}

          \[
              \textcolor{blue}{\text{Quantidade de } \textcolor{red}{\text{CO}_2} \text{ em excesso} = 1,00 \, \text{mol} - 0,85 \, \text{mol} = 0,15 \, \text{mol}}
          \]
          %RESPOSTA%



          % ======= Questão 14 ========= %

    \item A reação entre amônia e metano é catalisada por platina. Formam-se
          cianeto de hidrogênio e hidrogênio gasosos.
          \begin{enumerate}[align=left, labelsep=-0.5em]
              \item[a)] Escreva a equação química balanceada da reação.
                    %RESPOSTA%
                    \\[10pt]
                    \textcolor{blue}{Primeiro, escrevemos a equação molecular inicial:}
                    \[
                        \textcolor{red}{\text{CH}_4 (g) + \text{NH}_3 (g) \xrightarrow{\text{Pt}} \text{HCN} (g) + \text{H}_2 (g)}
                    \]
                    \textcolor{blue}{Agora, verificamos o balanceamento dos átomos.}
                    \\[10pt]
                    \textcolor{blue}{Começamos com o carbono. No lado dos reagentes, temos 1 átomo de carbono \(\textcolor{red}{(\text{C})}\) no \(\textcolor{red}{\text{CH}_4}\). No lado dos produtos, também temos 1 átomo de carbono no \(\textcolor{red}{\text{HCN}}\). Como a quantidade é igual, o carbono já está balanceado.}
                    \\[10pt]
                    \textcolor{blue}{Agora, analisamos o nitrogênio. No lado dos reagentes, temos 1 átomo de nitrogênio \(\textcolor{red}{(\text{N})}\) no \(\textcolor{red}{\text{NH}_3}\). No lado dos produtos, também há 1 átomo de nitrogênio no \(\textcolor{red}{\text{HCN}}\). O nitrogênio já está balanceado.}
                    \\[10pt]
                    \textcolor{blue}{Por fim, verificamos o hidrogênio. No lado dos reagentes, temos 4 átomos de hidrogênio no \(\textcolor{red}{\text{CH}_4}\) e 3 no \(\textcolor{red}{\text{NH}_3}\), totalizando 7 átomos de hidrogênio. No lado dos produtos, há 1 átomo de hidrogênio no \(\textcolor{red}{\text{HCN}}\) e moléculas de \(\textcolor{red}{\text{H}_2}\), que contêm 2 átomos cada. Para igualar os 7 átomos, colocamos o coeficiente 3 na frente do \(\textcolor{red}{\text{H}_2}\), garantindo 6 átomos de hidrogênio no \(\textcolor{red}{\text{H}_2}\) mais 1 no \(\textcolor{red}{\text{HCN}}\), totalizando 7.}
                    \\[10pt]
                    \textcolor{blue}{Agora, a equação balanceada é:}
                    \[
                        \textcolor{red}{\text{CH}_4 (g) + \text{NH}_3 (g) \xrightarrow{\text{Pt}} \text{HCN} (g) + 3 \text{H}_2 (g)}
                    \]
                    %RESPOSTA%

              \item[b)] Calcule as massas dos reagentes para a obtenção de $2,70$ kg de
                    cianeto de hidrogênio, supondo-se $80\%$ de rendimento da reação (massas
                    molares em g/mol: $\text{H}= 1$; $\text{C}= 12$; $\text{N}= 14$).
                    %RESPOSTA%
                    \\[10pt]
                    \textcolor{blue}{Primeiro, calculamos a quantidade de matéria de \(\textcolor{red}{\text{HCN}}\) a ser produzida. Sabemos que a massa de \(\textcolor{red}{\text{HCN}}\) é $2,70$ kg, ou seja, $2700$ g. A massa molar de \(\textcolor{red}{\text{HCN}}\) é:}
                    \[
                        \textcolor{blue}{\text{Massa Molar de } \textcolor{red}{\text{HCN}} = 12 + 1 + 14 = 27 \, \text{g/mol}}
                    \]
                    \textcolor{blue}{Agora, calculamos a quantidade de matéria de \(\textcolor{red}{\text{HCN}}\) necessária para produzir $2700$ g:}
                    \[
                        \textcolor{blue}{\text{Quantidade de matéria de } \textcolor{red}{\text{HCN}} = \frac{2700 \, \text{g}}{27 \, \text{g/mol}} = 100 \, \text{mol}}
                    \]
                    \textcolor{blue}{Como a reação tem $80\%$ de rendimento, a quantidade de \(\textcolor{red}{\text{HCN}}\) que teoricamente seria produzida em um rendimento de $100\%$ seria:}
                    \[
                        \textcolor{blue}{\text{Quantidade de matéria de } \textcolor{red}{\text{HCN}} = \frac{100 \, \text{mol}}{0,80} = 125 \, \text{mol}}
                    \]
                    \textcolor{blue}{Agora, utilizamos a equação balanceada para calcular as quantidades de \(\textcolor{red}{\text{CH}_4}\) e \(\textcolor{red}{\text{NH}_3}\) necessárias. A equação balanceada mostra que 1 mol de \(\textcolor{red}{\text{CH}_4}\) reage com 1 mol de \(\textcolor{red}{\text{NH}_3}\) para produzir 1 mol de \(\textcolor{red}{\text{HCN}}\). Portanto, a quantidade de \(\textcolor{red}{\text{CH}_4}\) e \(\textcolor{red}{\text{NH}_3}\) necessária para produzir $125$ mol de \(\textcolor{red}{\text{HCN}}\) é:}
                    \[
                        \textcolor{blue}{\text{Quantidade de matéria de } \textcolor{red}{\text{CH}_4} = 125 \, \text{mol}}
                    \]
                    \[
                        \textcolor{blue}{\text{Quantidade de matéria de } \textcolor{red}{\text{NH}_3} = 125 \, \text{mol}}
                    \]
                    \textcolor{blue}{Agora, calculamos as massas dos reagentes. A massa molar de \(\textcolor{red}{\text{CH}_4}\) é:}
                    \[
                        \textcolor{blue}{\text{Massa Molar de } \textcolor{red}{\text{CH}_4} = 12 + 4(1) = 16 \, \text{g/mol}}
                    \]
                    \textcolor{blue}{Portanto, a massa de \(\textcolor{red}{\text{CH}_4}\) necessária é:}
                    \[
                        \textcolor{blue}{\text{Massa de } \textcolor{red}{\text{CH}_4} = 125 \, \text{mol} \times 16 \, \text{g/mol} = 2000 \, \text{g} = 2,00 \, \text{kg}}
                    \]
                    \textcolor{blue}{A massa molar de \(\textcolor{red}{\text{NH}_3}\) é:}
                    \[
                        \textcolor{blue}{\text{Massa Molar de } \textcolor{red}{\text{NH}_3} = 14 + 3(1) = 17 \, \text{g/mol}}
                    \]
                    \textcolor{blue}{Portanto, a massa de \(\textcolor{red}{\text{NH}_3}\) necessária é:}
                    \[
                        \textcolor{blue}{\text{Massa de } \textcolor{red}{\text{NH}_3} = 125 \, \text{mol} \times 17 \, \text{g/mol} = 2125 \, \text{g} = 2,125 \, \text{kg}}
                    \]
                    \textcolor{blue}{Portanto, as massas dos reagentes necessárias para obter $2,70$ kg de \(\textcolor{red}{\text{HCN}}\) com $80\%$ de rendimento são:}
                    \[
                        \textcolor{blue}{\text{Massa de } \textcolor{red}{\text{CH}_4} = 2,00 \, \text{kg}}
                    \]
                    \[
                        \textcolor{blue}{\text{Massa de } \textcolor{red}{\text{NH}_3} = 2,125 \, \text{kg}}
                    \]
                    %RESPOSTA%

          \end{enumerate}

          % ======= Questão 15 ========= %

    \item A uréia - $\text{CO(NH}_{2})_{2}$ - é uma substância utilizada como fertilizante
          e é obtida pela reação entre o gás carbônico e amônia, conforme a equação:
          \[
              \text{CO}_{2}(g) + 2\text{NH}_{3}(g) \rightarrow \text{CO(NH}_{2})_{2}(s)
              + \text{H}_{2}\text{O}(g)
          \]
          Sabendo-se que $89,6$ litros de gás amônia reagem completamente no processo
          com o gás carbônico, nas CNTP, a massa de uréia, obtida em gramas, é igual
          a: Dados: $\text{C}= 12$; $\text{N}= 14$; $\text{O}= 16$; $\text{H}= 1$.
          %RESPOSTA%
          \\[20pt]
          \textcolor{blue}{Primeiro, calculamos a quantidade de matéria de \(\textcolor{red}{\text{NH}_3}\) que reage. Nas CNTP, 1 mol de gás ocupa 22,4 L.}

          $ \textcolor{blue}{\text{Portanto, a quantidade de matéria de } \textcolor{red}{\text{NH}_3} \text{ em 89,6 L é:}}$

          \[
              \textcolor{blue}{\text{Quantidade de matéria de } \textcolor{red}{\text{NH}_3} = \frac{89,6 \, \text{L}}{22,4 \, \text{L/mol}} = 4,00 \, \text{mol}}
          \]
          \textcolor{blue}{A equação balanceada mostra que 2 mols de \(\textcolor{red}{\text{NH}_3}\) reagem com 1 mol de \(\textcolor{red}{\text{CO}_2}\) para produzir 1 mol de \(\textcolor{red}{\text{CO(NH}_2)_2}\). Portanto, a quantidade de matéria de \(\textcolor{red}{\text{CO(NH}_2)_2}\) produzida é:}
          \[
              \textcolor{blue}{2 \, \text{mols de } \, \textcolor{red}{\text{NH}_3} \rightarrow 1 \, \text{mol de } \, \textcolor{red}{\text{CO(NH}_2)_2}}
          \]
          \[
              \textcolor{blue}{4,00 \, \text{mols de } \, \textcolor{red}{\text{NH}_3} \rightarrow x \, \text{mols de } \, \textcolor{red}{\text{CO(NH}_2)_2}}
          \]
          \textcolor{blue}{Resolvendo a regra de três:}
          \[
              \textcolor{blue}{2x = 4,00 \times 1}
          \]
          \[
              \textcolor{blue}{x = \frac{4,00 \times 1}{2} = 2,00 \, \text{mols de } \, \textcolor{red}{\text{CO(NH}_2)_2}}
          \]
          \textcolor{blue}{Agora, calculamos a massa de \(\textcolor{red}{\text{CO(NH}_2)_2}\) produzida. A massa molar de \(\textcolor{red}{\text{CO(NH}_2)_2}\) é:}
          \[
              \textcolor{blue}{\text{Massa Molar de } \textcolor{red}{\text{CO(NH}_2)_2} = 12 + 2(14 + 2(1)) + 16 = 60 \, \text{g/mol}}
          \]
          \textcolor{blue}{Portanto, a massa de \(\textcolor{red}{\text{CO(NH}_2)_2}\) produzida é:}
          \[
              \textcolor{blue}{\text{Massa de } \textcolor{red}{\text{CO(NH}_2)_2} = 2,00 \, \text{mol} \times 60 \, \text{g/mol} = 120 \, \text{g}}
          \]
          %RESPOSTA%
          % ======= Questão 16 ========= %

    \item Um produto comercial empregado na limpeza de esgotos contém pequenos pedaços
          de alumínio, que reagem com $\text{NaOH}$ para produzir bolhas de
          hidrogênio. A reação que ocorre é expressa pela equação:
          \[
              2\text{Al}+ 2\text{NaOH}+ 6\text{H}_{2}\text{O}\rightarrow 3\text{H}_{2}
              + 2\text{NaAlO}_{2}
          \]
          Calcular o volume de $\text{H}_{2}$, medido a $0^{\circ}$C e $1$ atmosfera
          de pressão,que será liberado quando $0,162$ g de alumínio reagirem totalmente.
          Massas atômicas: $\text{Al}= 27$; $\text{H}= 1$. Volume ocupado por $1$ mol
          do gás a $0^{\circ}$C e $1$ atmosfera $= 22,7$ litros.
          %RESPOSTA%
          \\[20pt]
          \textcolor{blue}{Primeiro, calculamos a quantidade de matéria de \(\textcolor{red}{\text{Al}}\) que reage.}
          \[
              \textcolor{blue}{\text{Quantidade de matéria de } \textcolor{red}{\text{Al}} = \frac{0,162 \, \text{g}}{27 \, \text{g/mol}} = 0,006 \, \text{mol}}
          \]
          \textcolor{blue}{A equação balanceada mostra que 2 mols de \(\textcolor{red}{\text{Al}}\) produzem 3 mols de \(\textcolor{red}{\text{H}_2}\). Portanto, a quantidade de matéria de \(\textcolor{red}{\text{H}_2}\) produzida é:}
          \[
              \textcolor{blue}{2 \, \text{mols de } \, \textcolor{red}{\text{Al}} \rightarrow 3 \, \text{mols de } \, \textcolor{red}{\text{H}_2}}
          \]
          \[
              \textcolor{blue}{0,006 \, \text{mol de } \, \textcolor{red}{\text{Al}} \rightarrow x \, \text{mols de } \, \textcolor{red}{\text{H}_2}}
          \]
          \textcolor{blue}{Resolvendo a regra de três:}
          \[
              \textcolor{blue}{2x = 0,006 \times 3}
          \]
          \[
              \textcolor{blue}{x = \frac{0,006 \times 3}{2} = 0,009 \, \text{mol de } \, \textcolor{red}{\text{H}_2}}
          \]
          \textcolor{blue}{Agora, calculamos o volume de \(\textcolor{red}{\text{H}_2}\) produzido. Nas condições normais de pressão e temperatura (CNTP), 1 mol de gás ocupa 22,7 L. Portanto, o volume de \(\textcolor{red}{\text{H}_2}\) produzido é:}
          \[
              \textcolor{blue}{\text{Volume de } \textcolor{red}{\text{H}_2} = 0,009 \, \text{mol} \times 22,7 \, \text{L/mol} = 0,2043 \, \text{L}}
          \]
          \textcolor{blue}{Portanto, o volume de \(\textcolor{red}{\text{H}_2}\) liberado é de 0,2043 L.}
          %RESPOSTA%
          % ======= Questão 17 ========= %

    \item O hidróxido de alumínio reage com ácido sulfúrico como a seguir:
          \[
              2\text{Al(OH)}_{3}(s) + 3\text{H}_{2}\text{SO}_{4}(aq) \rightarrow \text{Al}
              _{2}(\text{SO}_{4})_{3}(aq) + 6\text{H}_{2}\text{O}(l)
          \]
          Qual reagente é o limitante quando $0,450$ mol de $\text{Al(OH)}_{3}$
          reage com $0,55$ mol de $\text{H}_{2}\text{SO}_{4}$? Qual quantidade de matéria
          de $\text{Al}_{2}(\text{SO}_{4})_{3}$ pode ser formada sob essas condições?
          Qual quantidade de matéria do reagente em excesso sobra após a reação se
          completar?
          %RESPOSTA%
          \\[20pt]
          \textcolor{blue}{Primeiro, verificamos a quantidade de matéria de cada reagente. Temos 0,450 mol de \(\textcolor{red}{\text{Al(OH)}_{3}}\) e 0,55 mol de \(\textcolor{red}{\text{H}_{2}\text{SO}_{4}}\).}

          \textcolor{blue}{A equação balanceada mostra que 2 mols de \(\textcolor{red}{\text{Al(OH)}_{3}}\) reagem com 3 mols de \(\textcolor{red}{\text{H}_{2}\text{SO}_{4}}\). Portanto, a quantidade de \(\textcolor{red}{\text{H}_{2}\text{SO}_{4}}\) necessária para reagir com 0,450 mol de \(\textcolor{red}{\text{Al(OH)}_{3}}\) é:}
          \[
              \textcolor{blue}{\text{Quantidade de } \textcolor{red}{\text{H}_{2}\text{SO}_{4}} = 0,450 \, \text{mol} \times \frac{3}{2} = 0,675 \, \text{mol}}
          \]
          \textcolor{blue}{Como temos apenas 0,55 mol de \(\textcolor{red}{\text{H}_{2}\text{SO}_{4}}\), o \(\textcolor{red}{\text{H}_{2}\text{SO}_{4}}\) é o reagente limitante.}

          \textcolor{blue}{Agora, calculamos a quantidade de matéria de \(\textcolor{red}{\text{Al}_{2}(\text{SO}_{4})_{3}}\) que pode ser formada. A equação balanceada mostra que 3 mols de \(\textcolor{red}{\text{H}_{2}\text{SO}_{4}}\) produzem 1 mol de \(\textcolor{red}{\text{Al}_{2}(\text{SO}_{4})_{3}}\). Portanto, a quantidade de \(\textcolor{red}{\text{Al}_{2}(\text{SO}_{4})_{3}}\) produzida é:}
          \[
              \textcolor{blue}{3 \, \text{mols de } \, \textcolor{red}{\text{H}_{2}\text{SO}_{4}} \rightarrow 1 \, \text{mol} \, \textcolor{red}{\text{Al}_{2}(\text{SO}_{4})_{3}}}
          \]
          \[
              \textcolor{blue}{0,55 \, \text{mol} \, \textcolor{red}{\text{H}_{2}\text{SO}_{4}} \rightarrow x \, \text{mols de } \, \textcolor{red}{\text{Al}_{2}(\text{SO}_{4})_{3}}}
          \]
          \textcolor{blue}{Resolvendo a regra de três:}
          \[
              \textcolor{blue}{3x = 0,55 \times 1}
          \]
          \[
              \textcolor{blue}{x = \frac{0,55 \times 1}{3} = 0,183 \, \text{mol} \, \textcolor{red}{\text{Al}_{2}(\text{SO}_{4})_{3}}}
          \]

          \textcolor{blue}{Por fim, calculamos a quantidade de matéria do reagente em excesso \(\textcolor{red}{(\text{Al(OH)}_{3})}\) que sobra após a reação. A quantidade de \(\textcolor{red}{\text{Al(OH)}_{3}}\) que reage é:}
          \[
              \textcolor{blue}{2 \, \text{mols de } \, \textcolor{red}{\text{Al(OH)}_{3}} \rightarrow 3 \, \text{mols de } \, \textcolor{red}{\text{H}_{2}\text{SO}_{4}}}
          \]
          \[
              \textcolor{blue}{x \, \text{mols de } \, \textcolor{red}{\text{Al(OH)}_{3}} \rightarrow 0,55 \, \text{mol} \, \textcolor{red}{\text{H}_{2}\text{SO}_{4}}}
          \]
          \textcolor{blue}{Resolvendo a regra de três:}
          \[
              \textcolor{blue}{3x = 0,55 \times 2}
          \]
          \[
              \textcolor{blue}{x = \frac{0,55 \times 2}{3} = 0,367 \, \text{mol} \, \textcolor{red}{\text{Al(OH)}_{3}}}
          \]

          \textcolor{blue}{Portanto, a quantidade de \(\textcolor{red}{\text{Al(OH)}_{3}}\) em excesso é:}
          \[
              \textcolor{blue}{\text{Quantidade de } \textcolor{red}{\text{Al(OH)}_{3}} \text{ em excesso} = 0,450 \, \text{mol} - 0,367 \, \text{mol} = 0,083 \, \text{mol}}
          \]
          %RESPOSTA%

          % ======= Questão 18 ========= %

    \item O propano ($\text{C}_{3}\text{H}_{8}$) é um componente do gás natural utilizado
          para cozinhar ou aquecer ambientes.
          \begin{enumerate}[align=left, labelsep=-0.5em]
              \item[a)] Escreva a equação balanceada que representa a combustão do propano
                    no ar.
                    %RESPOSTA%
                    \\[10pt]
                    \textcolor{blue}{Primeiro, escrevemos a equação molecular inicial:}
                    \[
                        \textcolor{red}{\text{C}_{3}\text{H}_{8} (g) + \text{O}_{2} (g) \rightarrow \text{CO}_{2} (g) + \text{H}_{2}\text{O} (g)}
                    \]
                    \textcolor{blue}{Agora, verificamos o balanceamento dos átomos. No lado dos reagentes, temos 3 átomos de carbono \(\textcolor{red}{(\text{C})}\) no \(\textcolor{red}{\text{C}_{3}\text{H}_{8}}\) e 1 átomo de carbono \(\textcolor{red}{(\text{C})}\) no lado dos produtos. Para balancear o carbono, colocamos o coeficiente 3 na frente de \(\textcolor{red}{\text{CO}_{2}}\):}
                    \[
                        \textcolor{red}{\text{C}_{3}\text{H}_{8} (g) + \text{O}_{2} (g) \rightarrow 3 \, \text{CO}_{2} (g) + \text{H}_{2}\text{O} (g)}
                    \]
                    \textcolor{blue}{Agora, verificamos o balanceamento do hidrogênio. No lado dos reagentes, temos 8 átomos de hidrogênio \(\textcolor{red}{(\text{H})}\) no \(\textcolor{red}{\text{C}_{3}\text{H}_{8}}\), e no lado dos produtos, temos 2 átomos de hidrogênio \(\textcolor{red}{(\text{H})}\) no \(\textcolor{red}{\text{H}_{2}\text{O}}\). Para balancear o hidrogênio, colocamos o coeficiente 4 na frente de \(\textcolor{red}{\text{H}_{2}\text{O}}\):}
                    \[
                        \textcolor{red}{\text{C}_{3}\text{H}_{8} (g) + \text{O}_{2} (g) \rightarrow 3 \, \text{CO}_{2} (g) + 4 \, \text{H}_{2}\text{O} (g)}
                    \]
                    \textcolor{blue}{Por fim, verificamos o balanceamento do oxigênio. No lado dos reagentes, temos 2 átomos de oxigênio \(\textcolor{red}{(\text{O})}\) no \(\textcolor{red}{\text{O}_{2}}\), e no lado dos produtos, temos 10 átomos de oxigênio \(\textcolor{red}{(\text{O})}\) (6 no \(\textcolor{red}{3 \, \text{CO}_{2}}\) e 4 no \(\textcolor{red}{4 \, \text{H}_{2}\text{O}}\)). Para balancear o oxigênio, colocamos o coeficiente 5 na frente de \(\textcolor{red}{\text{O}_{2}}\):}
                    \[
                        \textcolor{red}{\text{C}_{3}\text{H}_{8} (g) + 5 \, \text{O}_{2} (g) \rightarrow 3 \, \text{CO}_{2} (g) + 4 \, \text{H}_{2}\text{O} (g)}
                    \]
                    \textcolor{blue}{Agora, a equação está balanceada.}
                    \[
                        \textcolor{red}{\text{C}_{3}\text{H}_{8} (g) + 5 \, \text{O}_{2} (g) \rightarrow 3 \, \text{CO}_{2} (g) + 4 \, \text{H}_{2}\text{O} (g)}
                    \]
                    %RESPOSTA%

              \item[b)] Quantos gramas de dióxido de carbono são produzidos pela combustão
                    de $3,65$ mols de propano? Considere que o oxigênio é o reagente em
                    excesso nesta reação. Dados: Massa molar (g/mol): $\text{O}= 16,00$;
                    $\text{H}= 1,01$; $\text{C}= 12,01$.
                    %RESPOSTA%
                    \\[10pt]
                    \textcolor{blue}{Primeiro, utilizamos a equação balanceada para calcular a quantidade de matéria de \(\textcolor{red}{\text{CO}_{2}}\) produzida. A equação balanceada mostra que 1 mol de \(\textcolor{red}{\text{C}_{3}\text{H}_{8}}\) produz 3 mols de \(\textcolor{red}{\text{CO}_{2}}\). Portanto, a quantidade de \(\textcolor{red}{\text{CO}_{2}}\) produzida é:}
                    \[
                        \textcolor{blue}{1 \, \text{mol} \, \textcolor{red}{\text{C}_{3}\text{H}_{8}} \rightarrow 3 \, \text{mols de } \, \textcolor{red}{\text{CO}_{2}}}
                    \]
                    \[
                        \textcolor{blue}{3,65 \, \text{mols de } \, \textcolor{red}{\text{C}_{3}\text{H}_{8}} \rightarrow x \, \text{mols de } \, \textcolor{red}{\text{CO}_{2}}}
                    \]
                    \textcolor{blue}{Resolvendo a regra de três:}
                    \[
                        \textcolor{blue}{x = 3,65 \times 3 = 10,95 \, \text{mol} \, \textcolor{red}{\text{CO}_{2}}}
                    \]

                    \textcolor{blue}{Agora, calculamos a massa de \(\textcolor{red}{\text{CO}_{2}}\) produzida. A massa molar de \(\textcolor{red}{\text{CO}_{2}}\) é:}
                    \[
                        \textcolor{blue}{\text{Massa Molar de } \textcolor{red}{\text{CO}_{2}} = 12,01 + 2(16,00) = 44,01 \, \text{g/mol}}
                    \]
                    \textcolor{blue}{Portanto, a massa de \(\textcolor{red}{\text{CO}_{2}}\) produzida é:}
                    \[
                        \textcolor{blue}{\text{Massa de } \textcolor{red}{\text{CO}_{2}} = 10,95 \, \text{mol} \times 44,01 \, \text{g/mol} = 481,04 \, \text{g}}
                    \]
                    \textcolor{blue}{Portanto, a massa de \(\textcolor{red}{\text{CO}_{2}}\) produzida pela combustão de $3,65$ mols de propano é de $481,04$ g.}
                    %RESPOSTA%
          \end{enumerate}
\end{enumerate}
\end{document}